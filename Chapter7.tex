\chapter{Conclusion}\label{c:Conclusion}
What has the thesis argued
unified framework of describing epistemics as functionally motivated by a need to establish a shared deictic ground in terms of relationships to knowledge
can be analysed in terms of the strength of the claim over eipstemic authority
    ability to make this claim is conditioned by varied factors across languages.

How does this work fit into the broader picture
Literature is slowly shifting towards more unified analyses regardless - citations given
Overlaps are a point of discussion
Chapter 6 also needs to fit in here

Why is it important 
platform for describing and comparing forms and functions across boundaries
grounded in real data - this is how things are actually happening


Limitations
doesn't necessarily replace more specific frameworks and descriptive categories
early stages theoretical analysis, more data and more detailed data specifically on these things around the world will shed more light
    as in, not data from comprehensive grammars with no specific focus on this area

Further research
as above