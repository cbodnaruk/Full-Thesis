\chapter{Methodology}\label{c:Methods}
\section{Overview}
This chapter describes the data collection and analysis methodologies used in this project, specifically the features used to summarise data. Section \ref{s:Methods:Collection} will discuss the collection of data, including the steps taken to ensure a representative cross-section of the \lfam\ family. Within this, Section \ref{ss:Methods:Bayesian} compares the distribution of languages sampled with some other proposed phylogenies, in particular those developed through Bayesian analyses in \citesA{Sagart2019Baye}{ZhangM2019Baye}{ZhangH2020Baye}. It will also consider the usefulness of these methodologies and their results both in the development of a representative sample and in linguistics in general. A number of cases where there are multiple conflicting analyses are discussed in Section \ref{ss:Description:Conflicts}, including the decision-making process behind which analysis was ultimately used in this survey. Section \ref{s:Methods:Schema} will address the specific features used to separate and categorise the languages surveyed. The resulting findings will subsequently be presented in Chapter \ref{c:Description}.

In order to ensure better coverage of the \lfam\ family in this survey (discussed further in Section \ref{ss:Methods:RepSample}), some fieldwork was conducted to fill in a gap in the available data, specifically on the Lhokpu language. Descriptive publication on the language are limited to \citeA{Grollmann2018}, a comparison of a word list and some basic grammatical forms with Dhimal and Toto (Dhimalish: Nepal and India respectively), which did not provide enough detail to include it in the survey. This fieldwork, detailed in Section \ref{s:Methods:FieldMethods}, allowed for the language to be included in the survey though its findings are at this stage still very preliminary.  

\section{Collection}\label{s:Methods:Collection}
\subsection{Developing a Representative Sample}\label{ss:Methods:RepSample}
Section \ref{ss:THOverview:Subfamilies} introduced the set of subfamilies proposed by \citeA{VanDriem2014} to represent a phylogeny-agnostic view of the \lfam\ family. Using this set of subfamilies as a foundation for the selection of languages provides a solid foundation to ensure no language groups are missed, while also avoiding taking a position on the debates in the literature where it is likely not necessary or helpful. That being said, simply selecting languages from each subfamily in equal number will not necessarily solve the issues in representativeness surrounding the statistical studies discussed in Section \ref{ss:THOverview:HighLevelStructure}.

\citeA{VanDriem2014}, while reflective of the state of description at its original time of writing, is at risk of being out of date by the current view of the family. For instance, more recent documentation work on Lhokpu, focussing on vocabulary and some verbal and nominal affixation, has suggested that the language is likely closely related to the Dhimalish languages of Dhimal and Toto \cite{Grollmann2018}. In addition to the linguistic evidence, the relationship is fairly easily justified geographically -- Lhokpu is spoken in about three Chiwogs or village blocks (Singye, Sangloong Sangteng, and Thongsa Tobchhenthang) in Bhutan's Samtse Dzongkhag, which are between 15 and 30 kilometres upstream of Totopara in India, where the Toto language is spoken \cite{Basumatary2016}. Both languages are geographically not contiguous, however, with Dhimalish, to which, interestingly, \citeA{Grollmann2018} suggest that Lhokpu appears more closely related. \citeA{Grollmann2018} also note that the Dhimalish subfamily including Dhimal and Toto is perhaps not as well proven or established as its inclusion in \citeA{VanDriem2014} might suggest, specifically that many of the shared forms are in fact shared much more widely than just Dhimal and Toto, and might therefore simply represent shared retentions from a much earlier proto-language. The geographic proximity of Lhokpu to Toto may also have little meaning for the historical development of the languages, as \citeA{VanDriem2004} suggests that Lhokpu (or its ancestor) may have at one time been much more widespread across Western Bhutan, and may in fact be a substrate under Dzongkha.

\citeA{Post2017} note that three further subfamilies, all spoken in Arunachal Pradesh in the Eastern Himalayas, remain merely speculative. Van Driem's (2014) Hrusish, Siangic, and Midzuish subfamilies may also not yet be sufficiently proven. While, in every case, there is some level of evidence to support the groupings, Hrusish and Midzuish could to an extent be simply explained by high degrees of contact, and all three subgroups are lacking the level of description necessary to confidently prove any subgroups \cite{Post2017}. \citeA{BlenchPost2014} go even further, questioning whether or not it has been sufficiently proven that Siangic, Hrusish, Midzuish, Digarish, and Kho-Bwa are even \lfam\ languages at all. \citeA{Wu2022} use Bayesian phylogenetic analysis focussing on the languages of Arunachal Pradesh to attempt to shed some light on these groups (excluding the Siangic group, which was not included), and attempt to position them both as members of the family in general, and also in relation to nearby neighbouring languages. They ultimately conclude that they mostly likely are related to the wider \lfam\ family, though diverged very early. Specifically, the Kho-Bwa languages could share a common ancestor to a time-depth of approaching 3000 years before present, similar to that of the Ngwi-Burmese subfamily, and their divergence from their closest related subfamily of the Hrusish languages could have been yet earlier, at around 5500 years before present \cite{Wu2022}. Following this research, among others, the most recent update of Glottolog (at time of writing, 5.0) \cite{glottolog} has separated the Hruso language from the \lfam\ language family entirely, keeping the other members of van Driem's Hrusish family (Bangru and Sajolang) as a subfamily within the family.

Van Driem (2014) also groups together the Tibetic and East Bodish groups into a single Bodish subfamily. While it is clear that East Bodish languages are not a subgroup of Tibetic languages \cite{Hyslop2017}, both \citeA{ZhangM2019Baye} and \citeA{ZhangH2020Baye}, as well as wider literature, seem to agree that Tibetic languages and East Bodish languages do in turn share a fairly close common ancestor and can be rightly considered to share a branch. The decision of what level to draw the line at here seems similarly arbitrary as it does with the Ngwi-Burmese subfamily, especially given some uncertainty surrounding the membership of Tshangla in this combined Bodish subfamily \cite{Thurgood2017STIntro}. Largely because of the large amount of literature available on the Tibetic\footnote{The term Tibetic is used here per \citeA{Tournadre2014}.} and East Bodish languages, and the availability of such data and insights to me from Gwendolyn Hyslop as my lead supervisor, East Bodish and Tibetic are separated here, contra \citeA{VanDriem2014}. 

The challenges in producing a representative sample of the languages in the \lfam\ family are a core problem in statistical models of \citesA{Sagart2019Baye}{ZhangM2019Baye}{ZhangH2020Baye}, but are also an issue that must be faced, to some extent in this project. In order to be able to make claims about the \lfam\ family as a whole, an even spread of the languages in the family need to be examined. This selection, however, arguably does not need to be as precise when attempting to analyse a smaller set of languages manually, than with the much larger scale computational analysis undertaken in \citesA{Sagart2019Baye}{ZhangM2019Baye}{ZhangH2020Baye}. That is, this project needs to consider data from the whole family to the extent that clades of divergent or conservative languages are not missed or misinterpreted, but is not attempting to develop any statistical measures of the language family whereby a greater level of coverage in one area might skew results. For example, by virtue of its older academic tradition, there is a much wider field of literature covering specifically Tibetic languages, with specific detail into the field of this project (such as \citesA{Garrett2001}{GarfieldDeVilliers2017}{Woodbury1986}{DeLancey1986}{ZeislerForthcoming}, among countless others), but little more than grammatical sketches in other areas. While this imbalance in available data would be unrepresentative in a statistical analysis, the less quantitative and more qualitative approach of this project allows for a greater ability to account for these potential biases.

While the necessity to build a dataset that is as representative as possible is not nearly as strong with this project as with some others, the methodology by which the subfamilies were selected for \citeA{VanDriem2014} does lend itself to prioritizing smaller, less researched language groups. Because it is focussed on well established subfamilies, and given that language groups with higher levels of research will likely have genetic relationships established to a deeper time depth or to a higher level, we can expect to see large groups of well researched languages given under a single subfamily, while smaller groups of underresearched languages will be listed separately. That is to say, there is not necessarily a similar amount of diversity within each of van Driem's (2014) subfamilies, but that there is likely more diversity in the larger, more widely researched and therefore better established ones.

In practise, the effect of this is visible in the number of languages present in each subfamily, using Glottolog \cite{glottolog} as a guide. At the largest end of the scale is the Ngwi-Burmese subfamily, with 127 languages listed on Glottolog, the majority of which (101) are specifically from the subfamily's Ngwi (Loloish) branch. Behind this, with 54 languages, is the Kukish branch. The language counts for the other subfamilies are given in \tabref{t:Methods:SubfamilyLanguageCount}.

\begin{longtable}{r c}
    Subfamily & Number of languages \\
    \hline
    Lepcha  & 1  \\
    \hline
    Meithei & 1   \\
    \hline
    Tshangla    & 1  \\
    \hline
    Lhokpu\footnote{Recent research suggests that Lhokpu may be closely related to the Dhimalish languages \cite{Grollmann2018}.}  & 1  \\
    \hline
    Dura    & 1   \\
    \hline
    Black Mountain  & 1  \\
    \hline
    Pyu & 1  \\
    \hline
    Gongduk & 1   \\
    \hline
    Mru\footnote{Some evidence suggests that Hkongso may be more closely related to Mru \cite{Wright2009}.} & 1 \\
    \hline
    Tujia\footnote{Glottolog gives north and south varieties, though \citeA{VanDriem2014} only gives one language.}   & 2  \\
    \hline
    Magaric & 2  \\
    \hline
    Chepangic   & 2   \\
    \hline
    Digarish    & 2   \\
    \hline
    Raji-Raute  & 2    \\
    \hline
    Siangic & 2 \\
    \hline
    Midzuish\footnote{Called Kman-Meyor in Glottolog.}    & 2 \\
    \hline
    Karbi\footnote{Glottolog gives Hills and Amri varieties, though \citeA{VanDriem2014} only gives one language.}   & 2 \\
    \hline

    Dhimalish   & 2-3 \\
    \hline
    Ersuish & 3   \\
    
    \hline
    Hrusish\footnote{The Hruso language has been separated in Glottolog 5.0 to be a complete isolate, though Bangru and Sajolang remain under the label Miji.} & 3  \\
    \hline
    Nungish & 3    \\
    \hline
    Bái\footnote{Membership of Caijia and Longjia unclear \cite{Lue2022}.}    & 3-6  \\
    \hline
    Newaric & 5    \\
    \hline
    Nàic    & 5    \\
    \hline
    East Bodish\footnote{Grouped with Tibetic into ``Bodish'' in \citeA{VanDriem2014}.} & 7    \\
    \hline
    Ao  & 7    \\
    \hline
    Kho-Bwa & 7 \\
    \hline
    Zeme    & 7    \\
    \hline
    rGyalrongic & 8   \\
    \hline
    Kachinic\footnote{Called Jingpho-Luish in Glottolog}    & 9   \\
    \hline
    Tangkhul    & 9  \\
    \hline
    Tani    & 10 \\
    \hline
    Angami-Pochuri  & 10 \\
    \hline
    Qiangic & 12 \\
    \hline
    Tamangic    & 13  \\
    \hline
    West Himalayish & 15 \\
    \hline
    Karenic & 20 \\
    \hline
    Sinitic & 26  \\
    \hline
    Brahmaputran\footnote{The Glottolog ``Brahmaputran'' branch also includes Kachinic/Jingpho-Luish, which \citeA{VanDriem2014} separates.}    & 29  \\
    \hline
    Kiranti & 31 \\
    \hline
    Tibetic  & 44  \\
    \hline
    Kukish  & 54    \\
    \hline
    Ngwi-Burmese & 127 \\
    \hline
    \caption{The number of languages in each of van Driem's (2014) subfamilies, per Glottolog \cite{glottolog}.}\label{t:Methods:SubfamilyLanguageCount}
    \end{longtable}


Languages themselves were selected based on available documentation, using Glottolog \cite{glottolog} as a reference database for descriptive work available. To avoid accidentally selecting a particularly abberant language and not properly representing a given subfamily, an initial goal of two languages per subfamily was set. For larger subfamilies, such as Ngwi-Burmese, Kukish, Tibetic, and Kiranti, more languages were surveyed both in an attempt to ensure even coverage of the larger subfamily, as well as because, in general research, more of these languages were referenced and therefore considered.

There were a number of cases where it was possible to even survey two languages in a subfamily. The first case is, of course, the internal isolates discussed in Section \ref{ss:THOverview:Subfamilies}. In addition, there are a number of smaller subfamilies which do not have the descriptive coverage to allow two languages to be surveyed. Either there was only one described language in the subfamily, or there were multiple, but only one has any coverage of epistemic marking. This last point poses a problem, in that, in languages with more limited descriptive analyses available, it is difficult to tell if reference to epistemic marking as studied in this project is omitted as it would be outside the scope of the current stage of description, or because it simply does not exist in the language. There were also a small number of subfamilies for which no descriptive data could be found. The first of these, Lhokpu, will be discussed in greater detail below as the opportunity arose in the course of this project to undertake fieldwork in a Lhokpu community in Bhutan and fill the gap in the data (though, as mentioned above, Lhokpu may well be better classified as Dhimalish). Gongduk, another internal isolate spoken in Bhutan \cite{VanDriem2001b}, also had no available data. During the course of this project, a grammar of Raji (Raji-Raute: Nepal) appears to have been published, written by Dubi Nanda Dhakal, but I have not been able to find this. There is similarly no documentation available for the Digarish languages in English, and as such my information on these languages comes from personal communication with Rolf Hotz Molina and Naomi Peck, who are both working on PhD projects on these languages. For Digarish, and Midzuish, there are unpublished sketch grammars by Roger Blench available on his website, though these are specifically not for wider use so were not included in the survey.\todo{Should I email Roger about these?} Glottolog also notes a body of descriptive work on Kaman (Midzuish) in Mandarin, which was also not accessible for this project. Black Mountain Mönpa has a sketch grammar by \citeA{Hyslop2016}, which makes no reference to any forms relevant to this project. It is, however, not clear if this is as they do not occur, or simply have not yet been sufficiently described.

With this goal of two languages per subfamily where possible, to be expanded upon after in larger subfamilies, languages were selected by the breadth of description available, as well as the recency of the description. That is, full published grammars were taken over doctoral and masters theses, and newer studies were taken over older ones. Published grammars were preferred as they will have gone through a more complete review process, and are often more in-depth than doctoral and especially masters theses. In any case, regardless of the level of review or detail for the publication, analyses were taken (where no alternative analyses exist, see Sunwar discussed in Section \ref{ss:Description:Conflicts}) to be accurate, and no attempts were made to reanalyse data presented. In some cases, the usage of terminology is discussed in relation to the description provided, and, especially in Section \ref{ss:Description:ClassByFunction} and Chapter \ref{c:Discussion}, theoretical conclusions are drawn about data based on the analysis available, but beyond what is explicitly stated. In this latter case, it is possible that there is further data which would prove wrong the analysis synthesised here, but which were not included in the publication as they were not seen to be necessary by the original author.

Newer studies were preferred as they are more likely to discuss the categories and functions at issue in this project. As discussed in Chapter \ref{c:Introduction}, studies into perspective-marking in \lfam\ languages have become significantly more common over the past two decades, and much of the research undertaken prior to that, and even more so prior to the publication of \citeA{ChafeNichols1986}, either does not consider perspective-marking at all, or does so in a way that is less immediately accessible in the context of contemporary theories and frameworks. That is to say that it is often simply much easier to find the relevant information in a more recent publication.

As was mentioned above, some further data has been used from languages that appear regularly or in relevant places throughout the literature, even if they were not initially selected. In particular, discussions of the epistemic system in Lhasa Tibetan and early Tibetic languages are common throughout the literature (see, for instance, \citesA{DeLancey2012}{Garrett2001}{Hill2012}{Hill2014}{Zemp2021}), an area of study which was particularly relevant to the diachronic considerations discussed in Chapter \ref{c:History}. These languages are in addition to the set of languages surveyed systematically here, and are not listed in \appref{a:TableOfLanguages} as such.

\subsection{Comparison to Proposed Phylogenies}\label{ss:Methods:Bayesian}
\todo{Note to Gwen and Lila: This section has sort of shifted from my original intention of comparing my sample with the trees presented here, as I wasn't sure how best to visually represent it. I also found I had more to say about the papers methodologically than I initially thought, so it's still a bit of a mess.}
While, as a result of the uncertainty surround the actual historically accurate phylogeny of the \lfam\ family, it is not possible to compare the sample of languages selected for this project to the actual family to gauge its representativeness (this impossibility of course beign the primary reason for the use of the subfamilies proposed by \citeA{VanDriem2014}), it is possible to compare the sample to the various proposed phylogenies and assess how representative the sample is for each.

Given their disagreement, and the discussion thus far about their methodology, this section will compare the phylogenies provided in \citesA{Sagart2019Baye}{ZhangM2019Baye}{ZhangH2020Baye}. Some other proposed phylogenies, such as that put forward in \citeA{STEDT}, are not included here as they remain largely agnostic about the family's structure outside of a few primary divisions. In Figures \ref{f:Methods:Sagart}-\ref{f:Methods:Zhang2020}, the trees given in their respective sources are reproduced to the subfamily level. That is, the branches are reproduced exactly, but rather than terminating in individual languages (as in \citesA{Sagart2019Baye}{ZhangM2019Baye}{ZhangH2020Baye}), or in other subfamilies, the branches terminate in the equivalent subfamily from the set used in this project given in Table \ref{t:Methods:SubfamilyLanguageCount}. While the branching structure of the subfamilies is accurately reproduced here, the time depth of the various divergences given in the Bayesian analyses is not.

Not every subfamily is represented in every tree, a point which forms part of the criticism of the Bayesian analyses. This occurs for these analyses when a given sample did not include any data from a given subfamily, and therefore does not represent or account for it in the final result. Missing subfamilies for each of \citesA{Sagart2019Baye}{ZhangM2019Baye}{ZhangH2020Baye} are given in Table \ref{t:Methods:BayesMissingSubfamilies}.

\begin{figure}
\centering
  \scalebox{0.35}{\begin{forest} 
   phylogeny,
[\lfam 
[
  [,tikz={\node [draw,red,fit to=tree] {};}
    [Kachinic]
    [Brahmaputran]
  ]
  [Sinitic,tikz={\node [draw,red,fit to=tree] {};}]
]
[
  [
    [Karbi]
    [,tikz={\node [draw,red,fit to=tree] {};}
      [Tangkhul]
      [Kukish]
    ]
  ]
  [
    [
      [Chepang]
      [
        [Tshangla]
        [,tikz={\node [draw,red,fit to=tree] {};}
          [Tani]
          [Digarish]
        ]
      ]
      [Kiranti,tikz={\node [draw,red,fit to=tree] {};}]
    ]
    [
      [W. Himalayish,tikz={\node [draw,red,fit to=tree] {};}]
      [
        [Nungish]
        [,tikz={\node [draw,red,fit to=tree] {};}
          [Tibetic]
          [
            [
              [Qiangic]
              [rGyalrongic]
            ]
            [Ngwi-Burmese]
          ]
        ]
      ]
    ]
  ]
]
]
\end{forest}}
\caption{The \lfam\ family as per \citeA{Sagart2019Baye}, showing the positions of the subfamilies used in this project. Branches with over 80\% posterior probability (that is, branches with high confidence, see ) have been marked with a red box. Time depth has not been reproduced in this tree.}
\label{f:Methods:Sagart}
\end{figure}

\begin{figure}
  \centering
  
    \scalebox{0.3}{\begin{forest} 
    phylogeny,
  [\lfam 
      [Sinitic,tikz={\node [draw,red,fit to=tree] {};}]
      [
          [
            [Karen]
            [,tikz={\node [draw,red,fit to=tree] {};}
              [Kukish]
              [
                [Ao]
                [
                  [
                    [Tangkhul]
                    [Zeme]
                  ]
                  [Angami-Pochuri]
                ]
              ]
            ]
          ]
          [
            [,tikz={\node [draw,red,fit to=tree] {};}
              [Kachinic]
              [Brahmaputran]
            ]
            [
              [
                [Digarish]
                [Tani]
              ]
              [
                [
                  [
                    [Kiranti]
                    [,tikz={\node [draw,red,fit to=tree] {};}
                      [Chepang]
                      [Magar]
                    ]
                  ]
                  [Nungish,tikz={\node [draw,red,fit to=tree] {};}]
                ]
                [
                  [West Himalayish]
                  [
                    [,tikz={\node [draw,red,fit to=tree] {};}
                      [Tamangic]
                      [
                        [Tibetic]
                        [East Bodish]
                      ]
                    ]
                    [,tikz={\node [draw,red,fit to=tree] {};}
                      [
                        [Nàic]
                        [
                          [Ersuish]
                          [
                            [Qiangic*]
                            [rGyalrongic*]
                          ]
                        ]
                      ]
                      [Ngwi-Burmese]
                    ]
                  ]
                ]
              ]
            ]
          ]
      ]
  ]
  \end{forest}}
  \caption{The \lfam\ family as per \citeA{ZhangM2019Baye}. *The phylogeny here does not in fact support these two subfamilies, but rather gives two rGyalrongic languages -- rGyalrong Maerkang (glottolog: Situ) and Caodeng (glottolog: Tshobdun) -- as an outgroup to the Qiangic clade, and two others -- Daofu and Ergong Danba (glottolog: both under Stau-dGebshes) are given as members of one of two Qiangic branches. Branches with over 80\% posterior probability have been marked with a red box. Time depth has not been reproduced in this tree.}\label{f:Methods:Zhang2019}
  \end{figure}

  \begin{sidewaysfigure}
    \centering
    
      \scalebox{0.35}{\begin{forest} 
phylogeny,
    [\lfam 
      [
        [Sinitic,tikz={\node [draw,red,fit to=tree] {};}]
        [
          [,tikz={\node [draw,red,fit to=tree] {};}
            [Lepcha]
            [Kiranti]
          ]
          [
            [
              [W. Himalayish]
              [
                [Digarish,tikz={\node [draw,red,fit to=tree] {};}]
                [,tikz={\node [draw,red,fit to=tree] {};}
                  [Siangic]
                  [Tani]
                ]
              ]
            ]
            [
              [,tikz={\node [draw,red,fit to=tree] {};}
                [Kachinic]
                [Brahmaputran]
              ]
              [
                [Kho-Bwa,tikz={\node [draw,red,fit to=tree] {};}]
                [Hrusish,tikz={\node [draw,red,fit to=tree] {};}]
              ]
            ]
            [
              [
                [
                  [
                    [Magaric*]
                    [Karen,tikz={\node [draw,red,fit to=tree] {};}]
                  ]
                  [
                    [Dhimalish]
                    [
                      [Magaric*]
                      [Chepangic]
                    ]
                  ]
                ]
                [
                  [Karbi]
                  [
                    [Kukish,tikz={\node [draw,red,fit to=tree] {};}]
                    [
                      [Meithei]
                      [,tikz={\node [draw,red,fit to=tree] {};}
                        [
                          [Angami-Pochuri**]
                          [Ao]
                        ]
                        [
                          [Tangkhul]
                          [
                            [Zeme]
                            [Angami-Pochuri**]
                          ]
                        ]
                      ]
                    ]
                  ]
                ]
              ]
              [
                [
                  [Midzuish]
                  [Nungish,tikz={\node [draw,red,fit to=tree] {};}]
                ]
                [
                  [
                    [Tamangic,tikz={\node [draw,red,fit to=tree] {};}]
                    [
                      [Tshangla]
                      [,tikz={\node [draw,red,fit to=tree] {};}
                        [East Bodish]
                        [Tibetic]
                      ]
                    ]
                  ]
                  [
                    [Tujia]
                    [
                      [
                        [rGyalrongic***]
                        [,tikz={\node [draw,red,fit to=tree] {};}
                          [
                            [Ngwi-Burmese†]
                            [Ersuish]
                          ]
                          [
                            [Qiangic]
                            [rGyalrongic***]
                          ]
                        ]
                      ]
                      [Ngwi-Burmese,tikz={\node [draw,red,fit to=tree] {};}]
                    ]
                  ]
                ]
              ]
            ]
          ]
        ]
      ]
    ]
    \end{forest}}
    \caption{The \lfam\ family as per \citeA{ZhangH2020Baye}. In this analysis, a number of van Driem's (2014) subfamilies are divided, with either single or multiple languages separated from the rest of their subfamily. These have been marked (*, **, ***, †) and are discussed in the text. Branches with over 80\% posterior probability have been marked with a red box. Time depth has not been reproduced in this tree.}\label{f:Methods:Zhang2020}
    \end{sidewaysfigure}

As discussed above, for each of the terminal nodes given in Figures \ref{f:Methods:Sagart}-\ref{f:Methods:Zhang2020} (that is, van Driem's fallen leaves), 1-3 languages has been sampled. While initially the intention of presenting Figures \ref{f:Methods:Sagart}-\ref{f:Methods:Zhang2020} was to compare the representativeness of this project's sample against their proposals for a phylogeny of the \lfam\ family, it is clear from \tabref{t:Methods:BayesMissingSubfamilies} that, in all cases, the coverage of this project is substantially larger, especially in terms of smaller language families or internal isolates. Because \citeA{VanDriem2014} categorised the entire family into generally very fine-grained subfamilies, there are no languages in any of the three studies that cannot be categorised into one of the subfamilies used in this project, and therefore that can initially be seen as gaps in this project's sample. Rather, to identify possible gaps in this survery's sample, it is necessary to identify locations where \citesA{Sagart2019Baye}{ZhangM2019Baye}{ZhangH2020Baye} disagree with \citeA{VanDriem2014}. A number of small disagreements are noted for \citesA{ZhangM2019Baye}{ZhangH2020Baye}

We can also consider areas where van Driem's (2014) larger subgroups are presented as multiple smaller subfamilies in the other data. Namely, \citeA{ZhangM2019Baye} divides the Ngwi-Burmese family into three classifications, diverging approximately 2,000 years ago. Taking these three branches (Loloish, Nusu, Burmish) as separate subfamilies for the sake of representativeness, then the sample for this project contains two Loloish (here: Ngwi) languages, one Burmish language, and no Nusu languages. As is discussed above, the Ngwi-Burmese subfamily is the largest by some margin, which gives some credence to a division into multiple smaller subfamilies. However, when considering the estimated age of the subfamily per the Bayesian analyses, the combined Ngwi-Burmese branch is a similar age to many of the other of van Driem's (2014) subfamilies represented in \citeA{ZhangM2019Baye} (which range from ~750 years old for Tani to ~3,000 years old for Kiranti), as well as in \citeA{ZhangH2020Baye}, in which the subfamilies vary in age from ~500 years old (Ersuish) to ~4,500 years old (West Himalayish). This is to say that although there are a great number of contemporary Ngwi-Burmese languages attested, the branch itself does not appear to be any older than many of the smaller branches used in this survey, and as such there is less of a reason to treat it as too large or too high-level.

The oversampling of the Ngwi-Burmese family in this project's data can at least in part alleviate the arguable underrepresentation of the languages when accounting for the size of the subfamily, but only adding one or two extra languages will not come close to balancing this statistic across the data. This is especially the case given that there are a number of subfamilies in which 50-100\% of languages have been sampled. Again, the qualitative nature of the analysis being undertaken means that the essentially unavoidable imbalance of this measure (languages sampled in a subfamily as a percentage of that subfamily's total number of languages) does not pose a problem as it does in quantitative statistical analyses such as those discussed above.

\begin{table}
  \centering
  \begin{tabular}{l | l | l}
    \hline
    \citeA{Sagart2019Baye} & \citeA{ZhangM2019Baye} & \citeA{ZhangH2020Baye} \\
    \hline
    Angami-Pochuri & Bai & Bai* \\
Ao & Black Mountain & Black Mountain \\
Bai & Dimalish & Dura \\
Black Mountain & Dura & Gongduk \\
Dimalish & Gongduk & Lhokpu \\
Dura & Hrusish & Mru \\
East Bodish & Karbi & Naic \\
Ersuish & Kho-Bwa & Newaric \\
Gongduk & Lepcha & Pyu \\
Hrusish & Lhokpu & Raji-Raute \\
Karenic & Meithei &  \\
Kho-Bwa & Midzuish &  \\
Kukish & Mru &  \\
Lepcha & Newaric &  \\
Lhokpu & Pyu &  \\
Magaric & Raji-Raute &  \\
Meithei & Siangic &  \\
Midzuish & Tshangla &  \\
Mru & Tujia &  \\
Naic &  &  \\
Newaric &  &  \\
Pyu &  &  \\
Raji-Raute &  &  \\
Siangic &  &  \\
Tamangic &  &  \\
Tujia &  &  \\
Zeme &  &
  \end{tabular}
  \caption{Subfamilies not represented in the initial data or results of \citesA{Sagart2019Baye}{ZhangM2019Baye}{ZhangH2020Baye}. Internal isolates are given in italics. * Bai was specifically excluded in this study as the extensive borrowing from Sinitic languages would cause problems for the analysis.}\label{t:Methods:BayesMissingSubfamilies}
\end{table}

The unrepresented subfamilies shown in Table \ref{t:Methods:BayesMissingSubfamilies} are, to an extent, a good indicator for how representative the given samples are. Of course, some of the missing data is unavoidable, and is repeated in this project, namely for Raji-Raute, Gongduk, Black Mountain, and Pyu. \citeA{ZhangH2020Baye} specifically report that Bai (marked with an asterisk in Table \ref{t:Methods:BayesMissingSubfamilies}) was excluded due to high levels of borrowing from Sinitic languages, which causes problems when comparing cognates. For \citeA{ZhangH2020Baye}, this leaves only a few smaller subfamilies unrepresented, and for \citeA{ZhangM2019Baye} a few more, though most are still fairly small or underdocumented. \citeA{Sagart2019Baye}, however, in addition to the smaller or inaccessible subfamilies excluded by the others, has a few more glaring omissions. Namely, no data from any Karenic, Kukish languages were included, nor from the smaller and, at least according to \citesA{ZhangM2019Baye}{ZhangH2020Baye}, closely related Angami-Pochuri and Ao subfamilies. These omissions are, as has been already pointed out in \citeA{Hyslop2020Millet} and even in \citeA{ZhangH2020Baye}, problematic in a quantitative analysis in which a representative sample is a necessity, especially given the much wider coverage of \citeA{ZhangH2020Baye} compared to the two earlier studies. Even in cases where data are simply not available, the large gaps in documentary research across the \lfam\ family may mean that, at this stage, the sort of quantitative statistical research undertaken in these projects is simply not yet viable, at least until a truly representative sample can be developed. Such a sample would ot necessarily have to include data from all of the currently underdescribed languages, but would at least be able to be informed by a better understanding of relationships between families. For instance, if further evidence proves further that Lhokpu can be grouped into a Dimalish subfamily with Dhimal and Toto, then it need not be included separately if that family is already sufficiently represented. Similarly, further research on the other underdescribed languages of Bhutan, Gongduk and Black Mountain Monpa, may well allow them to quite clearly be aligned with existing subgroups, but may also distance them further from any clear classification, further necessitating their inclusion in a representative sample of the family.

There is the potential here, however, for a certain circular logic in developing a sample to be used for research into phylogenies, in that the development of a representative sample needs to be informed by a general understanding of the phylogeny of the family in order to sufficiently and more or less equally survey all areas of the family. This understanding, however, is exactly the outcome the research itself is trying to reach. This is further confounded by the uncertainty the studies discussed here have around any reconstructions to deeper time depths, as is discussed below.

Only briefly discussed in \citeA{ZhangH2020Baye} is the statistical confidence surrounding the various bifurcations given in the study's proposed phylogeny. While the study does not suggest that the given tree is clearly correct, it does suggest that the results of the analysis could be used to inform further studies into the history of the people of the Himalayas \cite[5]{ZhangH2020Baye}. The posterior probability of a given divergence is the calculated likelihood of the given branch being valid given the input data and starting parameters ranging from 0 (no  confidence) to 1 (complete confidence) \cite{Alfaro2006}. It can be seen as the probability that everything under a given branch belongs under that branch at the exclusion of others. For instance, the root of the tree will always have a posterior probability of 1, as the entire tree falls under it and there are no alternatives. While in biology, strongly supported clades need a posterior probability of 0.95, no such consesus has been reached in linguistics, with given thresholds for valid trees ranging from 0.7, 0.9, or with no threshold at all \cite{Dolin2022}. The middle level divergences in \citeA{ZhangH2020Baye} fall well below these thresholds, with the lowest, in a number of places, falling as low as 0.04 (or 4\% confidence). In one case, a proposed bifurcation is estimated to have occurred earlier than that of its proposed parent, which is not possible. The result of this is that for most divergences analysed to have occurred earlier that 4,000 years before present, there is little to no confidence from the analysis that the final output is correct. \citeA{ZhangH2020Baye} do note this in their result that they have over a posterior probability of over 0.95 for a number of clades at various levels. That is to say that although there are a number of clades and subfamilies it does confidently claim, for the most part it cannot actually make any confident claims about the relationships of these lower level clades to each other.\footnote{I discuss this in depth here as, while these figures are clearly presented in \citesA{Sagart2019Baye}{ZhangH2020Baye} and in the supplementary materials for \citeA{ZhangM2019Baye}, I do not believe the actual meaning of these figures would be particularly accessible to a linguist who did not otherwise specialise in this area. The results are provided faithfully and the well-supported clades are listed by \citeA[3]{ZhangH2020Baye}, but the extent to which many of the other clades are \textit{not} supported, and as such the extent to which the analyses do not particularly draw conclusions beyond what is possible with traditional methods, is not, I believe, made clear enough that a linguist working on \lfam\ languages would, on first read, understand the conclusions actually being drawn.} An exception to this is for the Sinitic languages, which were placed as an outgroup with a posterior probability of 0.8. Clades with posterior probabilities of over 0.8 (80\% confidence) have been highlighted on the figures above in red boxes, and can be seen as the clades about which the analysis is able to make a claim over. That is, in Figure \ref{f:Methods:Zhang2020} for example, there is sufficiently strong evidence to support the Kukish subfamily, as well as a clade of the Angami-Pochuri, Ao, Tangkhul, and Zeme subfamilies, but the branches connecting them do not have sufficiently strong evidence to view them as a valid claim.

\citeA{ZhangM2019Baye} overall do not reach the same low posterior probabilities as in \citeA{ZhangH2020Baye}, though do still only report higher higher probabilities at smaller subfamily levels, with probabilities below 0.5 for many of the high level divergences. This suggests that, while more confident than the other analysis, it is still unable to make an solid claims regarding the relationships of well established clades to one another. \citeA{Sagart2019Baye} explicitly use a cutoff of 0.8 for their confident clades, and face a similarly uncertain high-level phylogeny. They do, however, report high confidence for a large clade including Tibetic, Qiangic, rGyalrongic, and Ngwi-Burmese languages they label ``Tibeto-Dulong'' (p. 10318) to a time depth of approximately 5,000 years before present. \citeA{ZhangH2020Baye} report a posterior probability of 0.33 for this grouping.

With this noted, the uncertainty surrounding relationships between established subfamilies noted by \citeA{VanDriem2014} remains to an extent in these studies. Some relationships with a shallower time depth are reported (though inconsistently between analyses) to a larger scale than in the model used for this project, but at the highest level the relationships between different branches of the family remain unclear for both the more traditional comparative method and the more recent Bayesian analysis. This further supports the decision here to approach the development of a representative sample from a phylogenetically agnostic foundation.

\subsection{Languages with conflicting analyses}\label{ss:Description:Conflicts}
A challenge in taking the analyses presented in the literature at face value is that, in cases where multiple descriptions of a single langauge exist (as discussed in Section \ref{ss:Description:StateOfDescription}), there may be different, conflicting analyses of a particular form or function. This section presents a number of examples of cases where a decision had to be made, and discusses why that decision was made the way it was. 

\paragraph{Sunwar}
In his initial descriptions of mirativity, \citeA{DeLanceyMirativity1997} gives Sunwar (Kiranti: Nepal) as an example of a language showing grammaticalised mirativity. In particular, DeLancey describes the copulas \textit{tshə} and \textit{'baak-} as being distinguished based on the newness of knowledge. DeLancey reports that the use of each copula is conditioned independently of the source of the speaker's knowledge (evidentiality), but is rather conditioned by whether or not the information is known without qualification by the speaker (\textit{tshə}) or is information they have just learned, through any of reportative, inferential, or direct evidence (\textit{'baak-}). Example \ref{e:Description:SunwarMirative} shows this distinction in a minimal pair, with the non-mirative used in situations where the speaker has perhaps lived in Kathmandu and is familiar with Tangka and the mirative form used in situations where the speaker perhaps did not know Tangka was in Kathmandu but had just seen him, or had just been told he was there \cite[42]{DeLanceyMirativity1997}.

\begin{exe}
        \ex\label{e:Description:SunwarMirative}
        \begin{xlist}
                \ex 
                \gll Tangka Kathmandu-m tshaa \\
                Tangka Kathmandu-\textsc{loc} \textsc{tsha.3sg} \\
                \glt `Tangka is in Kathmandu.' (non-mirative)

                \ex
                \gll Tangka Kathmandu-m 'baâ-tə \\
                Tangka Kathmandu-\textsc{loc} exist-\textsc{3.sg.past} \\
                \glt `Tangka is in Kathmandu.' (mirative)
        \end{xlist}
        Sunwar \cite[Kiranti: Nepal,][41-42]{DeLanceyMirativity1997}
\end{exe}

\citeA{Borchers2008} disagrees with this analysis, though concedes that she and DeLancey are working with data from different Sunwar-speaking communities, and notes that DeLancey's analysis is working with a smaller corpus than hers. Borchers suggests instead that \textit{'baak-} ``is used to express the general way that things are'', whereas \textit{tshə} ``denotes the concrete and recent state of affairs'' \cite[164]{Borchers2008}. \citeA{Hill2012} is also critical of DeLancey's analysis, though in an overall argument against mirativity as valid cross-linguistic category. That being said, Hill's criticism of the mirative analysis, while referencing \citeA{Borchers2008} for support, relies only on a reanalysis of the meagre data presented in \citeA{DeLanceyMirativity1997} (here in Example \ref{e:Description:SunwarMirative}) and a discussion of edge cases one would not reasonably expect DeLancey to have discussed given the level of detail in the description given in his paper.

The question thus becomes one of which analysis to follow for this typology. That is, in order to enter data from Sunwar into the database, we must make a decision about whose analysis to follow. In this case, given Borchers, at least by her accounts, worked with substantially more data, and spent a much longer time in the field than DeLancey (who worked with a single speaker living in the United States \cite{DeLanceyMirativity1997}). This is, in all reality, a fairly minor decision. It is, in this case, a single point of data in a substantially larger database, and despite Hill's (2012) strong criticism of DeLancey's description, \citeA{Borchers2008} does give a number of possible reasons for the difference in analysis, and does not appear to go to the same length as Hill in actively attempting to refute DeLancey. There also continues to be other languages analysed as marking mirativity in the sample, and as such mirativity as a concept is still considered in this typological analysis.

\paragraph{Lhasa Tibetan}
There is a similar disagreement in the literature over the best way to analyse the evidential system in Lhasa Tibetan, again involving disagreement over the mirative between \citeA{DeLanceyMirativity1997} and \citeA{Hill2012}, though with a greater number of other possible analyses. Epistemic marking in Lhasa Tibetan varies between equative copular clauses, and existential copular, and verb clauses. Specifically, there are two epistemic bases in the equative copula system, compared to three in the existential copulas and verbal morphology \cite{DeLancey2017Tibetan}. These forms are given in Table \ref{t:Description:LhasaEpistemics}, adapted from \citeA[11]{Garrett2001} and using his labels for the 2-3 evidential bases. It is the precise labelling of these bases in a theoretical sense that has been debated in the literature.

\begin{table}
        \begin{tabular}{l|c|c|c}
         & Ego & Direct & Indirect \\ \hline
        Equative Copulas & \textit{yin} & \multicolumn{2}{c}{\textit{red}} \\
        Existential Copulas & \textit{yod} & \textit{ḥdug} & \textit{yodred} \\
        Verbal Morphology (past)\footnote{This paradigm is conditioned by both epistemics and tense/aspect. For the sake of simplicity, the past tense paradigm is given here} & \textit{-pa-yin} & \textit{-song} & \textit{-pa-red}
        \end{tabular}
        \caption{The Lhasa Tibetan epistemic system, adapted from \citeA[11]{Garrett2001}.}\label{t:Description:LhasaEpistemics}
 \end{table}

\citesA{DeLanceyMirativity1997} suggests that, in the three-base systems, the \textit{ḥdug} form represents information with an immediately accessible information source, which he analyses as mirative. As with Sunwar, \citeA{Hill2012} argues against this analysis, rather arguing that the perceived immmediacy of the evidence is a result of the actual condition for the use of the base being the presence of direct sensory evidence. This analysis of a this form as marking direct sensory evidence is also followed by \citeA{Garrett2001}, and is visible in the labelling of Table \ref{t:Description:LhasaEpistemics}. More recently, \citeA{DeLancey2017Tibetan} takes a stance between the two, suggesting that the form is conditioned by direct perception, but that (at least in some cases), this also logically suggests an immediacy of the origo's evidence that the information is also new (though not necessarily unexpected).

\citeA{DeLancey2017Tibetan} also give a different analysis of the conditions for the use of \textit{yodred} forms to \citeA{Garrett2001}. While Garrett suggests that the forms are dependent on some indirect information source, such as hearsay or inference (though this is a major simplification of Garrett's very detailed analysis of the usage of the form), DeLancey suggests an analysis in which the \textit{yodred} forms mark evidentially generic information, or information that is known without qualification or because it is simply generally known \cite[392]{DeLancey2017Tibetan}. A similar claim for a more generic factual evidential function is made for other Tibetic languages by \citeA{Zemp2020}, specifically referring to copulas. In some cases, the factual or netural function is described for the cognate of the \textit{yin} form (see also \citeA{Bodnaruk2023a}) contrasted against an evidentially marked alternative, and some cases for the form in contrast to the cognate of the \textit{yin} form, which in these latter cases marks specific speaker involvement. Zemp extends this second category to the equative copulas in Lhasa Tibetan, in which the \textit{yin} form marks personal involvement, while the \textit{red} form is evidentially neutral \cite[39]{Zemp2020}.

The distinction between \textit{yin} and \textit{red} has also been described as egophoric \cite{EgoIntro}, and does on the surface follow the expected pattern of egophoric contrasts. \citesA{Hill2017}{Gawne2017} argue against egophoricity as a separate cross-linguistic category, but rather frame it as a specific evidential base contrasted with other evidential meanings and not with a \textsc{non-ego} form that is simply defined against it. This ``egophoric evidential'' (as opposed to an egophoric marker in a theoretically distinct category) seems to generally agree with Zemp's (2021) analysis, though focusses on the 3-way distinction seen in other areas of the Lhasa Tibetan grammar. 

Unlike in Sunwar, the actual usage of the forms in Lhasa Tibetan is not in any of the literature substantially at odds. Rather, as the brief description above begins to summarise, the analyses differ in purely theoretical terms, questioning which cross-linguistic categories and theoretical frameworks and lenses best represent the well-described usage of the forms given in Table \ref{t:Description:LhasaEpistemics}. This discussion is by no means unnecessary, but is importantly not one of description per se. In fact, the clearly blurred boundaries between the categories (the 3-term system can and has been described as mirative, evidential, egophoric, and combinations of those three) begins to suggest that perhaps this siloed approach of analysis is insufficient here, a direction that \citeA{Hill2017} begin to move in, but perhaps face challenges in attempting to collapse the distinctions solely into the framework of evidentiality. Paradigms that appear to mark more than one category of epistemic marking will be further discussed in Section \ref{sss:Description:MixedSystems}, and the theoretical implications in detail in Section \ref{s:Discussion:Mixed}.


\section{Data Collation}\label{s:Methods:Schema}
\subsection{Database Overview}
After initial data collection as described above, the notes and summaries written on each language (with reference back to the source material) were summarised into a database that marked whether or not a certain feature was present in a certain form or function in a given language. This summarisation is, of course, reductive, and no such format will be able to succinctly \textit{and} completely describe the full nature of even a single paradigm in a language. Rather, the database this created provided an easy-to-access and easy-to-reference summary of a number of key features that were expected to be relevant to the analysis stages of the project, from which the source material can be more easily referenced. It is descriebd here to explain how data from the numerous different sources were collated and referenced throughout the process of analysis, to be able to more easily see any trends that did emerge and later more easily locate relevant data and examples. This does not mean that the features noted for each language at this stage are perfectly reflective of the actual typological categories described in Chapter \ref{c:Description}. Rather, the more schematised summaries created here were used to come develop those categories and act as a point of reference to the original sources.

The database can be divided in to four general sections: the scope and form of a given marking or paradigm, the function of the marking(s), the extent to which intersubjective reference has been described or suggested in the literature for the markings, and some other features such as the variety of functions in a single paradigm, or the presence of engagement marking on nominal or demonstrative structures (as per \citeA{EvansBergqvistSanRoque2018b}). Additionally, the database does record if anything of note could be found in the literature at all (in the few cases where the answer to this is no, the rest of the entry is empty). In cases where a single language has multiple varied markings or paradigms, one record in the entry will contain multiple `yes' responses to some features.

The analysis undertaken was, for reasons outlined in Section \ref{ss:Methods:RepSample}, primarily qualitative. This involved reading descriptions of epistemic systems in publications and sorting them into this database, and then using this broad overview to begin to draw preliminary theoretical and typological conclusions about the data overall. These preliminary conclusions were then able to be compared specifically to the collected data by referencing the summaries collected in the database. That is, the qualitative nature of the analysis meant that the database itself was not generally used directly for analysis, but as a point-of-reference database to quickly find relevant data and descriptions from the overall sample.

\subsection{Scope and Form}
This feature records where in a given clause the marking is located, and with what scope. Specifically, whether the marking occurs on Verbal Morphology, at the Noun Phrase level, on the Verb Phrase, or as a Discourse particle (i.e. at the speech act level). The specific forms of a given marking or paradigm are not recorded in the summary.

For example in Kurtöp (East Bodish: Bhutan), the obligatory epistemic marking paradigm in the perfective aspect marking features such as mirativity, egophoricity, and evidentiality, appears in the form of a closed paradigm of compulsory suffixes attached to the main verb of a clause \cite{Hyslop2018}, shown in \exref{e:Methods:KurtopScope1} and \exref{e:Methods:KurtopScope2}. This is recorded with a + in the ``Verbal Morphology'' column of the Scope and Form section. Additionally, however, the same distinctions are marked in other domains with the use of specific copulas, recorded with a second + in the ``Verb Phrase'' column.

\begin{exe}
\ex Verbal Morphology \label{e:Methods:KurtopScope1}
\begin{xlist}
\ex
\gll ngat ge-\textbf{shang} \\
1.\textsc{sg.abs} go-\textsc{\textbf{pfv:ego}} \\
\glt `I went.' (p.300)

\ex
\gll tshe khit ge-\textbf{mu} \\
\textsc{dm} \textsc{3.sg.abs} go-\textsc{\textbf{pfv:ind}} \\
\glt `Then he left.' (p.303)

\end{xlist}

\ex Copulas \label{e:Methods:KurtopScope2}
\begin{xlist}
\ex 
\gll mau zangu ngaksi \textbf{nawala} \\
\textsc{dn} zangu \textsc{quot} \textsc{\textbf{cop.exis}} \\
\glt `There is this (thing) down there called ``zangu''.' (p.310)

\ex
\gll Hâ-pa=the \textbf{nâ} \\
Hâ-\textsc{dnz=indef} \textsc{\textbf{cop.exis.mir}} \\
\glt `He is a Hâpa (from Hâ).' (p.310)
\end{xlist}
Kurtöp \cite[East Bodish: Bhutan,][]{Hyslop2017}
\end{exe}

\subsection{Function of Marking}
The Function of Marking section describes the function of the form or forms in a paradigm in relation to the cross-linguistic categories of epistemic modalilty (EM), evidentiality, egophoricity, mirativity, and engagement. It is split into two subsections, the first recording instances of `prototypical' forms of a given category. That is, a column will have a + for instances where the given paradigm marks only the category in question, and where the paradigm closely reflects the definition of the category in the literature.

For instance, the conjunct/disjunct paradigm in Kathmandu Newar (Newaric: Nepal), as originally described in \citeA{HaleNewar1980}, fits very closely with the definitions given for egophoricity to the point that it given as an illustrative example of egophoricity at its simplest level in \citeA[3]{EgoIntro}. As such, it would be marked with a + in the egophoricity column of the prototypcial subsection. In a language such as the aforementioned Kurtöp, however, the marking of mirativity, egophoricity, and evidentiality all occur on different forms in the single paradigm, which as a result does not strictly adhere to the definitions given for such categories in the literature \cites{DeLancey2012}{EgoIntro}{Aikhenvald2018Intro}.

In these cases where it would be insufficient to represent a paradigm as adhering to any well-defined category, another subsection recording loose examples of a given category is used. In this case, while not strictly fitting the definitions for a single mirative, egophoric, or evidential paradigm, the paradigm would be marked as having some broad instance of all three.

\subsection{Described addressee-perspective}
This section notes whether or not there are any cases of addressee-perspective either described or exemplified in the available literature on a given language, either in interrogative or declarative structures. This assessment is, however, reliant on the extent of literature coverage on a given language. For instance, \citeA[408]{Bodt2020} explicitly describes a mirative meaning of the Duhumbi (Kho-Bwa: India) copula \textit{le} as marking the perspective of the addressee in declaratives, noting that the ``speaker considers the information of the proposition with copula \textit{le} as new to the addressee and relevant to him in the moment of speaking...'', contrasted with the copula \textit{beʔ}, which does not carry this mirative meaning, exemplified in \exref{e:Methods:Duhumbi}. While it is easy to give a definitive `yes' where evidence is given in the literature, in many cases there is simply no reference made to any intersubujective-like features or cases (such as the common occurence in evientials in interrogatives \cite{Aikhenvald2018Intro}). In these cases it is not possible to say that grammaticalised intersubjectivity \textit{does not} exist, but rather only that it has not yet been readily obseved.

\begin{exe}
\ex 
\begin{xlist}
\ex 
\gll Gonpa pʰu tᶝhaŋkʰo le \\
temple mountain on-\textsc{loc} \textsc{cop.le} \\
\glt `The temple is on top of the mountain.' [New to the addressee]

\ex 
\gll Gonpa pʰu tᶝhaŋkʰo beʔ \\
temple mountain on-\textsc{loc} \textsc{cop.ex} \\
\glt `The temple is on top of the mountain.' [Neutral to the addressee]
\end{xlist}
Duhumbi \cite[Kho-Bwa: India][408, notes on relation to addressee added by me]{Bodt2020}
\end{exe}

\subsection{Other Features}
Finally, the database records four other possible features that might be useful to quickly access and reference. These are the presence of mixed function paradigms, that is paradigms with forms that do not fit into a single established cross-linguistic category, the presence of nominal engagement as referred to in the introduction to this section (Section \ref{s:Methods:Schema}), the diachronic source for the the forms documented (where given), and whether or not forms are obligatory.

The first of these, the mixed paradigms, could potentially fairly closely mirror the cases where no prototypical example of a category is identified, but with a key difference. Systems whereby some form of subjective or intersubjective marking (such as evidentiality in the example given below) is spread across a number of different domain's of a language's grammar do not fit into the aforementioned `prototypical' category, nor do they fit into this `mixed paradigm' group. For instance, in Meithei \cite[295]{Chelliah1997}, evidentiality is marked across a number of different domains in the grammar, such as derivational morphology, clitics, and complementation, as opposed to being marked by a single discrete paradigm. It is worth noting that this ``scattered'' marking of evidentiality is widespread, and is by no means ignored in the literature \cite[23]{Aikhenvald2014}. The example of Kurtöp, given above, however, does represent a mixed paradigm, where a single paradigm contains multiple different functional categories.

The presence of nominal engagement in \lfam\ languages has not been widely documented, but has the potential to give insights into the development of verbal or clause-level marking in the wider family where they do appear. Nominal engagement has been documented in Purik Tibetan \cite[Tibetic: India:][]{Zemp2021} and in Phola \cite[Ngwi-Burmese, PRC:][]{GonzalezPerez2023}.

For example, in Purik Tibetan, demonstratives \textit{de} and \textit{e} are functionally contrastive in certain positions by the attentiveness of the addressee, with \textit{e} working to redirect their attention to a referent \cite{Zemp2021}, as opposed to a state of affairs already clear to both speech act participants. Examples of this are given in \exref{e:Methods:PurikDem}.

\begin{exe}
  \ex \label{e:Methods:PurikDem}
  \begin{xlist}
    \ex 
    \gll kulik-po di-ka pʰjal-la \textbf{de} \\
    key-\textsc{def} this-\textsc{loc} hanging-\textsc{dat} \textsc{\textbf{top}} \\
    \glt `The key’s hanging here (right in front of your eyes).' (p.412)

    \ex
    \gll sŋuntʃoqtʃoq e \\
    deep.green \textsc{top2} \\
    \glt `Look, how green it is over there!' (p.413)
  \end{xlist}
  Purik Tibetan \cite[Tibetic:India][]{Zemp2021}
\end{exe}

\subsection{Example}\label{ss:Methods:MagarExample}
This section works through the classification of an example language to show how it this schema has been used in practise. Here, the description of Magar by \citeA{GrunowHarsta2008} is used as it features distinctions across a number of the areas being recorded in the schema. The epistemic system in Magar is split across all three discrete systems, marking inferential and reportative evidentials, miratives, and quotatives.

\subsubsection{Brief Description}
Miratives in Magar are marked in a variety of forms, either nominalisations or related constructions. In Example \ref{e:Methods:MagarMirativeIntro}, mirativity is marked the nominaliser \textit{-o} followed by a grammmaticalised copula now carrying aspectual meaning \textit{le}. 

\begin{exe}
  \ex\label{e:Methods:MagarMirativeIntro}
  \begin{xlist}
    \ex
    \gll thapa i-laŋ le \\
    Thapa \textsc{p.dem-loc} \textsc{cop} \\
    \glt `Thapa is here.' (non-mirative)

    \ex 
    \gll thapa i-laŋ le-\textbf{o} \textbf{le} \\
    Thapa \textsc{p.dem-loc} \textsc{cop-\textbf{mir}} \textsc{\textbf{impf}} \\
    \glt `(I realize to my surprise that) Thapa is here!' (mirative)
  \end{xlist}
  \cite[Magar,][480]{GrunowHarsta2008}
\end{exe}

\citeA[480]{GrunowHarsta2008} also gives one other nominaliser that can convey mirative meaning, a form \textit{cyo \~ cʌ}. The distribution of these forms tends to follow the person of the subject of the clause, with the latter \textit{cyo \~ cʌ} form mainly used for third-person subjects and the former \textit{-o le} for subjects who are speech act participants, though Example \ref{e:Methods:MagarMirativeIntro} is a clear exeption to this. The forms appear to always reflect a speaker-origo, including in questions, or can reflect a character origo in narratives. Examples of speaker-origo in narratives for rhetorical effect are also given, though are described as ``feigned'' (p. 493). One example, reproduced in Example \ref{e:Methods:MagarMirativeAddOri}, may well show addressee-origo in a declarative construction. Here, the mirative is used both in an interrogative (\ref{e:Methods:MagarMirativeAddOri:q}) and confirmation of that question (\ref{e:Methods:MagarMirativeAddOri:a}). Given information being confirmed by the speaker cannot be in the moment new to said speaker, an alternative explanation for the function of the mirative here is necessary. \citeA[486]{GrunowHarsta2008} explains the use of mirative in the response as not marking information as new to the speaker, but marking information as something she cannot mentally integrate. An alternative view, at least seeing this data in isolation, is that the interrogative here is an explanation of disbelief on the part of the first speaker, hence marked with the mirative, and that the second speaker is also reflecting the first speaker's expression of disbelief. This is to say that the second speaker is agreeing that the first speaker is perhaps correct to use the mirative construction, arguably reflecting their addressee's perspective.

\begin{exe}
  \ex \label{e:Methods:MagarMirativeAddOri}
  \begin{xlist}
    \ex \label{e:Methods:MagarMirativeAddOri:q}
    \gll mi-ja ma-phunɦ-o le-sa si-cʌ ale \\
    \textsc{poss}-child \textsc{neg}-{give birth}-\textsc{mir} \textsc{impf-infr} die-\textsc{att} \textsc{cop} \\
    \glt `She just died, undelivered!?'

    \ex \label{e:Methods:MagarMirativeAddOri:a}
    \gll ã ma-phunɦ-o le-a \\
    yes \textsc{neg}-{give birth}-\textsc{mir} \textsc{impf-pst} \\
    \glt `Yes, undelivered!'
  \end{xlist}
  \cite[Magar,][487]{GrunowHarsta2008}
\end{exe}

Evidentiality in Magar is also marked across multiple grammatical domains. Direct first-hand evidence, or statements of general cultural fact are unmarked, inferential evidentials are marked with the verb suffix \textit{-sa} and reportatives are marked with a particle \textit{ta}. \citeA{GrunowHarsta2008} gives a minimal triplet of these three meanings, reproduced in Example \ref{e:Methods:MagarEvidIntro}.

\begin{exe}
  \ex \label{e:Methods:MagarEvidIntro}
  \begin{xlist}
    \ex Direct
    \gll ho-se taɦ-raɦ-a \\
    \textsc{d.dem-def} reach-come-\textsc{pst} \\
    \glt `He has arrived.' (I see him.)

    \ex Inferential
    \gll ho-se taɦ-raɦ-le-sa-a \\
    \textsc{d.dem-def} reach-come-\textsc{impf-infr-pst} \\
    \glt `He has arrived.' (I see his bag.)

    \ex Reportative
    \gll ho-se taɦ-raɦ-a ta \\
    \textsc{d.dem-def} reach-come-\textsc{pst} \textsc{rep} \\
    \glt `He has arrived.' (They say.)
  \end{xlist}
  \cite[Magar,][497]{GrunowHarsta2008}
\end{exe}

In addition to these forms, there is also a quotative construction for direct quotes, formed with a subordinate clause and the speech verb \textit{de} `say' (p. 498). Unlike in some languages, the grammaticalised forms given in Example \ref{e:Methods:MagarEvidIntro} are not obligatory, though it is not clear how this factors into Grunow-Hårsta's analysis of the direct evidential as null-marked.

Both the inferential and reportative are used in interrogative structures, where they reflect the perspective of the addressee. Additionally, both are used in narratives, but with different functions. The inferential evidential is used in narratives when narrating a story told in images (see the Family Problems Picture Task in Section \ref{p:Methods:FamilyProblems}), as well as reflecting the character's perspective, both in direct speech from the character, and in narration. The reportative is used very widely in narrative, but references exclusively speaker perspective, marking the speaker's source of information for the narrative itself. The forms \textit{-sa} \textsc{infr} and \textit{ta} \textsc{rep}, being marked respectively as a verb suffix and separate particle, can cooccur (p. 513). This marks reportative evidence for the speaker, whose sources in turn had inferential evidence.

Independent of these forms, Magar also has a set of epistemic modal markers which can cooccur with both miratives and evidentials, which can in turn cooccur with each other. On these grounds, Grunow-Hårsta describes these as three distinct systems within the grammar. Potentially, given the formal difference in the location of the marking between the inferential and reportative marking, this could be further expanded to the point that there is no single paradigm of mutually exclusive forms filling the same grammatical slot in Magar. This ``scattered'' system is not uncommon in the region \cite[480-481]{GrunowHarsta2008}. In terms of the typology set out in \cite{Aikhenvald2004}, Magar would be a B1 system, marking visual, inferred, and reported evidence. As will be discussed in detail in Chapter \ref{c:Discussion}, this does not manage to fully characterise the overall epistemic marking system (in a broad sense) of scattered morphemes in Magar. This is to say, in a system where all forms are marked in different grammatical slots and can cooccur with other in seemingly any combination, there is less justification to treat the forms traditionally classified as evidentiality as separate systems beyond the traditional theoretical boundaries which I am advocating against in this analysis.

\subsubsection{Representation in the summarised database}
Table \ref{t:Methods:SchemaExample} shows the representation of this description of Magar in the database described above. The first section, the Language Metadata, gives basic information on the language, including its coordinates according to Glottolog. The next section, Scope and Form, notes that there is evidence of grammatical epistemic marking, and that it appears both as verbal morphology (the nominaliser component of the mirative and the inferential suffix) and as aa discourse particle or at the speech act level (the copular component of the mirative and the reportative marker). The Function of Markings section notes that there is (albeit with limited description) a system of epistemic modality marking, as well as an `other' system that does not easily fit into the a single group. This other system is then noted in the next section to feature functions of both mirative and evidential meanings, referring to the scattered system described above. Described Intersubjectivity notes that there is clear reference to the addressee's perspective in interrogative constructions, and potentially in declaratives (see Example \ref{e:Methods:MagarMirativeAddOri}). Finally, in Other Features, no diachronic source is noted as none was given by \citeA{GrunowHarsta2008}, the fact that the system is not obligatory is noted, as well as the lack of mixed paradigm or noted nominal engagement marking.
\begin{table}
  \begin{tabular}{|l|l|}
  \hline\hline
  \multicolumn{2}{|l|}{Language Metadata} \\ \hline
  Language & Magar \\ \hline
  Subfamily & Magaric \\ \hline
  Source & Grunow-Harsta 2008 \\ \hline
  Glottolog Coordinates & 27.41, 87.06 \\ \hline \hline
  \multicolumn{2}{|l|}{Scope and Form} \\ \hline
  Evidence of epistemic marking & + \\ \hline
  Verb Morphology & + \\ \hline
  Noun Phrase &  \\ \hline
  Verb Phrase &  \\ \hline
  Discourse particle/speech act level & + \\ \hline \hline
  \multicolumn{2}{|l|}{Function of Markings} \\ \hline
  EM & + \\ \hline
  Ev &  \\ \hline
  Ego &  \\ \hline
  Eng &  \\ \hline
  Mir &  \\ \hline
  Other? & + \\ \hline \hline
  EM &  \\ \hline
  Ev & + \\ \hline
  Ego &  \\ \hline
  Eng &  \\ \hline
  Mir & + \\ \hline \hline
  Term(s) used in source & Mirative, Direct/Factual, Reportative, Inferred \\ \hline \hline
  \multicolumn{2}{|l|}{Described Addressee-Perspective} \\ \hline
  Evidence of IS in interrogative structures? & + \\ \hline
  Evidence of IS in declarative structures? & +? {[}486-487{]}, (c) reads a bit like IS \\ \hline \hline
  \multicolumn{2}{|l|}{Other Features} \\ \hline
  Diachronic Source? &  \\ \hline
  Obligatory & No \\ \hline 
  Mixed Paradigm & No \\ \hline
  Nominal Engagement & No \\ \hline \hline
  \end{tabular}
  \caption{Magar, described in \citeA{GrunowHarsta2008}, represented as an entry in the summarised database described in Section \ref{s:Methods:Schema}.}\label{t:Methods:SchemaExample}
  \end{table}



\section{Field Methods}\label{s:Methods:FieldMethods}
\subsection{Lhokpu Field Work}
The opportunity arose during this project to collect first-hand data on Lhokpu (Subgroup unclear: Bhutan) to fill in a gap in the available data. As has been discussed above, Lhokpu has been potentially aligned with the Dimalish subfamily, and while the conclusions in \citeA{Grollmann2018} are certainly well supported, they have not necessarily been totally proven. As such, Lhokpu has been included in this study as an internal isolate.

Lhokpu is spoken in a small number of villages in the Dophuchen and Tading gewogs of Samtse District, in  south-western Bhutan. There are two non-contiguous groups of speakers, one located about 15km up the Amo Chhu (River) from the other, which is in turn a similar distance upriver of Totopara, the village in which the potentially closely related language Toto is spoken. \citeA{Grollmann2018} estimate approximately 2500 speakers across all villages, though more recent estimates (Mareike Wulff, p.c. 2024) are as low as 800, many (potentially all) of whom also speak Nepali, and to a lesser extent, Dzongkha and English.

For this project, limited time was available to collect data for analysis, and as such the data collected needed to efficiently reflect any epistemic-marking system that might exist in the language. With the time available (especially given this was a smaller component of the larger project), recording large amounts of natural and unprovoked dialogue, stories, and other content, and then identifying the relevant forms therein was not feasible. It can also be difficult to establish what the actual epistemic contet of forms used in these settings would be, as it may not be totally clear to a researcher what the relationships of the speaker and addressee are to any given piece of information. On the other hand, it has been noted that speakers are not typically consiously aware of epistemic distinctions in language, and as such it can be difficult to ask or directly elicit them \cite{Grzech2020}. As such, as middle ground approach was used here, in which elicitation activities were used to generate naturalistic conversation data within an established and controlled epistemic context. These activities in particular draw from the \citeA{Gawne2020} and her discussion on using such tools, as well as \citeA{Grzech2020}. \todo{Gwen can you judge if this is too explicit or too vague in terms of not upsetting Bhutan immigration?}

\subsection{Field Methods and Elicitation Activities}
By establishing a set of contrastive epistemic contexts across the elicitation activities run, it is possible to, at least to a certain extent, ascertain more clearly the conditioning factors behind the selection of forms. The two primary activities that were run were the ``Family Problems Picture Task'' \cite{SanRoque2012a} and the ``Man and Tree Picture Sets'' \cite{Levinson1992}. These activities comprised the majority of the field work undertaken, and were supplemented by a small amount of elicitation of basic language structures and the collection of a wordlist.

\paragraph{Family Problems Picture Task}\label{p:Methods:FamilyProblems}
The Family Problems Picture Task was specifically developed with the elicitation of epistemic forms in mind \cite{SanRoque2012a}, and involves four parts. First, two participants are presented with a set of images \cite{Carroll2009} depicting various interactions between family members in a pseudo-random order\footnote{the order is random for the participants, but is given by \citeA{SanRoque2012a} to allow for consistency and easier analysis down the line, so analysts do not need to work out which image is being described.} and are asked to describe them. Next, they are asked to confer and place the images in an order that depicts a story, and finally are asked to tell the story, once in third-person and once in first-person.

Across these parts, a number of different epistemic contexts are created. Table \ref{t:Methods:FamilyProblemsEvidentials} shows the these contexts, and the parts in which they are present. Visual evidence can be for information can be found in the descriptions and discussion, as participants are seeing the images for the first time, and subsequently discussing their contents. Similarly, inferential evidence could also be found in both, though potentially more prominently in the discussion phase, as participants are piecing together a story from the illustrations, requiring them to draw inferences about the exact events depicted. The description phase also shows equal epistemic authority over the images for both participants, as neither one will have seen the activity before. This equal authority might also be reflected as shared information, new information (i.e. mirative), or non-origo authority. Non-origo authority is distinguished from equal authority in that while they both accurately describe the epistemic context created in the task, equal authority refers to systems encoding that both speech act participants have the same authority, but not the strength of that authority, whereas non-origo authority refers to systems encoding specifically the origo's (speaker in declarative, addressee in interrogative) lack of authority over the information at hand, but does not directly reflect the addressee's perspective. Participatory evidence, or egophoric evidence \cite[][see ]{Gawne2017}, could be marked in the first-person telling of the story, as could, potentially along with the other parts, factual or neutral evidence as per \citeA{Zemp2020}.

\begin{table}  
  \noindent\adjustbox{center}{\begin{tabular}{r|c|c|c|c}  
   & Description & Discussion & Third-person telling & First-person telling \\
  Visual evidence & ✔ & ✔ &  &  \\
  Inferential evidence & ✔ & ✔ &  &  \\
  Non-origo or equal authority & ✔ &  &  &  \\
  Participatory evidence &  &  &  & ✔ \\
  Factual or neutral evidence &  &  & ✔ & 
  \end{tabular}}
  \caption{Epistemic contexts covered by each part of the Family Problems Picture Task}\label{t:Methods:FamilyProblemsEvidentials}
\end{table}


In some cases, namely with the non-origo and equal authority, as well as the participatory and factual evidentials, the conditions for two of the given epistemic bases are met in a single part of the activity. That is to say that, for instance, participatory evidence and factual evidence could theoretically be triggered by the same epistemic context. They are, however, functionally distinct in theoretical terms and as such have been included separately. That being said, without further evidence or usage contexts that are able to distinguish, it would not be directly possible from the data produced by this activity alone to determine if a given form used in the first-person telling, as an example, is conditioned by direct speaker involvement in the form of participatory evidence, or by a broader higher origo-authority as seen in factual evidentials. The typology of these similar-yet-different epistemic bases is discussed in greater detail in Section \ref{ss:Description:ClassByFunction}.

\paragraph{Man and Tree Picture Sets}
The Man and Tree Picture Sets \cite{Levinson1992} are a series of image sets depicting plastic figures in various arrangements. Within the sets, different images show different arrangements of the same objects. Three sets were used in this project, some of which were combinations or subsets of the sets initially given in \citeA{Levinson1992}, and as such are given their own labels in the context of this project. The first set, \textsc{balls}, has four images showing red and yellow balls in various spatial and colour configurations. The second, \textsc{sawdust}, depicts a small pot full of sawdust, a plant, and a basket, with the pot variously covered, uncovered, overflowing, or entirely absent. The last set, \textsc{pigs}, shows a number of men, pigs, and small bushes in various numbers and configurations. The first two sets are substantially smaller than the {pigs}, and were used to teach participants the activity, and as a sort of warm-up activity.

The activity itself was run as a guessing game or director-matcher task, in which one participant, the matcher, has all images in the set laid out in front of them on cards, and the other, the director, has all images in a deck face-down. Between the participants is a partition such that the director cannot see the matcher's face-up array of images. One by one, the director draws a card and describes it to the matcher, who asks questions in turn, until the matcher is able to select which card the director has just drawn. They confirm that the card is correct, then repeat the process until all cards have been drawn.

The activity was originally designed for the elicitation of spatial reference systems, for which it was very effective here (though the analysis stemming from this falls outside the scope of this thesis), but has been repurposed here with a degree of success for the elicitation of epistemics. As with the Family Problems Picture Task, the activity creates a number of different epistemic contexts, though with the simpler task with fewer stages, there are fewer epistemic contrasts developed. 

The Man and Tree activity has fewer separate stages than the Family Problems activity, and has been presented in Table \ref{t:Methods:ManTreeEvidentials} divided into the speech acts of each participant. The Director Description refers to the initial description of the image, and subsequent further comments on the image in response to questions from the matcher, which comprise the other speech acts in the activity. Visual evidence can be seen primarily in the initial descriptions of the images from the director. In all cases, however, this task shows unequal epistemic authority (contrasted with the Family Problems activity), in that the director has sole access to the aforementioned visual evidence at all times, and the matcher is either polling that visual evidence in asking questions, or polling a more authoritative evidence in confirming if they have selected the correct image. The useful difference between these two uses of unequal epistemic authority (the director and the matcher) then, is that one is speaker-origo, with the speaker referencing their own awareness, and the other is, being interrogative, likely addressee-origo.

\begin{table}
  \begin{tabular}{r|c|c}
   & Director Description & Matcher Questions \\
  Visual Evidence & ✔ & \\
  Unequal epistemic authority & ✔ & ✔ \\
  Shifted Origo in Questions & & ✔
  \end{tabular}
  \caption{Epistemic contexts covered by different areas of the Man and Tree Picture Task}\label{t:Methods:ManTreeEvidentials}
  \end{table}

The contrast between the equal epistemic authority in the Family Problems activity and the unequal epistemic authority here is a further distinction that is potentially able to be drawn from these two tasks. In other cases, it was these confirmation questions polling clearly non-shared unequal knowledge that was able to shed light on an epistemic system potentially conditioned by epistemic authority \cite{Bodnaruk2023}. In this case in particular, however, speakers did not verbally confirm their image choices. It is not clear if this is a fault of how the task was explained, if it was a result of the social dynamics between the participants, or just unfortunate chance.

\subsection{Outcomes}
These activities benefit from a solid foundation of analysis on the language such that individual forms can be better identified and separated after the activities are transcribed and translated \cite{Bodnaruk2023}. The lack of this foundation was a hindrance to the full analysis of the data gained, however, it was still possible to confirm some epistemic distinctions that had been originally attested in the basic elicitation. Transcriptions and translations were produced primarily in the field in \citeA{ELAN} together with local consultants, with some additional material being transcribed over WhatsApp after the conclusion of the field trip. Example \ref{e:Methods:LhokpuDistinction} shows the main epistemic contrast with a confident analysis from the elicitation activities. 

\begin{exe}
  \ex\label{e:Methods:LhokpuDistinction}
  \begin{xlist}
    \ex\label{e:Methods:LhokpuDistinction:mi1}
    \gll nosam rang-ka [ganmo \textbf{mi}] \\
    mind \textsc{pron-gen} [wife \textbf{\textsc{cop.exist}}] \\
    \glt `In his mind, “My wife is there”.' (Family Problems)

    \ex \label{e:Methods:LhokpuDistinction:mi2}
    \gll ka-lok dzeʔ nih-pu \textbf{mi} \\
    \textsc{1.sg-obl} dog two-\textsc{clf.gen} \textbf{\textsc{cop.exist}} \\
    \glt `I have two dogs.' (Elicited)

    \ex \label{e:Methods:LhokpuDistinction:miha1}
    \gll kona i-du meʔ \textbf{mihã} \\
    then \textsc{prox-loc} fire \textbf{\textsc{cop.exist.evd}} \\
    \glt `Then here there is fire' (Man and Tree - Sawdust)

    \ex \label{e:Methods:LhokpuDistinction:miha2}
    \gll kanka it-dra \textbf{mihã} \\
    old.man one-\textsc{clf.anim}  \textbf{\textsc{cop.exist.evd}} \\
    \glt `There is one old man.' (Family Problems)\footnote{Speakers explained the use of the classifiers \textit{-dra} and \textit{-pu} as marking human and general referents, however the `human' classifier \textit{-dra} was occasionally used to refer to animals in the Man and Tree activity, along with the general classifier \textit{-pu}. No further classifiers have been identified yet.}
  \end{xlist}
  (Lhokpu)
\end{exe}

The form \textit{mi} is used as an existential copula in cases referring specifically to personal knowledge or experience, seen in \ref{e:Methods:LhokpuDistinction:mi1} where it reflects the personal insights of the character, and in \ref{e:Methods:LhokpuDistinction:mi2} where it reflects the privileged access of the speaker to information about themself. The alternative form \textit{mihã} is used in cases with direct visual evidence, in both \ref{e:Methods:LhokpuDistinction:miha1} and \ref{e:Methods:LhokpuDistinction:miha2} reflecting the speaker seeing parts of the images for the first time. Without further research, however, it is difficult to say exactly which of the contrastive aspects of the two epistemic contexts is responsible for the difference in forms. In any case, it is clear that there is a binary epistemic distinction on Lhokpu existential copulas, and that it is generally speaking a distinction between higher vs lower authority, with higher authority in the data occurring from personal insights or participatory/ego evidence, and lower authority from visual evidence.

Less confident conclusions were also able to be tentatively drawn regarding epistemic marking in verbal morphology. A limited understanding of some areas of the phonology of Lhokpu limited the analysis that was able to be undertaken here. A verb suffix \textit{-ah} occurs throughout the data collected in the elicitation activities, but not in any directly elicited data. Notably, the directly elicited data, sentences that were translated directly from English or Dzongkha into Lhokpu by consultants, are devoid of epistemic context. While, of course, such context can be described or imagined, the challenges in consious awareness of the conditions on the usage of epistemic forms as discussed above mean that these described situations are not reliable indicators of the actual usage of epistemic markers. The suffix \textit{-ah} is potentially equivalent to a form \textit{-a(l)} given by \cite{Grollmann2018}, with the lateral coda (dropped word finally) debuccalised, mirroring a sound difference between the data collected in Jigme village in this project and Grollmann and Gerber's (2018) data, wherein Grollmann and Gerber's possessive pronoun suffix \textit{ŋa} is attested in Jigme as \textit{-ha}. It is this glottal coda that is particularly challenging in the analysis of the form, as its phonological behaviour is not yet clear. While it is certainly present in many words, and its presence appears to be contrastive, it is not yet clear if another attested verb suffix \textit{-a} is a separate morpheme or an allomorph of \textit{-ah} with the glottal coda deleted. It is also possible that the glottal coda is present but remains indetected in the analysis. It is here that the lacking foundational knowledge of phonology and basic verbal morphology on the language seriously begins to hinder the analysis that can be completed at this stage.

\cite[20-21]{Grollmann2018} describe \textit{-a(l)} as marking something ``not personally experienced by the speaker or as not belonging to the personal knowledge of the speaker'', though do not provide examples. This functional description appears, at least at this stage, to work with the data collected here.

\begin{exe}
\ex\label{e:Methods:LhokpuVerbal}
\begin{xlist}
\ex \label{e:Methods:LhokpuVerbal:coda}
\gll ŋan dokm̥eŋ-su dzoŋ-do-\textbf{ah} \\
person walking.stick-\textsc{com} stand-\textsc{prog-\textbf{evd?}} \\
\glt `The person is standing with the stick.'

\ex \label{e:Methods:LhokpuVerbal:nocoda}
\gll siŋ-hõ hut-a dzoŋ-do-\textbf{a} le~le \\
tree-\textsc{towards} look-? \textsc{asp-prog-\textbf{evd?}} downhill~\textsc{adv} \\
\glt `Looking downhill towards a tree.' \\
(Man and Tree - Pigs)

\ex \label{e:Methods:LhokpuVerbal:cold}
\gll ka tɕuŋ̊-do \\
\textsc{1.sg} be.cold-\textsc{prog} \\
\glt `I am cold.'

\ex \label{e:Methods:LhokpuVerbal:coldevid}
\gll tɕo ném-do-\textbf{ah} \\
water be.cold.to.touch-\textsc{prog-\textbf{evd?}} \\
\glt `The water is cold.' \\
(Elicitation)


\end{xlist}
(Lhokpu, Subfamily Unclear: Bhutan)
\end{exe}

Example \ref{e:Methods:LhokpuVerbal} shows both \textit{-ah} and an occurence of \textit{-a} that appears particularly likely to be an allomorph of \textit{-ah}, in both cases reflecting new, direct visual evidence, that was not previously part of the speaker's integrated knowledge. Much like with \textit{mi} and\textit{mihã}, it is difficult to say if, as suggested by \citeA{Grollmann2018} for \textit{-a(l)}, the use of \textit{-ah} is conditioned by the speaker's prior knowledge of some state of affairs, or if it is conditioned by the visual evidence the speaker has for their knowledge of that state of affairs. Notably, earlier in example \ref{e:Methods:LhokpuVerbal:nocoda}, another instance of \textit{-a} is attested, at first glance here appearing to be a non-final marker connecting the verb \textit{hut} `look' with the finite-marked verb \textit{dzong} `sit', here marking an aspectual distinction. This non-final marker analysis does not seem to work for the emphasised marker, however, suggesting that, in lieu of an analysis that can account for both uses here, there are two functions or meanings for the suffix \textit{-a}, perhaps one of which is an allomorph of \textit{-ah}. Examples \ref{e:Methods:LhokpuVerbal:cold} and \ref{e:Methods:LhokpuVerbal:coldevid} show a contrast between the use or lack of the \textit{-ah} suffix. It is not used in Example \ref{e:Methods:LhokpuVerbal:cold}, in which the speaker is referring to their internal experience using the verb \textit{tɕuŋ̊}, which is restricted to this meaning. In Example \ref{e:Methods:LhokpuVerbal:coldevid}, however, the speaker is referring to an external observation and does use the probable evidential \textit{-ah} with the verb \textit{ném}, which specifically refers to termperature of other objects assessed through touch.

Between these two domains in which a probable epistemic distinction has been observed, this is the total extent of the current analysis into epistemic marking in Lhokpu, and as such is the total data that can be included. As is the case with published descriptive material, it is difficult to distinguish confidently between a language lacking a certain functional contrast, and such a contrast simply not being described in the current analysis. As such, when reading wider literature, conclusions cannot readily be drawn about systems lacking features. This limitation extends to the data collected for Lhokpu, simply because the analysis is nowhere near complete enough to confidently exclude any features. Instead, it is possible to preliminarily describe the system as occuring across multiple domains of the language, and containing a number of contrastive forms, conditioned by the closeness of the origo to the information. This closeness may depend on evidence source (direct visual vs general knowledge), partipation or direct involvement, or some higher level claim of authority, though it is not yet clear if any single of these conditions is the sole relevant condition, or if it is in fact some combination thereof, or perhaps an entirely different condition.