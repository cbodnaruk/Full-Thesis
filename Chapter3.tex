\chapter{Development of database}\label{c:Methods}
\section{Overview}
This chapter describes the development of the database of epistemic marking in \lfam\ languages to be used for analysis, and the methodological decisions that were made throughout. Section \ref{s:Methods:Collection} will discuss the collection of data, including the steps taken to ensure a representative cross-section of the \lfam\ family. A number of cases containing multiple conflicting analyses are discussed in Section \ref{ss:Description:Conflicts}, including the decision-making process behind which analysis was ultimately used in this survey. Section \ref{s:Methods:Schema} will address the specific features used to separate and categorise the languages surveyed; namely, the presence of various types of epistemic marking and reference to addressee perspective, among other features, as well as an example entry with a full explanation of Magar \cite[Magaric: Nepal][]{GrunowHarsta2008}. The resulting findings will subsequently be presented in Chapter \ref{c:Description}.

\section{Collection}\label{s:Methods:Collection}
\subsection{Developing a Representative Sample}\label{ss:Methods:RepSample}
In conducting a large-scale survey of any language family, especially in a family as large and undescribed as the \lfam\ family, it is impossible to consider every single language as would be methodologically speaking ideal. Instead, it is important to ensure that the languages surveyed demonstrate as accurate a representation of the family as a whole as is possible. This ensures that major areas and subgroups of the family are not left out. Failure to make such considerations may result in missing significant pieces of data, making any generalisations about the family invalid, or only applicable to some potentially arbitrary subset of the languages of the family. This was discussed in terms of the phylogenetic analyses presented in Section \ref{ss:Methods:Bayesian}, in which large groups (such as Karenic in \citeA{Sagart2019Baye}) or small divergent groups (such as Kho-Bwa or Siangic in \citesA{Sagart2019Baye}{ZhangM2019Baye}) are missed, along with their informative data, potentially shifting results in the wrong direction.

Section \ref{ss:THOverview:Subfamilies} introduced the set of subfamilies proposed by \citeA{VanDriem2014} to represent a phylogeny-agnostic view of the \lfam\ family. Using this set of subfamilies as a foundation for the selection of languages provides a solid foundation to ensure no language groups are missed, while also avoiding engaging with methodological debates with no real consequence on this thesis. That being said, simply selecting languages from each subfamily in equal number will not necessarily solve the issues in representativeness surrounding the statistical studies discussed in Section \ref{ss:THOverview:HighLevelStructure}.


Van Driem (\citeyear{VanDriem2014}), while reflective of the state of description at its original time of writing, is at risk of being out of date by the current view of the family. For instance, more recent documentation work on Lhokpu, focussing on vocabulary and some verbal and nominal affixation, has suggested that the language is likely closely related to the Dhimalish languages of Dhimal and Toto \cite{Grollmann2018}. In addition to the linguistic evidence, the relationship is fairly easily justified geographically: Lhokpu is spoken in about three Chiwogs or village blocks (Singye, Sangloong Sangteng, and Thongsa Tobchhenthang) in Bhutan's Samtse Dzongkhag, which are between 15 and 30 kilometres upstream of Totopara in India, where the Toto language is spoken \cite{Basumatary2016}. Neither language is geographically contiguous, however, with Dhimalish, though interestingly \citeA{Grollmann2018} suggest that Lhokpu appears more closely related to Dhimal than Toto. \citeA{Grollmann2018} also note that the Dhimalish subfamily including Dhimal and Toto is perhaps not as well proven or established as its inclusion in \citeA{VanDriem2014} might suggest. Specifically, they argue that many of the shared forms are in fact shared much more widely than just Dhimal and Toto, and might therefore simply represent shared retentions from a much earlier proto-language. The geographic proximity of Lhokpu to Toto may also have little meaning for the historical development of the languages, as \citeA{VanDriem2004} suggests that Lhokpu (or its ancestor) may have at one time been much more widespread across western Bhutan, and in fact be a substrate under Dzongkha.

\citeA{Post2017} note that three further subfamilies, all spoken in Arunachal Pradesh in the eastern Himalayas, remain merely speculative. Van Driem's (2014) Hrusish, Siangic, and Midzuish subfamilies may also not yet be sufficiently proven. While in every case there is some level of evidence to support the groupings, Hrusish and Midzuish may be simply explained by high degrees of contact, and all three subgroups are lacking the level of description necessary to confidently prove any subgroups \cite{Post2017}. \citeA{BlenchPost2014} go even further, questioning whether or not it has been sufficiently proven that Siangic, Hrusish, Midzuish, Digarish, and Kho-Bwa are even \lfam\ languages at all. \citeA{Wu2022} use Bayesian phylogenetic analysis focussing on the languages of Arunachal Pradesh to attempt to shed some light on these groups (excluding the Siangic group, which was not included), and position them both as members of the family in general and in relation to nearby neighbouring languages. They ultimately conclude that they mostly likely are related to the wider \lfam\ family, though they diverged very early. Specifically, the Kho-Bwa languages may share a common ancestor from approximately 3,000 years before present, similar to that of the Ngwi-Burmese subfamily. Their divergence from their closest related subfamily of the Hrusish languages could have been yet earlier, at around 5500 years before present \cite{Wu2022}. Following this research, among others, the most recent update of Glottolog (at time of writing, 5.0) \citeA{glottolog} have separated the Hruso language from the \lfam\ language family entirely, keeping the other members of van Driem's Hrusish family (Bangru and Sajolang) as a subfamily within the family.

Van Driem (\citeyear{VanDriem2014}) also groups together the Tibetic and East Bodish groups into a single Bodish subfamily. While it is clear that East Bodish languages are not a subgroup of Tibetic languages \cite{Hyslop2017}, both \citeA{ZhangM2019Baye} and \citeA{ZhangH2020Baye}, as well as wider literature, seem to agree that Tibetic languages and East Bodish languages do in turn share a fairly close common ancestor and can be rightly considered to share a branch. The decision of what level to draw the line at here seems similarly arbitrary as it does with the Ngwi-Burmese subfamily, especially given some uncertainty surrounding the membership of Tshangla in this combined Bodish subfamily \cite{Thurgood2017STIntro}. Largely because of the large amount of literature available on the Tibetic\footnote{The term Tibetic is used here per \citeA{Tournadre2014}.} and East Bodish languages, and the availability of such data and insights to me from Gwendolyn Hyslop as my lead supervisor, East Bodish and Tibetic are separated here, contra \citeA{VanDriem2014}. 

The challenges in producing a representative sample of the languages in the \lfam\ family are a core problem in statistical models of \citesA{Sagart2019Baye}{ZhangM2019Baye}{ZhangH2020Baye}, but are also an issue that must be faced, in this project. In order to be able to make claims about the \lfam\ family as a whole, an even spread of the languages in the family need to be examined. This selection need not be as precise when attempting to analyse a smaller set of languages manually, compared to the much larger scale computational analysis undertaken in \citesA{Sagart2019Baye}{ZhangM2019Baye}{ZhangH2020Baye}. That is, this project must consider data from the whole family to the extent that clades of divergent or conservative languages are not missed or misinterpreted; however, it is not attempting to develop any statistical measures of the language family whereby a greater level of coverage in one area might skew results. For example, by virtue of its older academic tradition, there is a much wider field of literature covering specifically Tibetic languages, with specific detail into the field of this project (such as \citesA{Garrett2001}{GarfieldDeVilliers2017}{Woodbury1986}{DeLancey1986}{ZeislerForthcoming}, among countless others), but little more than grammatical sketches in other areas. While this imbalance in available data would be unrepresentative in a statistical analysis, the less quantitative and more qualitative approach of this project allows for a greater ability to account for these potential biases.

While the necessity to build a dataset that is as representative as possible is not nearly as strong with this project as with some others, the methodology by which the subfamilies were selected for \citeA{VanDriem2014} does lend itself to prioritising smaller language groups with less available research. Because it is focussed on well-established subfamilies, and given that language groups with higher levels of research will likely have genetic relationships established to a deeper time depth or to a higher level, we can expect to see large groups of well researched languages given under a single subfamily, while smaller groups of underresearched languages will be listed separately. That is to say, there is not necessarily a similar amount of diversity within each of van Driem's (2014) subfamilies; rather, there is likely more diversity in the larger, more widely researched and therefore better established ones.

In practice, the effect of this is visible in the number of languages present in each subfamily, using Glottolog \cite{glottolog} as a guide. At the largest end of the scale is the Ngwi-Burmese subfamily, with 127 languages listed on Glottolog, the majority of which (101) are specifically from the subfamily's Ngwi (Loloish) branch. Behind this with 54 languages is the Kukish branch. The language counts for the other subfamilies are given in \tabref{t:Methods:SubfamilyLanguageCount}.

\begin{longtable}{r c}
    Subfamily & Number of languages \\
    \hline
    Lepcha  & 1  \\
    \hline
    Meithei & 1   \\
    \hline
    Tshangla    & 1  \\
    \hline
    Lhokpu\footnote{Recent research suggests that Lhokpu may be closely related to the Dhimalish languages \cite{Grollmann2018}.}  & 1  \\
    \hline
    Dura    & 1   \\
    \hline
    Black Mountain  & 1  \\
    \hline
    Pyu & 1  \\
    \hline
    Gongduk & 1   \\
    \hline
    Mru\footnote{Some evidence suggests that Hkongso may be more closely related to Mru \cite{Wright2009}.} & 1 \\
    \hline
    Tujia\footnote{Glottolog gives north and south varieties, though \citeA{VanDriem2014} only gives one language.}   & 2  \\
    \hline
    Magaric & 2  \\
    \hline
    Chepangic   & 2   \\
    \hline
    Digarish    & 2   \\
    \hline
    Raji-Raute  & 2    \\
    \hline
    Siangic & 2 \\
    \hline
    Midzuish\footnote{Called Kman-Meyor in Glottolog.}    & 2 \\
    \hline
    Karbi\footnote{Glottolog gives Hills and Amri varieties, though \citeA{VanDriem2014} only gives one language.}   & 2 \\
    \hline

    Dhimalish   & 2-3 \\
    \hline
    Ersuish & 3   \\
    
    \hline
    Hrusish\footnote{The Hruso language has been separated in Glottolog 5.0 to be a complete isolate, though Bangru and Sajolang remain under the label Miji.} & 3  \\
    \hline
    Nungish & 3    \\
    \hline
    Bái\footnote{Membership of Caijia and Longjia unclear \cite{Lue2022}.}    & 3-6  \\
    \hline
    Newaric & 5    \\
    \hline
    Nàic    & 5    \\
    \hline
    East Bodish\footnote{Grouped with Tibetic into ``Bodish'' in \citeA{VanDriem2014}.} & 7    \\
    \hline
    Ao  & 7    \\
    \hline
    Kho-Bwa & 7 \\
    \hline
    Zeme    & 7    \\
    \hline
    rGyalrongic & 8   \\
    \hline
    Kachinic\footnote{Called Jingpho-Luish in Glottolog}    & 9   \\
    \hline
    Tangkhul    & 9  \\
    \hline
    Tani    & 10 \\
    \hline
    Angami-Pochuri  & 10 \\
    \hline
    Qiangic & 12 \\
    \hline
    Tamangic    & 13  \\
    \hline
    West Himalayish & 15 \\
    \hline
    Karenic & 20 \\
    \hline
    Sinitic & 26  \\
    \hline
    Brahmaputran\footnote{The Glottolog ``Brahmaputran'' branch also includes Kachinic/Jingpho-Luish, which \citeA{VanDriem2014} separates.}    & 29  \\
    \hline
    Kiranti & 31 \\
    \hline
    Tibetic  & 44  \\
    \hline
    Kukish  & 54    \\
    \hline
    Ngwi-Burmese & 127 \\
    \hline
    \caption{The number of languages in each of van Driem's (2014) subfamilies, per Glottolog \cite{glottolog}.}\label{t:Methods:SubfamilyLanguageCount}
    \end{longtable}


Languages themselves were selected based on available documentation, using Glottolog \cite{glottolog} as a reference database for descriptive work available. To avoid accidentally selecting a particularly aberrant language and not properly representing a given subfamily, an initial goal of two languages per subfamily was set. For larger subfamilies, such as Ngwi-Burmese, Kukish, Tibetic, and Kiranti, more languages were surveyed both in an attempt to ensure even coverage of the larger subfamily. This also reflects the greater frame of reference for these languages.

There were a number of cases where it was possible to even survey two languages in a subfamily. The first case is the internal isolates discussed in Section \ref{ss:THOverview:Subfamilies}. In addition, there are a number of smaller subfamilies that do not have the descriptive coverage to allow two languages to be surveyed; either there was only one described language in the subfamily, or there were multiple, but only one has any coverage of epistemic marking. This last point poses a problem in that in languages with more limited descriptive analyses available, it is difficult to tell if reference to epistemic marking is omitted because it would be outside the scope of the current stage of description, or because it simply does not exist in the language. There were also a small number of subfamilies for which no descriptive data could be found. In order to fill the gap in the data for Lhokpu, fieldwork in the Lhokpu in Bhutan was undertaken as part of this project, detailed in \appref{s:Methods:FieldMethods} (though, as mentioned above, Lhokpu may well be better classified as Dhimalish). Gongduk, another internal isolate spoken in Bhutan \cite{VanDriem2001b}, also had no available data. During the course of this project, a grammar of Raji (Raji-Raute: Nepal) appears to have been published and listed on Glottolog \cite{glottolog}, written by Dubi Nanda Dhakal, but I have not been able to access this. There is similarly no documentation available for the Digarish languages in English, and as such my information on these languages comes from personal communication with Rolf Hotz Molina and Naomi Peck, who are both working on PhD projects on these languages. For Digarish, and Midzuish, there are unpublished sketch grammars by Roger Blench available on his website, though these are specifically not for wider use so were not included in the survey.\footnote{\url{http://www.rogerblench.info/Language/NEI/Lingres/NEIlingres.htm}} Glottolog also notes a body of descriptive work on Kaman (Midzuish) in Mandarin, which was also not accessible for this project.

In terms of the much larger families, oversampling the Ngwi-Burmese family by assessing more than the two languages as discussed above can at least in part alleviate the arguable underrepresentation of the languages when accounting for the size of the subfamily. That being said, only adding one or two extra languages will not come close to balancing this statistic across the data. This is especially the case given that there are a number of subfamilies of which 50-100\% of languages have been sampled. Again, the qualitative nature of the analysis being undertaken means that the essentially unavoidable imbalance of this measure (languages sampled in a subfamily as a percentage of that subfamily's total number of languages) does not pose a problem as it does in quantitative statistical analyses.

With this goal of two languages per subfamily where possible---to be expanded upon after in larger subfamilies---languages were selected by the breadth of description available, as well as the recency of the description. Full published grammars were preferred over doctoral and Master's theses, and newer studies were preferred over older ones. Regardless of the level of review or detail for the publication, analyses were taken (where no alternative analyses exist, see Sunwar discussed in Section \ref{ss:Description:Conflicts}) to be accurate, and no attempts were made to reanalyse data presented. In some cases, the usage of terminology is discussed in relation to the description provided, and (especially in Section \ref{ss:Description:ClassByFunction} and Chapter \ref{c:Discussion}) theoretical conclusions are drawn about data based on the analysis available, but beyond what is explicitly stated. In this latter case, it is possible that there exists further data which would disprove the analysis synthesised here, but were not included in the publication as they were not seen to be necessary by the original author.

Newer studies were preferred as they are more likely to discuss the categories and functions at issue in this thesis. As discussed in Chapter \ref{c:Introduction}, studies into perspective-marking in \lfam\ languages have become significantly more common over the past two decades. Much of the research undertaken prior to that, and even more so prior to the publication of \citeA{ChafeNichols1986}, either does not consider perspective-marking at all, or does so in a way that is less immediately accessible in the context of contemporary theories and frameworks. It is often simply much easier to find the relevant information in a more recent publication.

As was mentioned above, some further data has been used from languages that appear regularly or in relevant places throughout the literature, even if they were not initially selected. In particular, discussions of the epistemic system in Lhasa Tibetan and early Tibetic languages are common throughout the literature (see, for instance, \citesA{DeLancey2012}{Garrett2001}{Hill2012}{Hill2014}{Zemp2021}). This is particularly relevant to the diachronic considerations discussed in Chapter \ref{c:History}. These languages are in addition to the set of languages surveyed systematically here, and as such are not included in the list of surveyed languages given in \appref{a:TableOfLanguages}.

\section{Data Collation}\label{s:Methods:Schema}
\subsection{Database Overview}
After initial data collection as described above, the notes and summaries written on each language (with reference back to the source material) were summarised into a database that marked whether or not a certain feature was present in a certain form or function in a given language. The database provides an easy-to-access and easy-to-reference summary of key features that were expected to be relevant to the analysis stages of the project. A limitation of this methodology is that it is reductive, and no such format will be able to succinctly \textit{and} completely describe the full nature of even a single paradigm in a language. It is descriebd here to explain how data from the numerous different sources were collated and referenced throughout the process of analysis, to be able to more easily see any trends that did emerge and later more easily locate relevant data and examples. This does not mean that the features noted for each language at this stage are perfectly reflective of the actual typological categories described in Chapter \ref{c:Description}. Rather, the more schematised summaries created here were used to come develop those categories and act as a point of reference to the original sources. Given the rapidly growing body of descriptive work on \lfam\ languages, particularly with regards to epistemic marking, and the scope of this project limiting my ability to confirm my understanding of the available analyses, I am treating this database as an ongoing project which I will continue to update as a resource for myself, and perhaps others moving forward. With this in mind (and to avoid formatting challenges), the database is available online rather than being included as an appendix to this thesis. In doing so, it can continue to be updated and refined as necessary. The database is available at \url{cbodnaruk.com/database}.

The database includes four sections: the scope and form of a given marking or paradigm (\sref{ss:Methods:ScopeForm}), the function of the marking(s) (\sref{ss:Methods:Functions}), the extent to which addressee perspective has been described or suggested in the literature for the markings (\sref{ss:Methods:Addressee}), and other features such as the variety of functions in a single paradigm, or the presence of engagement marking on nominal or demonstrative structures (as per \citeA{EvansBergqvistSanRoque2018b}, \sref{ss:Methods:Others}). Additionally, the database records if anything of note in regards to these criteria (epistemic marking) could be found in the literature at all (in the few cases where the answer to this is no, the rest of the entry is empty). In cases where a single language has multiple varied markings or paradigms, one record in the entry will contain multiple `yes' responses to some features. That is, for instance, a system could mark epistemic meaning as a verb suffix and a clause final particle, and no single designation is mutually exclusive with any other (aside from the question of the presence of said marking in the first place).

The analysis undertaken was, for reasons outlined in Section \ref{ss:Methods:RepSample}, primarily qualitative. This involved reading descriptions of epistemic systems in publications and sorting them into this database, and then using this broad overview to begin to draw preliminary theoretical and typological conclusions about the data overall. These preliminary conclusions were then able to be compared specifically to the collected data by referencing the summaries collected in the database. That is, the qualitative nature of the analysis meant that the database itself was not generally used directly for analysis, but as a point-of-reference database to quickly find relevant data and descriptions from the overall sample.

\subsection{Scope and Form}\label{ss:Methods:ScopeForm}
This feature records where in a given clause the marking is located, and with what scope. Specifically, whether the marking occurs on Verbal Morphology, at the Noun Phrase level, on the Verb Phrase, or as a Discourse particle (i.e. at the speech act level). The specific forms of a given marking or paradigm are not recorded in the summary.

For example, in Kurtöp (East Bodish: Bhutan), the obligatory epistemic marking paradigm in the perfective aspect marking features such as mirativity, egophoricity, and evidentiality, appears in the form of a closed paradigm of compulsory suffixes attached to the main verb of a clause \cite{Hyslop2018}, shown in \exref{e:Methods:KurtopScope1} and \exref{e:Methods:KurtopScope2}. This is recorded with a + in the ``Verbal Morphology'' column of the Scope and Form section. Additionally, however, the same distinctions are marked in other domains with the use of specific copulas, recorded with a second + in the ``Verb Phrase'' column.

\begin{exe}
\ex Verbal Morphology \label{e:Methods:KurtopScope1}
\begin{xlist}
\ex
\gll ngat ge-\textbf{shang} \\
1.\textsc{sg.abs} go-\textsc{\textbf{pfv:ego}} \\
\glt `I went.' (p.300)

\ex
\gll tshe khit ge-\textbf{mu} \\
\textsc{dm} \textsc{3.sg.abs} go-\textsc{\textbf{pfv:ind}} \\
\glt `Then he left.' (p.303)

\end{xlist}

\ex Copulas \label{e:Methods:KurtopScope2}
\begin{xlist}
\ex 
\gll mau zangu ngaksi \textbf{nawala} \\
\textsc{dn} zangu \textsc{quot} \textsc{\textbf{cop.exis}} \\
\glt `There is this (thing) down there called ``zangu''.' (p.310)

\ex
\gll Hâ-pa=the \textbf{nâ} \\
Hâ-\textsc{dnz=indef} \textsc{\textbf{cop.exis.mir}} \\
\glt `He is a Hâpa (from Hâ).' (p.310)
\end{xlist}
Kurtöp \cite[East Bodish: Bhutan,][]{Hyslop2017}
\end{exe}

\subsection{Function of Marking}\label{ss:Methods:Functions}
The Function of Marking section describes the function of the form or forms in a paradigm in relation to the cross-linguistic categories of epistemic modalilty (EM), evidentiality, egophoricity, mirativity, and engagement. It is split into two subsections, the first recording instances of `prototypical' forms of a given category. That is, a column will have a + for instances where the given paradigm marks only the category in question, and where the paradigm closely reflects the definition of the category in the literature.

For instance, the conjunct/disjunct paradigm in Kathmandu Newar (Newaric: Nepal), as originally described in \citeA{HaleNewar1980}, fits very closely with the definitions given for egophoricity to the point that it given as an illustrative example of egophoricity at its simplest level in \citeA[3]{EgoIntro}. As such, it would be marked with a + in the egophoricity column of the prototypcial subsection. In a language such as the aforementioned Kurtöp, however, the marking of mirativity, egophoricity, and evidentiality all occur on different forms in the single paradigm, which as a result does not strictly adhere to the definitions given for such categories in the literature \cites{DeLancey2012}{EgoIntro}{Aikhenvald2018Intro}.

In these cases where it would be insufficient to represent a paradigm as adhering to any well-defined category, another subsection recording loose examples of a given category is used. In this case, while not strictly fitting the definitions for a single mirative, egophoric, or evidential paradigm, the paradigm would be marked as having some broad instance of all three.

\subsection{Described addressee-perspective}\label{ss:Methods:Addressee}
This section notes whether or not there are any cases of addressee-perspective either described or exemplified in the available literature on a given language, either in interrogative or declarative structures. This assessment is, even more so than the other areas presented here, reliant on the extent of literature coverage on a given language. It takes a certain level of detail in description for epistemic markers at all to be described, but a greater level of detail again for their interactions with the perspectives of speech act participants to be considered. For example, while \citeA{Lidz2010} gives a detailed and highly exemplified description of evidential and other epistemic marking in Yongning Na (Naic: PRC), she does not give any detail of the use of the evidential forms in questions, if they are allowed at all. On the other hand, \citeA[408]{Bodt2020} explicitly describes a mirative meaning of the Duhumbi (Kho-Bwa: India) copula \textit{le} as marking the perspective of the addressee in declaratives, noting that the ``speaker considers the information of the proposition with copula \textit{le} as new to the addressee and relevant to him in the moment of speaking...'', contrasted with the copula \textit{beʔ}, which does not carry this mirative meaning, exemplified in \exref{e:Methods:Duhumbi}. While it is easy to give a definitive `yes' where evidence is given in the literature, in many cases such as Yongning Na above there is simply no reference made to any intersubjective-like features or cases (such as the common occurrence in evidentials in interrogatives \cite{Aikhenvald2018Intro}). In these cases it is not possible to say that grammaticalised intersubjectivity \textit{does not} exist, but rather only that it has not yet been readily observed.

\begin{exe}
\ex \label{e:Methods:Duhumbi}
\begin{xlist}
\ex 
\gll Gonpa pʰu t\textsuperscript{ɕ}haŋkʰo le \\
temple mountain on-\textsc{loc} \textsc{cop.le} \\
\glt `The temple is on top of the mountain.' [New to the addressee]

\ex 
\gll Gonpa pʰu t\textsuperscript{ɕ}haŋkʰo beʔ \\
temple mountain on-\textsc{loc} \textsc{cop.ex} \\
\glt `The temple is on top of the mountain.' [Neutral to the addressee]
\end{xlist}
Duhumbi \cite[Kho-Bwa: India][408, notes in relation to addressee added by me]{Bodt2020}
\end{exe}

\subsection{Other Features}\label{ss:Methods:Others}
Finally, the database records four other possible features that might be useful to quickly access and reference. These are the presence of mixed function paradigms, that is paradigms with forms that do not fit into a single established cross-linguistic category, the presence of nominal engagement as referred to in the introduction to this section (Section \ref{s:Methods:Schema}), the diachronic source for the forms documented (where given), and whether or not forms are obligatory.

The first of these, the mixed paradigms, could potentially fairly closely mirror the cases where no prototypical example of a category is identified, but with a key difference. Systems whereby some form of subjective or intersubjective marking (such as evidentiality in the example given below) is spread across a number of different domains of a language's grammar do not fit into the aforementioned `prototypical' category, nor do they fit into this `mixed paradigm' group. For instance, in Meithei \cite[295]{Chelliah1997}, evidentiality is marked across a number of different domains in the grammar, such as derivational morphology, clitics, and complementation, as opposed to being marked by a single discrete paradigm. It is worth noting that this ``scattered'' marking of evidentiality is widespread, and is by no means ignored in the literature \cite[23]{Aikhenvald2014}. The example of Kurtöp, given above, however, does represent a mixed paradigm, where a single paradigm contains multiple different functional categories.

The presence of nominal engagement in \lfam\ languages has not been widely documented, but has the potential to give insights into the development of verbal or clause-level marking in the wider family where they do appear. Nominal engagement has been documented in Purik Tibetan \cite[Tibetic: India:][]{Zemp2021} and in Phola \cite[Ngwi-Burmese, PRC:][]{GonzalezPerez2023}.

For example, in Purik Tibetan, demonstratives \textit{de} and \textit{e} are functionally contrastive in certain positions by the attentiveness of the addressee, with \textit{e} working to redirect their attention to a referent \cite{Zemp2021}, as opposed to a state of affairs already clear to both speech act participants. Examples of this are given in \exref{e:Methods:PurikDem}.

\begin{exe}
  \ex \label{e:Methods:PurikDem}
  \begin{xlist}
    \ex 
    \gll kulik-po di-ka pʰjal-la \textbf{de} \\
    key-\textsc{def} this-\textsc{loc} hanging-\textsc{dat} \textsc{\textbf{top}} \\
    \glt `The key’s hanging here (right in front of your eyes).' (p.412)

    \ex
    \gll sŋuntʃoqtʃoq e \\
    deep.green \textsc{top2} \\
    \glt `Look, how green it is over there!' (p.413)
  \end{xlist}
  Purik Tibetan \cite[Tibetic:India][]{Zemp2021}
\end{exe}

\subsection{Example}\label{ss:Methods:MagarExample}
This section works through the classification of an example language to show how it this schema has been used in practice. Here, the description of Magar by \citeA{GrunowHarsta2008} is used as it features distinctions across a number of the areas being recorded in the schema. The epistemic system in Magar is split across all three discrete systems, marking inferential and reportative evidentials, miratives, and quotatives.

\subsubsection{Brief Description}
Miratives in Magar are marked in a variety of forms, either nominalisations or related constructions. In \exref{e:Methods:MagarMirativeIntro}, mirativity is marked the nominaliser \textit{-o} followed by a grammmaticalised copula now carrying aspectual meaning \textit{le}. 

\begin{exe}
  \ex\label{e:Methods:MagarMirativeIntro}
  \begin{xlist}
    \ex
    \gll thapa i-laŋ le \\
    Thapa \textsc{p.dem-loc} \textsc{cop} \\
    \glt `Thapa is here.' (non-mirative)

    \ex 
    \gll thapa i-laŋ le-\textbf{o} \textbf{le} \\
    Thapa \textsc{p.dem-loc} \textsc{cop-\textbf{mir}} \textsc{\textbf{impf}} \\
    \glt `(I realize to my surprise that) Thapa is here!' (mirative)
  \end{xlist}
  \cite[Magar,][480]{GrunowHarsta2008}
\end{exe}

\citeA[480]{GrunowHarsta2008} also gives one other nominaliser that can convey mirative meaning, a form \textit{cyo \~ cʌ}. The distribution of these forms tends to follow the person of the subject of the clause, with the latter \textit{cyo \~ cʌ} form mainly used for third-person subjects and the former \textit{-o le} for subjects who are speech act participants, though \exref{e:Methods:MagarMirativeIntro} is a clear exeption to this. The forms appear to always reflect a speaker-origo, including in questions, or can reflect a character origo in narratives. Examples of speaker-origo in narratives for rhetorical effect are also given, though are described as ``feigned'' (p. 493). One example, reproduced in \exref{e:Methods:MagarMirativeAddOri}, may well show addressee-origo in a declarative construction. Here, the mirative is used both in an interrogative (see \exref{e:Methods:MagarMirativeAddOri:q}) and confirmation of that question (see \exref{e:Methods:MagarMirativeAddOri:a}). Given information being confirmed by the speaker cannot be in the moment new to said speaker, an alternative explanation for the function of the mirative here is necessary. \citeA[486]{GrunowHarsta2008} explains the use of mirative in the response as not marking information as new to the speaker, but marking information as something she cannot mentally integrate. An alternative view, at least seeing this data in isolation, is that the interrogative here is an explanation of disbelief on the part of the first speaker, hence marked with the mirative, and that the second speaker is also reflecting the first speaker's expression of disbelief. This is to say that the second speaker is agreeing that the first speaker is perhaps correct to use the mirative construction, arguably reflecting their addressee's perspective.

\begin{exe}
  \ex \label{e:Methods:MagarMirativeAddOri}
  \begin{xlist}
    \ex \label{e:Methods:MagarMirativeAddOri:q}
    \gll mi-ja ma-phunɦ-o le-sa si-cʌ ale \\
    \textsc{poss}-child \textsc{neg}-{give birth}-\textsc{mir} \textsc{impf-infr} die-\textsc{att} \textsc{cop} \\
    \glt `She just died, undelivered!?'

    \ex \label{e:Methods:MagarMirativeAddOri:a}
    \gll ã ma-phunɦ-o le-a \\
    yes \textsc{neg}-{give birth}-\textsc{mir} \textsc{impf-pst} \\
    \glt `Yes, undelivered!'
  \end{xlist}
  \cite[Magar,][487]{GrunowHarsta2008}
\end{exe}

Evidentiality in Magar is also marked across multiple grammatical domains. Direct first-hand evidence, or statements of general cultural fact are unmarked, inferential evidentials are marked with the verb suffix \textit{-sa} and reportatives are marked with a particle \textit{ta}. \citeA{GrunowHarsta2008} gives a minimal triplet of these three meanings, reproduced in \exref{e:Methods:MagarEvidIntro}.

\begin{exe}
  \ex \label{e:Methods:MagarEvidIntro}
  \begin{xlist}
    \ex Direct
    \gll ho-se taɦ-raɦ-a \\
    \textsc{d.dem-def} reach-come-\textsc{pst} \\
    \glt `He has arrived.' (I see him.)

    \ex Inferential
    \gll ho-se taɦ-raɦ-le-sa-a \\
    \textsc{d.dem-def} reach-come-\textsc{impf-infr-pst} \\
    \glt `He has arrived.' (I see his bag.)

    \ex Reportative
    \gll ho-se taɦ-raɦ-a ta \\
    \textsc{d.dem-def} reach-come-\textsc{pst} \textsc{rep} \\
    \glt `He has arrived.' (They say.)
  \end{xlist}
  \cite[Magar,][497]{GrunowHarsta2008}
\end{exe}

In addition to these forms, there is also a quotative construction for direct quotes, formed with a subordinate clause and the speech verb \textit{de} `say' (p. 498). Unlike in some languages, the grammaticalised forms given in \exref{e:Methods:MagarEvidIntro} are not obligatory, though it is not clear how this factors into Grunow-Hårsta's analysis of the direct evidential as null-marked.

Both the inferential and reportative are used in interrogative structures, where they reflect the perspective of the addressee. Additionally, both are used in narratives, but with different functions. The inferential evidential is used in narratives when narrating a story told in images (see the Family Problems Picture Task in Section \ref{p:Methods:FamilyProblems}), as well as reflecting the character's perspective, both in direct speech from the character, and in narration. The reportative is used very widely in narrative, but references exclusively speaker perspective, marking the speaker's source of information for the narrative itself. The forms \textit{-sa} \textsc{infr} and \textit{ta} \textsc{rep}, being marked respectively as a verb suffix and separate particle, can cooccur (p. 513). This marks reportative evidence for the speaker, whose sources in turn had inferential evidence.

Independent of these forms, Magar also has a set of epistemic modal markers which can cooccur with both miratives and evidentials, which can in turn cooccur with each other. On these grounds, Grunow-Hårsta describes these as three distinct systems within the grammar. Potentially, given the formal difference in the location of the marking between the inferential and reportative marking, this could be further expanded to the point that there is no single paradigm of mutually exclusive forms filling the same grammatical slot in Magar. This ``scattered'' system is not uncommon in the region \cite[480-481]{GrunowHarsta2008}. In terms of the typology set out in \cite{Aikhenvald2004}, Magar would be a B1 system, marking visual, inferred, and reported evidence. As will be discussed in detail in Chapter \ref{c:Discussion}, this does not manage to fully characterise the overall epistemic marking system (in a broad sense) of scattered morphemes in Magar. This is to say, in a system where all forms are marked in different grammatical slots and can cooccur with other in seemingly any combination, there is less justification to treat the forms traditionally classified as evidentiality as separate systems beyond the traditional theoretical boundaries which I am advocating against in this analysis.

\subsubsection{Representation in the summarised database}
Table \ref{t:Methods:SchemaExample} shows the representation of this description of Magar in the database described above. The first section, the Language Metadata, gives basic information on the language, including its coordinates according to Glottolog. The next section, Scope and Form, notes that there is evidence of grammatical epistemic marking, and that it appears both as verbal morphology (the nominaliser component of the mirative and the inferential suffix) and as aa discourse particle or at the speech act level (the copular component of the mirative and the reportative marker). The Function of Markings section notes that there is (albeit with limited description) a system of epistemic modality marking, as well as an `other' system that does not easily fit into a single group. This other system is then noted in the next section to feature functions of both mirative and evidential meanings, referring to the scattered system described above. Described Intersubjectivity notes that there is clear reference to the addressee's perspective in interrogative constructions, and potentially in declaratives (see \exref{e:Methods:MagarMirativeAddOri}). Finally, in Other Features, no diachronic source is noted as none was given by \citeA{GrunowHarsta2008}, the fact that the system is not obligatory is noted, as well as the lack of mixed paradigm or noted nominal engagement marking.
\begin{table}
  \caption{Magar, described in \citeA{GrunowHarsta2008}, represented as an entry in the summarised database described in Section \ref{s:Methods:Schema}.}\label{t:Methods:SchemaExample}
  \begin{tabular}{|l|l|}
  \hline\hline
  \multicolumn{2}{|l|}{Language Metadata} \\ \hline
  Language & Magar \\ \hline
  Subfamily & Magaric \\ \hline
  Source & Grunow-Harsta 2008 \\ \hline
  Glottolog Coordinates & 27.41, 87.06 \\ \hline \hline
  \multicolumn{2}{|l|}{Scope and Form} \\ \hline
  Evidence of epistemic marking & + \\ \hline
  Verb Morphology & + \\ \hline
  Noun Phrase &  \\ \hline
  Verb Phrase &  \\ \hline
  Discourse particle/speech act level & + \\ \hline \hline
  \multicolumn{2}{|l|}{Function of Markings} \\ \hline
  EM & + \\ \hline
  Ev &  \\ \hline
  Ego &  \\ \hline
  Eng &  \\ \hline
  Mir &  \\ \hline
  Other? & + \\ \hline \hline
  EM &  \\ \hline
  Ev & + \\ \hline
  Ego &  \\ \hline
  Eng &  \\ \hline
  Mir & + \\ \hline \hline
  Term(s) used in source & Mirative, Direct/Factual, Reportative, Inferred \\ \hline \hline
  \multicolumn{2}{|l|}{Described Addressee-Perspective} \\ \hline
  Evidence of AP in interrogative structures? & + \\ \hline
  Evidence of AP in declarative structures? & +? {[}486-487{]}, (c) reads a bit like IS \\ \hline \hline
  \multicolumn{2}{|l|}{Other Features} \\ \hline
  Diachronic Source? &  \\ \hline
  Obligatory & No \\ \hline 
  Mixed Paradigm & No \\ \hline
  Nominal Engagement & No \\ \hline \hline
  \end{tabular}

  \end{table}

\subsection{Languages with conflicting analyses}\label{ss:Description:Conflicts}
A challenge in taking the analyses presented in the literature at face value is that, in cases where multiple descriptions of a single language exist (as discussed in Section \ref{ss:Description:StateOfDescription}), there may be different, conflicting analyses of a particular form or function. This section presents a number of examples of cases where a decision had to be made, and discusses why that decision was made the way it was. 

\paragraph{Sunwar}
In his initial descriptions of mirativity, \citeA{DeLanceyMirativity1997} gives Sunwar (Kiranti: Nepal) as an example of a language showing grammaticalised mirativity. In particular, DeLancey describes the copulas \textit{tshə} and \textit{'baak-} as being distinguished based on the newness of knowledge. DeLancey reports that the use of each copula is conditioned independently of the source of the speaker's knowledge (evidentiality), but is rather conditioned by whether or not the information is known without qualification by the speaker (\textit{tshə}) or is information they have just learned, through any of reportative, inferential, or direct evidence (\textit{'baak-}). \exref{e:Description:SunwarMirative} shows this distinction in a minimal pair, with the non-mirative used in situations where the speaker has perhaps lived in Kathmandu and is familiar with Tangka and the mirative form used in situations where the speaker perhaps did not know Tangka was in Kathmandu but had just seen him, or had just been told he was there \cite[42]{DeLanceyMirativity1997}.

\begin{exe}
        \ex\label{e:Description:SunwarMirative}
        \begin{xlist}
                \ex 
                \gll Tangka Kathmandu-m tshaa \\
                Tangka Kathmandu-\textsc{loc} \textsc{tsha.3sg} \\
                \glt `Tangka is in Kathmandu.' (non-mirative)

                \ex
                \gll Tangka Kathmandu-m 'baâ-tə \\
                Tangka Kathmandu-\textsc{loc} exist-\textsc{3.sg.past} \\
                \glt `Tangka is in Kathmandu.' (mirative)
        \end{xlist}
        Sunwar \cite[Kiranti: Nepal,][41-42]{DeLanceyMirativity1997}
\end{exe}

\citeA{Borchers2008} disagrees with this analysis, though concedes that she and DeLancey are working with data from different Sunwar-speaking communities, and notes that DeLancey's analysis is working with a smaller corpus than hers. Borchers suggests instead that \textit{'baak-} ``is used to express the general way that things are'', whereas \textit{tshə} ``denotes the concrete and recent state of affairs'' \cite[164]{Borchers2008}. \citeA{Hill2012} is also critical of DeLancey's analysis, though in an overall argument against mirativity as valid cross-linguistic category. That being said, Hill's criticism of the mirative analysis, while referencing \citeA{Borchers2008} for support, relies only on a reanalysis of the meagre data presented in \citeA{DeLanceyMirativity1997} (here in \exref{e:Description:SunwarMirative}) and a discussion of edge cases one would not reasonably expect DeLancey to have discussed given the level of detail in the description given in his paper.

The question thus becomes one of which analysis to follow for this typology. That is, in order to enter data from Sunwar into the database, we must make a decision about whose analysis to follow. In this case, given Borchers, at least by her accounts, worked with substantially more data, and spent a much longer time in the field than DeLancey (who worked with a single speaker living in the United States \cite{DeLanceyMirativity1997}), the analysis in \citeA{Borchers2008} was used. This is, in all reality, a fairly minor decision. It is, in this case, a single point of data in a substantially larger database, and despite Hill's (2012) strong criticism of DeLancey's description, \citeA{Borchers2008} does give a number of possible reasons for the difference in analysis, and does not appear to go to the same length as Hill in actively attempting to refute DeLancey. There also continues to be other languages analysed as marking mirativity in the sample, and as such mirativity as a concept is still considered in this typological analysis.

\paragraph{Lhasa Tibetan}
Epistemic marking in Lhasa Tibetan varies between equative copular clauses, and existential copular, and verb clauses. Specifically, there are two epistemic bases in the equative copula system, compared to three in the existential copulas and verbal morphology \cite{DeLancey2017Tibetan}. These forms are given in Table \ref{t:Description:LhasaEpistemics}, adapted from \citeA[11]{Garrett2001} and using his labels for the 2-3 evidential bases. The precise labelling of these bases in a theoretical sense has been debated in the literature.

\begin{table}\caption{The Lhasa Tibetan epistemic system, adapted from \citeA[11]{Garrett2001}.}\label{t:Description:LhasaEpistemics}
        \begin{tabular}{l|c|c|c}
         & Ego & Direct & Indirect \\ \hline
        Equative Copulas & \textit{yin} & \multicolumn{2}{c}{\textit{red}} \\
        Existential Copulas & \textit{yod} & \textit{ḥdug} & \textit{yodred} \\
        Verbal Morphology (past)\footnote{This paradigm is conditioned by both epistemics and tense/aspect. For the sake of simplicity, the past tense paradigm is given here} & \textit{-pa-yin} & \textit{-song} & \textit{-pa-red}
        \end{tabular}
        
 \end{table}

\citesA{DeLanceyMirativity1997} suggests that, in the three-base systems, the \textit{ḥdug} form represents information with an immediately accessible information source, which he analyses as mirative. As with Sunwar, \citeA{Hill2012} argues against this analysis, rather arguing that the perceived immmediacy of the evidence is a result of the actual condition for the use of the base being the presence of direct sensory evidence. This analysis of this form as marking direct sensory evidence is also followed by \citeA{Garrett2001}, and is visible in the labelling of Table \ref{t:Description:LhasaEpistemics}. More recently, \citeA{DeLancey2017Tibetan} takes a stance between the two, suggesting that the form is conditioned by direct perception, but that (at least in some cases), this also logically suggests an immediacy of the origo's evidence that the information is also new (though not necessarily unexpected).

\citeA{DeLancey2017Tibetan} also give a different analysis of the conditions for the use of \textit{yodred} forms to \citeA{Garrett2001}. While Garrett suggests that the forms are dependent on some indirect information source, such as hearsay or inference (though this is a major simplification of Garrett's very detailed analysis of the usage of the form), DeLancey suggests an analysis in which the \textit{yodred} forms mark evidentially generic information, or information that is known without qualification or because it is simply generally known \cite[392]{DeLancey2017Tibetan}. A similar claim for a more generic factual evidential function is made for other Tibetic languages by \citeA{Zemp2020}, specifically referring to copulas. In some cases, the factual or netural function is described for the cognate of the \textit{yin} form (see also \citeA{Bodnaruk2023a}) contrasted against an evidentially marked alternative, and some cases for the form in contrast to the cognate of the \textit{yin} form, which in these latter cases marks specific speaker involvement. \citeA[39]{Zemp2020} extends this second category to the equative copulas in Lhasa Tibetan, in which the \textit{yin} form marks personal involvement, while the \textit{red} form is evidentially neutral.

The distinction between \textit{yin} and \textit{red} has also been described as egophoric \cite{EgoIntro}, and does on the surface follow the expected pattern of egophoric contrasts. \citesA{Hill2017}{Gawne2017} argue against egophoricity as a separate cross-linguistic category, but rather frame it as a specific evidential base contrasted with other evidential meanings and not with a \textsc{non-ego} form that is simply defined against it. This is similar to an analysis proposed by \citeA{Garrett2001}, who described this as an evidential base labelled ``ego''. This ``egophoric evidential'' (as opposed to an egophoric marker in a theoretically distinct category) seems to generally agree with Zemp's (2021) analysis, though focusses on the 3-way distinction seen in other areas of the Lhasa Tibetan grammar. 

Unlike in Sunwar, the actual usage of the forms in Lhasa Tibetan is not in any of the literature substantially at odds. Rather, as the brief description above begins to summarise, the analyses differ in purely theoretical terms, questioning which cross-linguistic categories and theoretical frameworks and lenses best represent the well-described usage of the forms given in Table \ref{t:Description:LhasaEpistemics}. This discussion is by no means unnecessary, but is importantly not one of description per se. In fact, the clearly blurred boundaries between the categories (the 3-term system can and has been described as mirative, evidential, egophoric, and combinations of those three) begins to suggest that perhaps this siloed approach of analysis is insufficient here, a direction that \citeA{Hill2017} begin to move in, but perhaps face challenges in attempting to collapse the distinctions solely into the framework of evidentiality. Paradigms that appear to mark more than one category of epistemic marking will be further discussed in Section \ref{sss:Description:MixedSystems}, and the theoretical implications in detail in Section \ref{s:Discussion:Mixed}.