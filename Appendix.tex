\appendix
\chapter{Field work and original data collection}\label{s:Methods:FieldMethods}
\section{Lhokpu}
During this project, primary data was collected on Lhokpu (Subgroup unclear: Bhutan) to fill in a gap in the available data. As has been discussed above, Lhokpu has been hypothesised as aligning with the Dimalish subfamily, and while the conclusions in \citeA{Grollmann2018} are certainly well supported, they have not been totally proven. As such, Lhokpu has been included in this study as an internal isolate of unclear subfamily per \citeA{VanDriem2014}.

Lhokpu is spoken in a small number of villages in the Dophuchen and Tading gewogs of Samtse district, in south-western Bhutan. There are two non-contiguous groups of speakers, one located about 15km up the Amo Chhu (River) from the other, which is in turn a similar distance upriver of Totopara, the village in which the potentially closely related language Toto is spoken. \citeA{Grollmann2018} estimate approximately 2500 speakers across all villages, though more recent estimates (Mareike Wulff, p.c. 2024) are as low as 800, many (potentially all) of whom also speak Nepali, and to a lesser extent, Dzongkha and English.

For this project, the data collected needed to efficiently reflect any epistemic-marking system that may exist in the language, as time in the field was strictly limited. With the time available (especially given this was a smaller component of the larger project), recording large amounts of natural and unprovoked dialogue, stories, and other content, and then identifying the relevant forms therein, was not feasible. It can also be difficult to establish what the actual epistemic content of forms used in these settings would be, as it may not be totally clear to a researcher what the relationships of the speaker and addressee are to any given piece of information. On the other hand, it has been noted that speakers are not typically consciously aware of epistemic distinctions in language, and as such it can be difficult to ask or directly elicit them \cites{Gawne2013}{Grzech2020}. As such, a middle-ground approach was used here, in which elicitation activities were used to generate naturalistic conversation data within an established and controlled epistemic context. These activities in particular draw from the work of \citesA{Gawne2013}{Gawne2020} and her discussion on using such tools, as well as \citeA{Grzech2020}.

\section{Field Methods and Elicitation Activities}
By establishing a set of contrastive epistemic contexts across the elicitation activities run, it is possible, at least to a certain extent, to ascertain more clearly the conditioning factors behind the selection of forms. The two primary activities that were run were the ``Family Problems Picture Task'' \cite{SanRoque2012a} and the ``Man and Tree Picture Sets'' \cite{Levinson1992}. These activities comprised the majority of the field work undertaken, and were supplemented by a small amount of elicitation of basic language structures and the collection of a wordlist.

\subsection{Family Problems Picture Task}\label{p:Methods:FamilyProblems}
The Family Problems Picture Task was specifically developed with the elicitation of epistemic forms in mind \cite{SanRoque2012a}, and involves four parts. First, two participants are presented with a set of images \cite{Carroll2009} depicting various interactions between family members in a pseudo-random order\footnote{The order is random for the participants, but is given by \citeA{SanRoque2012a} to allow for consistency and easier analysis down the line, so analysts do not need to work out which image is being described.} and are asked to describe them. Next, they are asked to confer and place the images in an order that depicts a story, and finally are asked to tell the story, once in third-person and once in first-person.

Across these parts, a number of different epistemic contexts are created. Table \ref{t:Methods:FamilyProblemsEvidentials} shows these contexts, and the parts in which they are present. Visual evidence can be found in the descriptions and discussion, as participants are seeing the images for the first time, and subsequently discussing their contents. Similarly, inferential evidence could also be found in both, though potentially more prominently in the discussion phase, as participants are piecing together a story from the illustrations, requiring them to draw inferences about the exact events depicted. The description phase also shows equal epistemic authority over the images for both participants, as neither one will have seen the activity before. This equal authority might also be reflected as shared information, new information (i.e. mirative), or non-origo authority. Non-origo authority is distinguished from equal authority in that while they both accurately describe the epistemic context created in the task, equal authority refers to systems encoding that both speech act participants have the same authority, but not the strength of that authority. In contrast, non-origo authority refers to systems encoding specifically the lack of authority over the information at hand of the origo (speaker in declarative, addressee in interrogative), but does not directly reflect the addressee's perspective. Participatory evidence, or egophoric evidence \cite{Gawne2017}, can be marked in the first-person telling of the story, as could, along with the other parts, factual or neutral evidence as per \citeA{Zemp2020}.

\begin{table}\caption{Epistemic contexts covered by each part of the Family Problems Picture Task}\label{t:Methods:FamilyProblemsEvidentials}
       \noindent\adjustbox{center}{\begin{tabular}{r|c|c|c|c}
                                                  & Description & Discussion & Third-person telling & First-person telling \\
                     Visual evidence              & ✔           & ✔          &                      &                      \\
                     Inferential evidence         & ✔           & ✔          &                      &                      \\
                     Non-origo or equal authority & ✔           &            &                      &                      \\
                     Participatory evidence       &             &            &                      & ✔                    \\
                     Factual or neutral evidence  &             &            & ✔                    &
              \end{tabular}}
       
\end{table}


In some cases, namely with the non-origo and equal authority, as well as the participatory and factual evidentials, the conditions for two of the given epistemic bases are met in a single part of the activity. For instance, participatory evidence and factual evidence could theoretically be triggered by the same epistemic context. They are, however, functionally distinct in theoretical terms and as such have been included separately. That being said, without further evidence or usage contexts that are able to distinguish, it would not be directly possible from the data produced by this activity alone to determine if a given form used in the first-person telling is conditioned by direct speaker involvement in the form of participatory evidence, or by a broader higher origo-authority as seen in factual evidentials. The typology of these similar-yet-different epistemic bases is discussed in greater detail in Section \ref{ss:Description:ClassByFunction}.

\subsection{Man and Tree Picture Sets}
The Man and Tree Picture Sets \cite{Levinson1992} are a series of image sets depicting plastic figures in various arrangements. Within the sets, different images show different arrangements of the same objects. Three sets were used in this project, some of which were combinations or subsets of the sets initially given in \citeA{Levinson1992}, and as such are given their own labels in the context of this project. The first set, \textsc{balls}, has four images showing red and yellow balls in various spatial and colour configurations. The second, \textsc{sawdust}, depicts a small pot full of sawdust, a plant, and a basket, with the pot variously covered, uncovered, overflowing, or entirely absent. The last set, \textsc{pigs}, shows a number of men, pigs, and small bushes in various numbers and configurations. The first two sets are substantially smaller than the {pigs}, and were used to teach participants the activity, and as a sort of warm-up activity.

The activity itself was run as a guessing game or director-matcher task, in which one participant, the matcher, has all images in the set laid out in front of them on cards, and the other, the director, has all images in a deck face-down. Between the participants is a partition such that the director cannot see the matcher's face-up array of images. One by one, the director draws a card and describes it to the matcher, who asks questions in turn, until the matcher is able to select which card the director has just drawn. They confirm that the card is correct, then repeat the process until all cards have been drawn.

The activity was originally designed for the elicitation of spatial reference systems, for which it was very effective here (though the analysis stemming from this falls outside the scope of this thesis), but has been repurposed here with a degree of success for the elicitation of epistemics. As with the Family Problems Picture Task, the activity creates a number of different epistemic contexts, though with the simpler task with fewer stages, there are fewer epistemic contrasts developed.

The Man and Tree activity has fewer separate stages than the Family Problems activity, and has been presented in Table \ref{t:Methods:ManTreeEvidentials} divided into the speech acts of each participant. The Director Description refers to the initial description of the image, and subsequent further comments on the image in response to questions from the matcher, which comprise the other speech acts in the activity. Visual evidence can be seen primarily in the initial descriptions of the images from the director. In all cases this task shows unequal epistemic authority (contrasted with the Family Problems activity), in that the director has sole access to the aforementioned visual evidence at all times, and the matcher is either polling that visual evidence in asking questions, or polling a more authoritative evidence in confirming if they have selected the correct image. The useful difference between these two uses of unequal epistemic authority (the director and the matcher) then, is that one is speaker-origo, with the speaker referencing their own awareness, and the other is, being interrogative, likely addressee-origo.

\begin{table}
       \caption{Epistemic contexts covered by different areas of the Man and Tree Picture Task}\label{t:Methods:ManTreeEvidentials}
       \begin{tabular}{r|c|c}
                                          & Director Description & Matcher Questions \\
              Visual Evidence             & ✔                    &                   \\
              Unequal epistemic authority & ✔                    & ✔                 \\
              Shifted Origo in Questions  &                      & ✔
       \end{tabular}
       
\end{table}

The contrast between the equal epistemic authority in the Family Problems activity and the unequal epistemic authority here is a further distinction able to be drawn from these two tasks. In other cases, it was these confirmation questions polling clearly non-shared unequal knowledge that was able to shed light on an epistemic system potentially conditioned by epistemic authority \cite{Bodnaruk2023}. In this case in particular, however, speakers did not verbally confirm their image choices. It is not clear if this is a fault of how the task was explained, if it was a result of the social dynamics between the participants, or just unfortunate chance.

\section{Outcomes}
These activities benefit from a solid foundation of analysis on the language such that individual forms can be better identified and separated after the activities are transcribed and translated \cite{Bodnaruk2023}. The lack of this foundation was a hindrance to the full analysis of the data gained; however, it was still possible to confirm some epistemic distinctions that had been originally attested in the basic elicitation. Transcriptions and translations were produced primarily in the field in \citeA{ELAN} together with local consultants, with some additional material being transcribed over WhatsApp after the conclusion of the field trip. \exref{e:Methods:LhokpuDistinction} shows the main epistemic contrast with a confident analysis from the elicitation activities.

\begin{exe}
       \ex\label{e:Methods:LhokpuDistinction}
       \begin{xlist}
              \ex\label{e:Methods:LhokpuDistinction:mi1}
              \gll nosam rang-ka [ganmo \textbf{mi}] \\
              mind \textsc{pron-gen} [wife \textbf{\textsc{cop.exist}}] \\
              \glt `In his mind, “My wife is there”.' (Family Problems)

              \ex \label{e:Methods:LhokpuDistinction:mi2}
              \gll ka-lok dzeʔ nih-pu \textbf{mi} \\
              \textsc{1.sg-obl} dog two-\textsc{clf.gen} \textbf{\textsc{cop.exist}} \\
              \glt `I have two dogs.' (Elicited)

              \ex \label{e:Methods:LhokpuDistinction:miha1}
              \gll kona i-du meʔ \textbf{mihã} \\
              then \textsc{prox-loc} fire \textbf{\textsc{cop.exist.evd}} \\
              \glt `Then here there is fire' (Man and Tree - Sawdust)

              \ex \label{e:Methods:LhokpuDistinction:miha2}
              \gll kanka it-dra \textbf{mihã} \\
              old.man one-\textsc{clf.anim}  \textbf{\textsc{cop.exist.evd}} \\
              \glt `There is one old man.' (Family Problems)\footnote{Speakers explained the use of the classifiers \textit{-dra} and \textit{-pu} as marking human and general referents, however the `human' classifier \textit{-dra} was occasionally used to refer to animals in the Man and Tree activity, along with the general classifier \textit{-pu}. No further classifiers have been identified yet.}
       \end{xlist}
       (Lhokpu)
\end{exe}

The form \textit{mi} is used as an existential copula in cases referring specifically to personal knowledge or experience, seen in \exref{e:Methods:LhokpuDistinction:mi1}, where it reflects the personal insights of the character, and in \exref{e:Methods:LhokpuDistinction:mi2}, where it reflects the privileged access of the speaker to information about themself. The alternative form \textit{mihã} is used in cases with direct visual evidence, in both \exref{e:Methods:LhokpuDistinction:miha1} and \exref{e:Methods:LhokpuDistinction:miha2} reflecting the speaker seeing parts of the images for the first time. Without further research, however, it is difficult to say exactly which of the contrastive aspects of the two epistemic contexts is responsible for the difference in forms. In any case, it is clear that there is a binary epistemic distinction on Lhokpu existential copulas, and that it is generally speaking a distinction between higher vs lower authority, with higher authority in the data occurring from personal insights or participatory/ego evidence, and lower authority from visual evidence.

Less confident conclusions were also tentatively drawn regarding epistemic marking in verbal morphology. A limited understanding of some areas of the phonology of Lhokpu limited the analysis that was able to be undertaken here. A verb suffix \textit{-ah} occurs throughout the data collected in the elicitation activities, but not in any directly elicited data. Notably, the directly elicited data---sentences that were translated directly from English or Dzongkha into Lhokpu by consultants---are devoid of epistemic context. While, of course, such context can be described or imagined, the challenges in conscious awareness of the conditions on the usage of epistemic forms, as discussed above, mean that these described situations are not reliable indicators of the actual usage of epistemic markers. The suffix \textit{-ah} is potentially equivalent to a form \textit{-a(l)} given by \cite{Grollmann2018}, with the lateral coda (dropped word finally) debuccalised, mirroring a sound difference between the data collected in Jigme village in this project and Grollmann and Gerber's (2018) data, wherein Grollmann and Gerber's possessive pronoun suffix \textit{-ŋa} is attested in Jigme as \textit{-ha}.\footnote{It is also possible from more recent data collection by Gwendolyn Hyslop, Mareike Wulff, and myself that this suffix is actually \textit{-ŋ̊a}, analysed differently due to the lack of in-depth documentation undertaken in both cases.} It is this glottal coda that is particularly challenging in the analysis of the form, as its phonological behaviour is not yet clear. While it is certainly present in many words, and its presence appears to be contrastive, it is not yet clear if another attested verb suffix \textit{-a} is a separate morpheme or an allomorph of \textit{-ah} with the glottal coda deleted. It is also possible that the glottal coda is present but remains undetected in the analysis. It is here that the lacking foundational knowledge of phonology and basic verbal morphology on the language seriously begins to hinder the analysis that can be completed at this stage.

\cite[20-21]{Grollmann2018} describe \textit{-a(l)} as marking something ``not personally experienced by the speaker or as not belonging to the personal knowledge of the speaker'', though do not provide examples. This functional description appears, at least at this stage, to work with the data collected here.

\begin{exe}
\ex \label{e:Methods:LhokpuVerbal}
\begin{xlist}

\ex \label{e:Methods:LhokpuVerbal:coda}
\gll ŋan dokm̥eŋ-su dzoŋ-do-\textbf{ah} \\
person walking.stick-\textsc{com} stand-\textsc{prog-\textbf{evd?}} \\
\glt `The person is standing with the stick.'

\ex \label{e:Methods:LhokpuVerbal:nocoda}
\gll siŋ-hõ hut-a dzoŋ-do-\textbf{a} le~le \\
tree-\textsc{towards} look-? \textsc{asp-prog-\textbf{evd?}} downhill~\textsc{adv} \\
\glt `Looking downhill towards a tree.' \\
(Man and Tree - Pigs)


\end{xlist}

\end{exe}

\begin{exe}
\ex \label{e:Methods:LhokpuVerbal2}
\begin{xlist}

\ex \label{e:Methods:LhokpuVerbal:cold}
\gll ka tɕuŋ̊-do \\
\textsc{1.sg} be.cold-\textsc{prog} \\
\glt `I am cold.'

\ex \label{e:Methods:LhokpuVerbal:cold2s}
\gll * na tɕuŋ̊-do \\
{} \textsc{2.sg} be.cold-\textsc{prog} \\
\glt *`You are cold.'

\ex \label{e:Methods:LhokpuVerbal:coldevid2}
\gll na ném-do-\textbf{ah} \\
\textsc{2.sg} be.cold.to.touch-\textsc{prog-\textbf{evd?}} \\
\glt `You are cold (to the touch).' \\


\ex \label{e:Methods:LhokpuVerbal:coldevid}
\gll tɕo ném-do-\textbf{ah} \\
water be.cold.to.touch-\textsc{prog-\textbf{evd?}} \\
\glt `The water is cold.' \\

\end{xlist}
(Elicitation) \\ 
Lhokpu (Subfamily Unclear: Bhutan)
\end{exe}


\exref{e:Methods:LhokpuVerbal} shows both \textit{-ah} and an occurrence of \textit{-a} that appears particularly likely to be an allomorph of \textit{-ah}, in both cases reflecting new, direct visual evidence, that was not previously part of the speaker's integrated knowledge. Much like with \textit{mi} and \textit{mihã}, it is difficult to say if, as suggested by \citeA{Grollmann2018} for \textit{-a(l)}, the use of \textit{-ah} is conditioned by the speaker's prior knowledge of some state of affairs, or if it is conditioned by the visual evidence the speaker has for their knowledge of that state of affairs. Notably, earlier in \exref{e:Methods:LhokpuVerbal:nocoda}, another instance of \textit{-a} is attested, at first glance here appearing to be a non-final marker connecting the verb \textit{hut} `look' with the finite-marked verb \textit{dzong} `sit', here marking an aspectual distinction. This non-final marker analysis does not seem to work for the emphasised marker, however. This suggests that, in lieu of an analysis that can account for both uses here, there are two functions or meanings for the suffix \textit{-a}, perhaps one of which is an allomorph of \textit{-ah}. \exrefs{e:Methods:LhokpuVerbal:cold}{e:Methods:LhokpuVerbal:coldevid} show a contrast between the use or lack of the \textit{-ah} suffix. It is not used in \exref{e:Methods:LhokpuVerbal:cold}, in which the speaker is referring to their internal experience using the verb \textit{tɕuŋ̊}, which is restricted to this meaning, seen by the rejection of \exref{e:Methods:LhokpuVerbal:cold2s}. In \exref{e:Methods:LhokpuVerbal:coldevid} and \exref{e:Methods:LhokpuVerbal:coldevid2}, however, the speaker is referring to an external observation and does use the (probable) evidential \textit{-ah} with the verb \textit{ném}, which specifically refers to temperature of other objects or individuals assessed through touch.

Between these two domains in which a probable epistemic distinction has been observed, this is the total extent of the current analysis into epistemic marking in Lhokpu, and as such is the total data that can be included. As is the case with published descriptive material, it is difficult to distinguish confidently between a language lacking a certain functional contrast, and such a contrast simply not being described in the current analysis. As such, when reading wider literature, conclusions cannot readily be drawn about systems lacking features. This limitation extends to the data collected for Lhokpu, simply because the analysis is nowhere near complete enough to confidently exclude any features. Instead, it is possible to preliminarily describe the system as occurring across multiple domains of the language, and containing a number of contrastive forms, conditioned by the closeness of the origo to the information. This closeness may depend on evidence source (direct visual vs general knowledge), participation or direct involvement, or some higher level claim of authority, though it is not yet clear if any single of these conditions is the sole relevant condition, or if it is in fact some combination thereof, or perhaps an entirely different condition.


\chapter{Table of Languages}\label{a:TableOfLanguages}
\begin{table}

\end{table}
\begin{longtable}[c]{ r c c }\caption{Languages surveyed in representative sample as discussed in Section \ref{s:Methods:Schema}. Subfamilies with no data available are also listed, marked with †. In cases where no data is available \textit{and} there are multiple languages in the subfamily, no language is given either.}
       \label{t:Appendix:LanguageReferences} \\
       Language           & Subfamily       & Source \\                                
       \hline \hline \endfirsthead 
       \caption{continued...} \\
       Language           & Subfamily       & Source \\                                
       \hline \hline \endhead 
       Tenyidie           & Angami-Pochuri  & \citeA{Kuolie2006}                     \\
       \hline
       Poumai Naga        & Angami-Pochuri  & \citeA{Veikho2021}                     \\
       \hline
       Jejara             & Ao              & \citeA{Barkman2014}                    \\
       \hline
       Mongsen Ao         & Ao              & \citeA{Coupe2007}                      \\
       \hline
       Bai                & Bai             & \citeA{Wiersma1990}                    \\
       \hline
       Mönpa              & Black Mountain  & \citesA{vanDriem1995}{Hyslop2016}      \\
       \hline
       Hakhun Tangsa      & Brahmaputran    & \citeA{Boro2017}                       \\
       \hline
       Garo               & Brahmaputran    & \citeA{Burling2003}                    \\
       \hline
       Atong              & Brahmaputran    & \citeA{Breugel2014}                    \\
       \hline
       Chepang            & Chepangic       & \citeA{Caughley1982}                   \\
       \hline
       Bhujel             & Chepangic       & \citeA{Regmi2007}                      \\
       \hline
       Idu                & Digarish        & \citeA{Blench2019}                     \\
       \hline
       Toto               & Dimalish        & \citeA{Basumatary2016}                 \\
       \hline
       Dhimali            & Dimalish        & \citeA{King2009}                       \\
       \hline
       Dura               & Dura            & \citeA{Schorer2016}                    \\
       \hline
       Kurtöp             & East Bodish     & \citesA{Hyslop2017}{Hyslop2018}        \\
       \hline
       Tawang Monpa       & East Bodish     & \citeA{Tombleson2020}                  \\
       \hline
       Lizu               & Ersuish         & \citeA{Chirkova2008}                   \\
       \hline
       Ersu               & Ersuish         & \citeA{Zhang2013}                      \\
       \hline
       Gongduk            & Gongduk         & NA\textsuperscript{†}                  \\
       \hline
       NA                 & Hrusish         & NA\textsuperscript{†}                  \\
       \hline
       Turung             & Kachinic        & \citeA{Morey2010}                      \\
       \hline
       Kadu               & Kachinic        & \citeA{Sangdong2012}                   \\
       \hline
       Karbi              & Karbi           & \citeA{Konnerth2020}                   \\
       \hline
       Kayah Monu         & Karenic         & \citeA{Aung2013}                       \\
       \hline
       Eastern Kayah      & Karenic         & \citeA{Solnit1986}                     \\
       \hline
       Duhumbi            & Kho-Bwa         & \citeA{Bodt2020}                       \\
       \hline
       Puroik             & Kho-Bwa         & \citeA{Lieberherr2017}                 \\
       \hline
       Chhathare Limbu    & Kiranti         & \citeA{Borchers2008}                   \\
       \hline
       Yakkha             & Kiranti         & \citeA{Schackow2015}                   \\
       \hline
       Daai Chin          & Kukish          & \citeA{SoHartmann2009}                 \\
       \hline
       Hyow               & Kukish          & \citeA{Zakaria2018}                    \\
       \hline
       Lepcha             & Lepcha          & \citeA{Plaisier2007}                   \\
       \hline
       Lhokpu             & Lhokpu          & Own Data                               \\
       \hline
       Nuosu (Sichuan Yi) & Ngwi-Burmese    & \citeA{Gerner2013}                     \\
       \hline
       Burmese            & Ngwi-Burmese    & \citeA{Soe1999}                        \\
       \hline
       Khatso             & Ngwi-Burmese    & \citeA{Donlay2019}                     \\
       \hline
       Magar              & Magaric         & \citeA{GrunowHarsta2008}               \\
       \hline
       Kham               & Magaric         & \citeA{Watters2002}                    \\
       \hline
       Meithei            & Meithei         & \citeA{Chelliah1997}                   \\
       \hline
       NA                 & Midzuish        & NA\textsuperscript{†}                  \\
       \hline
       Hkongso            & Mru             & \citeA{Wright2009}                     \\
       \hline
       Yongning Na        & Naic            & \citeA{Lidz2010}                       \\
       \hline
       Namuzi             & Naic            & \citeA{Pavlik2017}                     \\
       \hline
       Dolakha Newar      & Newaric         & \citeA{Genetti2007}                    \\
       \hline
       Kathmandu Newar    & Newaric         & \citesA{HaleNewar1980}{Hargreaves2017} \\
       \hline
       Trung              & Nungish         & \citeA{Perlin2020}                     \\
       \hline
       Anong              & Nungish         & \citeA{Sun2009}                        \\
       \hline
       Pyu                & Pyu             & NA\textsuperscript{†}                  \\
       \hline
       Munya              & Qiangic         & \citeA{Bai2019}                        \\
       \hline
       Qiang              & Qiangic         & \citeA{LaPolla2003}                    \\
       \hline
       NA                 & Raji-Raute      & NA\textsuperscript{†}                  \\
       \hline
       Eastern Geshiza    & rGyalrongic     & \citeA{Honkasalo2019}                  \\
       \hline
       Khroskyabs         & rGyalrongic     & \citesA{TaylorAdams2020}{Lai2017}      \\
       \hline
       Milang             & Siangic         & \citeA{Modi2017}                       \\
       \hline
       Southern Min       & Sinitic         & \citeA{Chen2020}                       \\
       \hline
       Tamang             & Tamangic        & \citeA{OwenSmith2014}                  \\
       \hline
       Western Tamang     & Tamangic        & \citeA{Regmi2018}                      \\
       \hline
       Maring             & Tangkhul        & \citeA{Kanshouwa2016}                  \\
       \hline
       Galo               & Tani            & \citeA{Post2007}                       \\
       \hline
       Tangam             & Tani            & \citeA{Post2017a}                      \\
       \hline
       Hile Sherpa        & Tibetic         & \citeA{Graves2007}                     \\
       \hline
       Amdo Tibetan       & Tibetic         & \citeA{Tribur2019}                     \\
       \hline
       Tshangla           & Tshangla        & \citesA{Andvik2010}{Grollmann2020}     \\
       \hline
       Tujia              & Tujia           & \citeA{Brassett2006}                   \\
       \hline
       Chhitkul-Rakchham  & West Himalayish & \citeA{Martinez2021}                   \\
       \hline
       Bunan              & West Himalayish & \citeA{Widmer2014}                     \\
       \hline
       Zeme               & Zeme            & \citeA{Chanu2017}                      \\
       \hline
       Inpui              & Zeme            & \citeA{Devi2014}                       \\
       \hline
\end{longtable}

\chapter{Static Database}\label{a:StaticDatabase}
Explain table and abbreviation key here
This sections presents a static and limited version of the database introduced in \cref{c:Methods}. It is included here for the sake of completeness in the submission of this thesis. A full and more accessible version of the database (and one that will continue to be updated) is available at \url{https://cbodnaruk.com/database/}.

Due to the size of the database and challenges in formatting it to fit in print, a number of column headings have been shortened, with a legend provided in Table \ref{t:Appendix:DatabaseLegend}
\begin{table}
       \centering
       \caption{Legend for the abbreviated headers in the printed static database given in Table \ref{t:Appendix:StaticDatabase}}\label{t:Appendix:DatabaseLegend}
       \begin{tabular}{r|l}
              Abbreviation & Meaning \\
              \hline
              A & Language Metadata \\
              \hline 
              B & Evidence of Epistemic Marking \\
              \hline 
              C1 & Verb Morphology \\
              C2 & Noun Phrase \\
              C3 & Verb Phrase \\
              C4 & Discourse Particle/Speech Act Scope \\
              \hline 
              D & Functions of Marking: Canonical \\
              EM & Epistemic Modality \\
              Ev & Evidentiality \\
              Ego & Egophoricity \\
              Eng & Engagement \\
              Mir & Mirative \\
              Oth & Other \\
              \hline 
              E & Functions of Marking: Non-Canonical \\
              \hline 
              F & Term(s) Used in Source \\
              \hline 
              G & Addressee Perspective \\
              G1 & Interrogative Structures \\
              G2 & Declarative Structures \\
              \hline 
              H & Other Features \\
              H1 & Diachronic Source (if given) \\
              H2 & Obligatory (if given) \\
              H3 & Mixed Paradigm \\
              H4 & Nominal Engagement \\ \hline
       \end{tabular}
\end{table} 
% Please add the following required packages to your document preamble:
% \usepackage{multirow}
% \usepackage[normalem]{ulem}
% \useunder{\uline}{\ul}{}
\begin{landscape}
       \begin{tiny}
       \begin{longtable}{p{1.2cm}p{1cm}|p{0.2cm}|p{0.2cm}p{0.2cm}p{0.2cm}p{0.2cm}|p{0.2cm}p{0.2cm}p{0.2cm}p{0.2cm}p{0.2cm}p{0.2cm}|p{0.2cm}p{0.2cm}p{0.2cm}p{0.2cm}p{0.2cm}|>{\raggedright\arraybackslash}p{2cm}|>{\raggedright\arraybackslash}p{1.5cm}>{\raggedright\arraybackslash}p{1.5cm}|>{\raggedright\arraybackslash}p{1.5cm}p{0.2cm}p{0.2cm}p{0.2cm}}

              \caption{A static version of the database introduced in \cref{c:Methods} and otherwise available at \url{https://cbodnaruk.com/database/}\label{t:Appendix:StaticDatabase}}\\
              \multicolumn{2}{l|}{A: Language Metadata} & B & \multicolumn{4}{l|}{C: Scope and Form} & \multicolumn{6}{l|}{D: Functions: Canonical} & \multicolumn{5}{l|}{E: Functions: Non-Can} & F & \multicolumn{2}{l|}{G: Addr} & \multicolumn{4}{l}{H: Other} \\
Language & Subfamily & & C1 & C2 & C3 & C4 & EM & Ev & Ego & Eng & Mir & Oth & EM & Ev & Ego & Eng & Mir &  & G1 & G2 & H1 & H2 & H3 & H4 \\ \hline \endfirsthead

\caption{continued}\\
Language & Subfamily  & B & C1 & C2 & C3 & C4 & EM & Ev & Ego & Eng & Mir & Oth & EM & Ev & Ego & Eng & Mir & F & G1 & G2 & H1 & H2 & H3 & H4 \\ \hline \endhead

Poumai Naga & Angami-Pochuri & + &  & + &  & + &  & + &  &  &  &  &  &  &  & + & + & evidential {[}277{]}, particle {[}333,336{]} (marking maybe mirative) & ? & + {[}138{]} demonstrative engagement &  &  & + & + \\
Tenyidie & Angami-Pochuri & -? &  &  &  &  &  &  &  &  &  &  &  &  &  &  &  &  &  &  &  &  &  &  \\
Jejara & Ao & + &  &  &  & + &  & + &  &  &  &  & + &  &  &  & + & Reportative, Mirative, Epsitemic Modal {[}144{]} & / & / & / &  &  &  \\
Mongsen Ao & Ao & - &  &  &  &  &  &  &  &  &  &  &  &  &  &  &  &  &  &  &  &  &  &  \\
Bai & Bai & - &  &  &  &  &  &  &  &  &  &  &  &  &  &  &  &  &  &  &  &  &  &  \\
Mönpa & Black Mountain & + & + &  &  &  &  &  &  &  & + &  &  &  &  &  &  & Evidential & / & / & / &  &  &  \\
Atong & Brahmaputran & + &  &  & + &  &  & + &  &  & + &  &  &  &  &  &  & Mirative {[}425{]}, Quotative {[}408{]} & / & + {[}426{]} narrative mirative & quotative from 'say' &  &  &  \\
Garo & Brahmaputran & + & + &  &  &  &  &  &  &  &  &  &  & + &  &  & + & "Terminal Suffix", descriptions but no terms & / & + counterexpective? {[}161{]} & / &  &  &  \\
Hakhun Tangsa & Brahmaputran & + &  &  &  & + &  & + &  &  &  &  &  &  &  &  &  & Reportative, Hearsay {[}352{]} & / & / & hearsay particle from 'to say'{[}352{]} &  &  &  \\
Bhujel & Chepangic & + &  & + & + &  & + &  &  &  &  & + &  & + &  &  & + & Mirative, Evidential & / & / &  &  &  &  \\
Chepang & Chepangic & + &  &  & + &  &  &  &  &  &  & + &  &  & + & + &  & Givenness, Information Flow & + {[}86{]} &  &  &  &  &  \\
Idu & Digarish & + &  &  &  & + &  &  &  &  &  &  & + & + &  &  &  & Evidential & ? & ? & ? &  &  &  \\
Dhimali & Dimalish & + &  &  &  & + &  &  &  &  &  &  & + & + &  &  & + & Deductive, Dubiative, Mirative & / & / &  &  &  &  \\
Toto & Dimalish & + & + &  &  &  & + &  &  &  &  &  &  &  &  &  &  & single Probability morpheme & ? & ? &  &  &  &  \\
Dura & Dura & + &  &  & + &  &  &  &  &  & + & + & + &  &  &  &  & Mirative, Presumptive (EM) & / & / & Mirative descended from nominalisation {[}233{]} &  &  &  \\
Kurtöp & East Bodish & + & + &  & + &  &  &  &  &  &  & + &  & + & + & + & + & Egophoricity and Mirativity, inference, dubiative and others {[}2018:115,134{]} & + & +? {[}2018: 125{]} & See Hyslop 2020 &  & + &  \\
Tawang Monpa & East Bodish & + &  &  & + &  &  &  &  &  & + & + &  & + & + &  &  & Perosnal,Testimonial,Neutral & + & / & / &  & + &  \\
Ersu & Ersuish & + &  &  &  & + &  & + &  &  &  & + &  &  &  & + &  & Direct, Inferred, Reportative & + & +? {[}2013:547 counterexpective q tag{]} &  &  &  &  \\
Lizu & Ersuish & + & + &  &  &  &  & + &  &  &  & + & + &  & + &  &  & First-hand, inferential, quotative; egophoric & + & / & Some details on reportative from "say", cognate in Ersu & Yes? &  &  \\
Gongduk & Gongduk & NA &  &  &  &  &  &  &  &  &  &  &  &  &  &  &  &  &  &  &  &  &  &  \\
Sajolang & Hrusish & + & + &  &  & + & + &  &  &  &  & + &  & + &  &  & + & Mirative, Evidential, Dubiative, Emphatic &  & + & dir evidential from "do", assumption from "seem" &  &  &  \\
Kadu & Kachinic & + &  &  &  & + & + &  &  &  &  &  &  & + &  &  & + & Hearsay, Mirative & ? & ? Mirative {[}404{]} &  &  &  &  \\
Turung & Kachinic & +? &  &  &  & + &  &  &  &  &  &  & + & + &  & + &  & Reportative Evidential {[}458{]}, epistemic meanings of imperatives {[}455-6{]} & / & / &  & No &  &  \\
Karbi & Karbi & + &  &  &  & + & + & + &  &  &  &  &  &  &  &  &  & Quotative, Reportative, Dubiative &  & + &  &  &  &  \\
Eastern Kayah & Karenic & + &  &  & + &  &  &  &  &  &  &  &  & + &  &  &  & quotative &  & / & verb to say &  &  &  \\
Kayah Monu & Karenic & - &  &  &  &  &  &  &  &  &  &  &  &  &  &  &  &  &  &  &  &  &  &  \\
Duhumbi & Kho-Bwa & + & + & +? &  &  &  & + &  &  &  & + &  &  &  &  & + & Evidentiality & /? & + {[}408,413{]} & Various are given in ch.8 &  &  &  \\
Puroik & Kho-Bwa & + &  &  & + &  &  & + &  &  &  &  &  &  &  &  &  & Quotative &  &  & Grammaticalised from 'say' &  &  &  \\
Chhathare Limbu & Kiranti & + &  &  &  & + &  &  &  &  &  & + &  & + &  &  & + & Descriptions given with no terms {[}2007{]} & / & / & As with Yakkha, borrowed from Nepali &  &  &  \\
Yakkha & Kiranti & + &  &  & + & + &  &  &  &  & + & + & + & + &  &  &  & Mirative, reportative, probability {[}516{]} &  & + & Mirative borrowed from Nepali &  &  &  \\
Sunwar & Kiranti & -? &  &  &  &  &  &  &  &  &  &  &  &  &  &  &  &  &  &  &  & Yes &  &  \\
Daai Chin & Kukish & + & + &  &  & + &  & + &  &  & + &  &  &  &  &  &  & Mirative, Evidential & ? & /? &  &  &  &  \\
Hyow & Kukish & + &  &  &  & + &  &  &  &  &  & + &  & + &  &  & + & Surprise, Unpredicted & ? & + {[}437{]} surprising marker marks general surprise &  &  &  &  \\
Lepcha & Lepcha & + &  &  & + &  &  &  &  &  &  & + &  &  &  &  & + & Dubiative, Possibility, inferential, certainty, discovery particles & / & + {[}136,165{]} & Predominantly forms of extant full verbs &  &  &  \\
Lhokpu & Lhokpu & + & + &  & + &  &  & + &  &  &  &  &  &  &  &  &  &  & + &  &  & Yes &  &  \\
Burmese & Lolo-Burmese & -? &  &  &  &  &  &  &  &  &  &  &  &  &  &  &  &  &  &  &  & Yes &  &  \\
Nuosu (Sichuan Yi) & Lolo-Burmese & + &  &  &  & + &  &  &  &  &  & + &  & + &  &  &  & Quotative only {[}377{]} & / & / & Grammaticalised from 'say' in proto-Yi & No? &  &  \\
Kham & Magaric & + &  &  & + & + & + &  &  &  & + & + &  & + &  &  &  & Mirative, Counterexpective, Reportative, Possibility, Probability & + two forms of questions based on expected response & + {[}291{]} mirative requires fact of discovery to be relevant to listener & Possibly grammaticalised from 'say' &  &  &  \\
Magar & Magaric & + & + &  &  & + & + &  &  &  &  & + &  & + &  &  & + & Mirative, Direct/Factual. Reportative, Inferred & + & +? {[}486-487, c reads a bit like IS{]} &  &  & + &  \\
Meithei & Meithei & + & + &  & + &  &  & + &  &  &  & + &  &  &  & + &  & Numerous; indirect evidence {[}211{]}, attitude enclitics {[}250{]}, more on {[}295{]} & + {[}295{]} & + & / & No &  &  \\
 & Midzuish & NA &  &  &  &  &  &  &  &  &  &  &  &  &  &  &  &  &  &  &  &  &  &  \\
Hkongso & Mru & + &  &  &  & + &  &  &  &  &  & + & + &  &  &  & + & Mirative, Dubiative, Speaker Oriented Modalities & / & / &  & No &  &  \\
Namuzi & Naic & +? &  &  &  & + &  &  &  &  &  &  & + &  &  &  & + & "surprise or assurance" {[}150{]} & ? & ? &  &  &  &  \\
Yongning Na & Naic & + &  &  & + &  &  & + &  &  &  &  & + & + &  &  &  & Common Sense, Evidentiality {[}476{]} & ? & ? &  & No & + &  \\
Dolakha Newar & Newaric & - &  &  &  &  &  &  &  &  &  &  &  &  &  &  &  &  &  &  &  &  &  &  \\
Kathmandu Newar & Newaric & + & + &  &  & + &  &  & + &  &  & + &  & + &  &  &  & Conjunct/Disjunct {[}Hargrave: 458{]}, Quotative {[}464{]} and some other clause particles {[}465{]} & + & / & / &  &  &  \\
Khatso & Ngwi-Burmese & + &  &  &  & + &  &  &  &  &  & + & + &  &  & + &  & Strong Assertion Marker, Epistemic Emphatic Particle &  & + {[}440-441{]} &  & No &  &  \\
Anong & Nungish & - &  &  &  &  &  &  &  &  &  &  &  &  &  &  &  &  &  &  &  &  &  &  \\
Trung & Nungish & + &  &  & + & + &  &  &  &  &  & + &  & + &  &  & + & Mirative, Evidential & /? & /? & Some from verbs &  & + &  \\
Pyu & Pyu & NA &  &  &  &  &  &  &  &  &  &  &  &  &  &  &  &  &  &  &  &  &  &  \\
Munya & Qiangic & + &  &  & + & + & + & + & + &  &  &  &  & + &  & + & + & Egophoric, Mirative, Evidential & / & / & Functionally very similar to Lhasa Tibetan &  & + &  \\
Qiang & Qiangic & + & + &  &  &  &  & + &  &  & + &  & + &  &  &  &  & Direct, Inferential/Mirative, Hearsay & + {[}207{]} & / & hearsay particle from 'to say'{[}204{]} &  & + &  \\
 & Raji-Raute & NA &  &  &  &  &  &  &  &  &  &  &  &  &  &  &  &  &  &  &  &  &  &  \\
Eastern Geshiza & rGyalrongic & + & + &  &  & + & + & + &  &  &  & + &  & + & + & + &  & Evidential, Mirative, Non-Shared Information, more terms therein {[}582{]}, Epistemic Certainty {[}575{]} & + & + & Some, various sources given for each form in chapter 9 &  & + &  \\
Khroskyabs & rGyalrongic & + &  &  &  &  &  &  &  &  &  &  &  & + &  &  & + &  &  &  &  &  & + &  \\
Milang & Siangic & + &  &  & + &  &  &  & + &  &  &  & + & + &  &  &  & Egophoric {[}455{]}, reportative, uncertain & + & ? & Lexical verbs {[}460{]} & Yes &  &  \\
Southern Min & Sinitic & -? &  &  &  &  &  &  &  &  &  &  & + &  &  &  &  & Adverbs/Modal Verbs & / & / &  &  &  &  \\
Tamang & Tamangic & + & + &  &  & + &  & + &  &  & + &  & + &  &  &  &  & Mirative, Evidentiality & + & ? &  & Yes & + &  \\
Western Tamang & Tamangic & + & + &  &  &  &  &  &  &  & + &  &  &  &  &  &  & Mirative & / & / &  &  &  &  \\
Maring & Tangkhul & + & + &  &  &  &  & + &  &  &  &  & + &  &  &  & + & Evidential, Mirative,   Direct &  &  &  &  &  &  \\
Galo & Tani & + & + &  & + &  &  &  & + &  &  & + &  &  &  &  &  & Conjunct/Disjunct {[}607{]}, Various Clause Final Particles {[}625{]} & + {[}607{]} & +? {[}634,639{]} & Some only (Uncertainty from irrealis marker) &  &  &  \\
Tangam & Tani & + &  &  &  & + &  &  &  &  &  & + & + & + &  &  & + & Hearsay, Counterexpective & ? & +? {[}73{]} counterexpective &  & No & + &  \\
Amdo Tibetan & Tibetic & + & + &  & + &  &  & + & + &  &  & + &  & + & + &  & + & Egophoricity & + {[}453{]} & ? &  &  &  &  \\
Hile Sherpa & Tibetic & + & + &  & + &  &  & +? & + &  &  & + &  & + & + &  & + & Mirative, Past-Observational, Observational, Disjunct,   Volitional & + {[}222{]} & / & / &  &  &  \\
Tshangla & Tshangla & + & + &  & + &  &  &  &  &  &  &  & +? & + & + &  & + & Subjective (Bjokapakha) {[}232{]}, Mirative (Standard) {[}227{]}, hearsay & + Grollman grammar {[}232{]} & / & Grollman 2020b & Yes & + &  \\
Tujia & Tujia & - &  &  &  &  &  &  &  &  &  &  &  &  &  &  &  &  &  &  &  &  &  &  \\
Bunan & West Himalayish & + & + &  &  &  &  & + & + &  &  &  &  &  &  &  &  &  &  & + {[}2014:503{]} conjuct portrays addressee as volitional & Ego/non-ego in present, tibetic evidentials in past & No & + &  \\
Chhitkul-Rakchham & West Himalayish & + &  &  & + &  & + & + &  &  &  &  &  &  &  &  & + & Perceptual, Dubiative, Assertive, Pers Experience, Pers Assertive & + {[}172{]} & + &  &  & + &  \\
Inpui & Zeme & - &  &  &  &  &  &  &  &  &  &  &  &  &  &  &  &  &  &  &  &  &  &  \\
Zeme & Zeme & - &  &  &  &  &  &  &  &  &  &  &  &  &  &  &  &  &  &  &  &  &  & 
       \end{longtable}
\end{tiny}
\end{landscape}