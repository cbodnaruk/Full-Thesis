\chapter{Discussion}\label{c:Discussion}
\section{Introduction}\label{s:Discussion:Introduction}
Having presented an overview of the trends seen in epistemic marking in \lfam\ languages in Chapter \ref{c:Description}, this chapter takes these typological observations and attempts to draw some conclusions on more theoretical terms about the nature of epistemic marking in the family. It does with with the specific goal of supporting the core argument of this thesis, that epistemic marking exists as single theoretical category with a coherent functional domain, and that this singular theoretical category can be supported by linguistic evidence from natural languages. Further to this, it suggests that functions within this combined category additionally share a common functional motivation for their development and sustained use. This will be argued in terms of two features of some epistemic systems in the survey: mixed systems, in which a functions from across multiple traditional cross-linguistic categories are marekd within a single system, and cases where the use of epistemic marking is explicitly conditioned by social factors.

Mixed systems, identified in \ref{sss:Description:MixedSystems} as a subset of complex systems in which the multiple epistemic bases do not all belong to the same traditional cross-linguistic functional category, are discussed in Section \ref{s:Discussion:Mixed}, with a detailed overview of the theoretical foundations of the analysis presented in Section \ref{ss:Discussion:MixedFoundation}. The implications of these mixed systems on paradigmatic systems and scattered systems are discussed in Sections \ref{sss:Discussion:Paradigmatic} and \ref{sss:Discussion:Scattered} respectively, with a more detailed definition of the term paradigmatic given in the former. Finally, the mixed systems identified in Kurtöp (East Bodish: Bhutan) and Eastern Geshiza (rGyalrongic: PRC) are presented as detailed case studies on how the theoretical conclusions drawn are reflected in actual attested data.

Section \ref{s:Discussion:Social} provides a similar discussion of the extension of epistemic meaning to the reflection of social structures and interpersonal relationships, rather than the relationship between the origo and a piece of knowledge. A number of challenges in the analysis of social conditions are presented in Section \ref{ss:Discussion:SocialChallenges}, followed by case studies of actual data from three languages or language groups in which social factors have been described as influencing the use of epistemic marking: a number of varieties of Amdo Tibetan (Tibetic: PRC), and Ladakhi (Tibetic: India). Additionally, data from the highly divergent Milang (Siangic: India) are discussed in terms of the inverse, that is the effects of epistemic marking on social expectations and norms.

Next, Section \ref{s:Discussion:Perspective} considers frameworks and tools for the analysis of perspective and intersubjectivity in terms of the data collected on \lfam\ languages. Specifically, it considers the theoretical tool of the origo and its usefulness in the various situations identified in the data and discussed in the case studies in Sections \ref{ss:Discussion:MixedCases} and \ref{ss:Discussion:SocialCases}, concluding that it is not always a useful theoretical tool, though does provide a useful conceptualisation for the phenomenon of the conversational presumption and the common shift in perspective and epistemic authority from speaker to addressee in questions. It then proposes a two-tiered approach to the discussion of perspective, suggesting that there is a noteworthy difference between the perspectives considered by the speaker in the construction of a speech act and the perspectives referenced in the meaning of the form the speaker ultimately selects.

Finally, Section \ref{s:Discussion:Motivations} preonsents the argument that the various functions of epistemic marking can further be united by a shared functional motivation, that of establishing a shared epistemic ground for communicati to take place more efficiently. This is to say that there is a shared higher level function across epistemic marking, coordinating the relationship of the speaker and addressee to the information being presented and each other, and to establish a shared awareness of these relationships such that possible ambiguity or disruptions to communication in the form of repair sequences can be avoided. While this is argued in theoretical terms, it is noted that there is a general lack of discource data, such as dialogs, used in the literature to exemplify epistemic forms, and as such there is substantial scope surrounding this conclusion for further research, in particular in descriptive or experimental work.

In the remainder of this section, \ref{ss:Discussion:SpeechActs} provides a brief foundation into the description of speech acts and the underlying functional content of forms as they are used in discussion in this chapter, specifically presenting two descriptive construals of the process behind the selection of forms. These two construals are not different in any real terms beyond their conceptualisation of the process, allowing for better explanation throughout this chapter.

\subsection{Theoretical Approaches to Speech Act Construction}\label{ss:Discussion:SpeechActs}
Much of the following sections discuss the internal process of constructing a speech act in a theoretical sense. I do not intend here to make any claims regarding psycholinguistics or neurology in terms of the actual mechanisms by which language is developed and constructed in the brain, but rather to speak theoretically about the decision-making process in choosing one specific form over another. This process is clearly not a conscious one, but this is particularly the case with epistemic distinctions. \citeA{Grzech2020} notes that the exact rules around the usage of epistemic forms in language are not reliably consciously available to speakers, as was discussed in Section \ref{s:Methods:FieldMethods} with regards to the field work undertaken on Lhokpu as part of this project. Whether conscious or not, however, there is necessarily still some process by which forms are chosen as language is constructed.

There are two general conceptualisations of this process that will be used in the following analysis. The first is a sort of bottom-up approach in which a form is selected for its given meaning. That is, a speaker is attempting to communicate some meaning \textit{xyz}, and as such selects forms with meaning \textit{x, y} and \textit{z}. This of course, is greatly simplified, not accounting for, for instance, agreement, in which multiple forms in a given speech act carry the same meaning. The essence of the conceptualisation is, in any case, that the speaker, in order to communicate a given meaning, will select forms that with the necessary component meanings for successful communication. The second conceptualisation, on the other hand, views the meanings of individual forms as being comprised of a series of conditions that need to be met in order for that form to be used, and that in selecting a given form the speaker considers these conditions against the conditions of the speech act - being both the propositional content of the speech act and its deictic context - and subsequently selects the forms whose conditions have been met. Here, if a speaker is trying to communicate \textit{abc}, they might, for example, consider a set or paradigm or forms with the conditions of \textit{c, d, e,} and \textit{f}. In choosing the form \textit{c}, the speaker is considering each form against their intended \textit{abc} to see which best fits, and as such, is also considering whatever conditions \textit{d, e,} and \textit{f} represent along with the actually applicable \textit{c}. This is particularly abstract without a concrete example, though forms a basis for the case studies presented in Section \ref{ss:Discussion:MixedCases}, where it is exemplified more clearly. The primary difference between these conceptualisations is in how much the speaker is explicitly seen to consider when selecting a form. That is, in the first conceptualisation, the speaker is only considering the meaning they wish to communicate, whereas in the second, the speaker is also necessarily considering conditions which are not necessarily relevant here. This wider consideration implied in the second conceptualisation serves the purpose of selecting a given form in opposition to similar ones, such as contrasting forms in a single paradigm. Additionally, it is easier to ascribe multiple conditions to an epistemic form where only one would be considered the primary meaning. This process would still be theoretically present in any case regardless of the conceptualisation through which the selection of forms is being described. In essence, the two methods used in the following analyses of describing forms, their meanings, and the processes around how they are selected, are not intended to be seen as two different processes, but are rather just two ways of describing or conceptualising the process however it actually works at a cognitive level.

\section{Mixed Systems}\label{s:Discussion:Mixed}
\subsection{Theoretical Foundations}\label{ss:Discussion:MixedFoundation}
Existing typological literature on types of epistemic marking, that is, on epistemic modality, evidentiality, egophoricity, mirativity, and engagement individually, tend to treat these topics in a siloed manner. That is, typologies of evidentiality such as \citeA{Aikhenvald2004} handle only evidential meanings, typologies of engagement such as \citesA{EvansBergqvistSanRoque2018a}{EvansBergqvistSanRoque2018b} handle only engagement structures, and so on. There is an exception to this in some literature on mirativity and egophoricity, which has been discussed by \citesA{Hill2012}{Hill2020} in terms of the relation between these theoretical categories and evidentiality. Beyond this, however, existing research is well suited for the description of systems which fit more neatly into a single one of these categories. As was presented in the typological observations described in Section \ref{sss:Description:MixedSystems}, there are a number of languages attested with epistemic-marking systems that mark meanings beyond the boundaries of a single of these categories. These are being labelled \textsc{mixed systems}. These are systems in which, for instance, meanings such as direct and indirect evidence are encoded along with engagement-like meanings such as non-shared information, mirative, epistemic modal meanings such as dubiative, or egophoricity. Mixed systems are more immediately obvious in what are being labelled here as paradigmatic systems, introduced in Section \ref{sss:Description:GroupSize} and discussed in detail below in Section \ref{sss:Discussion:Paradigmatic}, given forms are more clearly in functional opposition in that they occupy the same grammatical slot. The alternative to paradigmatic systems, scattered systems, can also be mixed systems however, as despite being more formally disparate, the system overall can still be viewed as a single analytical unit. This will be further discussed below in Section \ref{sss:Discussion:Scattered}. As was mentioned above, there have been some attempts to consolidate these systems in parts. Specifically, \citeA{Hill2012} argues for an analysis of mirative-seeming forms as visual evidentials, an argument which, while well founded in some cases, is not able to account for some mirative-marking forms in languages such as Kurtöp, presented in greater detail in the case studies in Section \ref{ss:Discussion:MixedCases}. \citeA{Hill2020} also argues for a view of egophoric marking as an evidential base rather than a category in its own right, pointing, among other examples, to the three-way distinction in Lhasa Tibetan between indirect, direct, and an egophoric base. Here, rather than treating two of these forms (all three of which are clearly functionally contrastive) as evidential and the third as separate and egophoric, Hill suggests that egophoricity is better viewed as an evidential base in which the self is the source of evidence. This aligns with research on language in Papua New Guinea, in which ``participatory'' evidence, or evidence gained through direct participation, is given as an evidential base \cite{SanRoque2012}. As with Hill's criticism of mirativity, however, it is not clear that this explanation works perfectly with all contrasts described as egophoric. Namely, this analysis is challenged by systems where personal authority is the key factor conditioning the use of the egophoric form as opposed to personal involvement. With this in mind, what is interesting about these mixed systems, or more specifically, what points do they make about the analysis of epistemic marking in \lfam\ languages?

Firstly, it is clear that there is a shortfall in the analytical capabilities of the existing literature when dealing with systems such as these. Larger scale typologies have not been developed to account for systems with such functional breadth, even though links between the more limited functional scopes of the traditional categories have previously been drawn, discussed in Section \todo{reference chapter 1}. In order to compare these systems, a broader typology is necessary to account for and potentially allow for a higher level of unity in the varied descriptions of these systems, a number of which are presented as case studies in Section \ref{ss:Discussion:MixedCases}. Further to this, the existence of these mixed systems has some pragmatic implications further justifying the use of the broader epistemic category of a valid coherent functional domain.

Two construals of the internal processes behind the construction of speech acts and the selection of specific morphemes at the exclusion of others is discussed in Section \ref{ss:Discussion:SpeechActs}. It is argued there that in any speech act, a given form is selected for use at the exclusion of any specifically contrasting forms. That is for example, in choosing to use a direct evidential form in a speech act, the speaker is doing so at the exclusion of an indirect one (assuming both exist in the language). In selecting one form at the exclusion of any functionally contrastive forms (or, in single term systems, at the exclusion of an unmarked speech act), a speaker must also be considering these contrastive forms and their appropriateness in relation to the speech act at hand. As a result of this, mixed systems must be viewed as a sum of their parts, rather than just, for instance, an evidential and engagement-marking system in close proximity. Mixed systems are, as discussed above, systems in which there are functional contrasts within the system that do not fit into a single traditional cross-linguistic category. With the idea that a speaker will consider the necessary conditions of each form within a system when choosing which one to use at the exclusion of the others, it can be concluded that in these mixed systems, regardless of which form a speaker ultimately chooses and the traditional cross-linguistic category to which that form belongs, they have internally (and subconsciously) considered the conditions of all applicable contrastive forms in the system and subsequently all applicable epistemic bases that could possibly be marked. This is to say that, in languages with these mixed systems, the speaker is never only considering, for instance, the source of their knowledge, or their confidence in said knowledge. Rather, they are necessarily always considering any epistemic base that could be marked within a system. These systems cannot be seen as limited to one category, or analysed in terms of single categories discretely, as such anaylses would not capture the actual internal processes of the speaker.

Notably, with the inclusion of engagement marking, as is seen in the case studies below, one of these conditioning factors being considered by speakers in languages with engagement-like contrasts is a projection of the perspective of the addressee. On the assumption that a speaker will consider all possible conditioning factors in all cases as argued above, speakers are also considering not only their own perspective, but also that of their addressee in any situation where epistemic marking is applicable. The domains of a languages grammar where said marking would be applicable vary from language to language. The presence of these contrasts reflecting the perspective of the addressee (or at least, the speaker's projection thereof) also appears to vary from language to language, but is argued in Sections \ref{s:Discussion:Perspective} and \ref{s:Discussion:Social} that these contrasts are more prevalent than might immediately appear to be the case.

The following sections expand on these theoretical foundations of mixed systems with specific reference to paradigmatic systems, providing a more rigorous discussion and definition of \textsc{paradigm} and \textsc{paradigmatic} than was given in Section \ref{sss:Description:GroupSize}, and scattered systems, discussing how they fit into the mixed system model despite their disparate formal marking.

\subsubsection{Paradigmatic Systems}\label{sss:Discussion:Paradigmatic}
The typological category of paradigmatic systems was introduced in Section \ref{sss:Description:GroupSize} as contrasted with scattered systems. While when contrasted against the alternative type of system the intended definition of the category is fairly clear, a more explicitly stated definition of the term \textsc{paradigm} will be useful in the following discussions. The challenge here is that ultimately the two categories of paradigmatic and scattered systems do not exist in a true binary, but rather can be taken to be more gradual. For example, in Kurtöp (East Bodish: Bhutan) as is discussed in a case study below, the epistemic system exists across multiple domain-restricted paradigms, as well as in a small number of forms that appear to exist outside these paradigms. As such, for the most part, the epistemic system in Kurtöp appears prototypically paradigmatic, but when factoring in these few clitics that can appear across multiple grammatical domains and in combintation with the full epistemic-marking paradigms, it is not totally so.

The term \textsc{paradigm} is widely used throughout linguistics and carries a generally accepted meaning, but specific definitions of such a basic term are harder to come by. Two levels of specificity appear to exist. On the one hand, defintions such as those in \citeA{Aronoff2023} and \citeA{Trask1993} give the paradigm specifically as a set of inflected forms of a single stem. Others more broadly define the term as referring to any set of linguistic forms with a common property, such as, at an extreme, all nouns \cite{Booij2007}. While \citeA{Trask1993} limits his definition of to the use that \citeA{Blevins2016} notes is prevalent in pedagogically inclined literature, he does give a broader definition of ``paradigmatic relation'', defining it as ``Any relation between two or more linguistic elements which are in some sense competing possibilities, in that exactly one of them may be selected to occupy some particular position in a structure.'' \cite[197]{Trask1993}. Ultimately this concept dates back to Saussure's contrast between syntagmatic and associative (or paradigmatic) relations \cite{Saussure2013}.

The definition used here exists between these two to some extent. Generally it refers to the set of possible inflections of a given stem as given in \citeA{Trask1993}, but refers more to the actual inflecting morphology rather than the composed forms. In this sense, it follows the more general Saussurean concept of paradigmatic relations as defined by \citeA{Trask1993} as sets of morphemes with any functional similarities. That is, in describing a set of forms as paradigmatic rather than scattered, they are being described as contributing to the set of inflected forms available for a given stem, and also as having their usage conditioned by a functionally coherent and similar factors. In practise, this first trait means that the forms in a paradigmatic system will occupy the same grammatical slot, whether that be as affixal morphology, clitics, or particles in a given location within a sentence (often final). This leads to a working definition of a set of forms which can occupy the same grammatical slot (per the pedagogical use of paradigm) and also cover a funtionally coherent set of meanings (per the Saussurean concept of paradigmatic relations). This defintion is given in Figure \ref{f:Discussion:Paradigm}. One challenge with Trasks defintion of paradigmatic relations quoted above is the necessity that these forms be mutually exclusive. In order for forms to occupy the same grammatical slot and in turn produce a neat set of inflected forms, this stipulation is understandable but does not consistently hold across the paradigmatic systems actually seen in the sample. That is, systems that otherwise appear very paradigmatic have been documented to allow cooccurrence of forms. In Eastern Geshiza (rGyalrongic: PRC), which will be discussed as case study in greater detail below, epistemic suffixes that occupy the same grammatical slot can cooccur. In these cases, the origo of the epistemic meaning is traced back along the line of sources. For instance, the cooccurrence of the the sensory and reportative forms marks that the current speaker knows the given information as they heard it from another, who in turn had first hand evidence. Despite this possibility, the system is still paradigmatic in that its forms still both occupy the same grammatical slot on the verb and carry meanings within the coherent domain of epistemic meaning.

\begin{figure}
    \begin{itemize}
        \item[] \textbf{Necessary traits of a grammatical paradigm:}
        \item[+] Set of forms occupying the same grammatical slot
        \item[+] Set of functions falling under a coherent functional domain
        \item[] \textbf{Unnecessary but common traits:}
        \item[?] Forms within set are totally mutually exclusive
    \end{itemize}
    \caption{Working defitinion of \textsc{paradigm}.}\label{f:Discussion:Paradigm}
\end{figure}

There is a risk of a circular definition in terms of the coherent functional domain criterion. The presence of these paradigms is in part being used in this thesis as proof that the meanings across these paradigms do fall into a coherent functional domain, while at the same time the existence of the paradigm is being defined against this coherent functional domain that is itself proven essentially by its presence across a single paradigm. This chapter aims, however, to show that this functional coherence can be seen outside of simply the fact that the forms exist in a paradigm, but rather than the same functional coherence is visible also in scattered systems. It also argues for the more general epistemic supercategory in terms beyond simple morphological structure and form.

There remains finally a question as to how to handle sets of forms which fulfil the first criterion, in that they occur in the same grammatical slot, but do not appear to a coherent functional domain. In Siyewu Khroskyabs (rGyalrongic: PRC), there is a large set of verbal prefixes that appear to occupy the same grammatical slot on the verb but do not reflect a coherent functional domain \cite[34]{TaylorAdams2020}. They are specifically described as not being in paradigmatic opposition, as they can cooccur, though as was discussed above, this is not seen here as an excluding factor. Functionally, these forms include the negative \textit{mə-}, an interrogative \textit{(t)ɕʰə(ɣ)-}, an evidential \textit{ʐə̂-}, among nine others. Here, despite being a set of forms that occupy the same grammatical slot, the lack of any functional coherence across the set precludes them from being considered a single paradigm, a conclusion which seems to logically hold. There is an interesting example Poumai Naga (Angami-Pochuri: India) there is a set of sentence final markers which occur after the verb in a similar formal position \cite{Veikho2021}. Functionally, two of the forms are epistemic and the other three are tense/aspect related. While, in isolation, they appear to occupy the same grammatical slot, an interesting pattern appears when forms are combined. Specifically, any of the three T/A markers can cooccur with either of the epistemic markers, and when they do, the T/A forms need to come before the epistemic ones for the construction to be considered grammatical \cite[278]{Veikho2021}. That is, while either of the functional groups can grammatically occur in isolation, when they are combined it becomes apparent that in fact there are two different grammatical slots being filled, each of which does show a coherent funtional domain and could therefore be considered a paradigm for the purposes of this analysis.

Mixed paradigmatic systems, then, are systems where sets of forms fulfil the criteria given in Figure \ref{f:Discussion:Paradigm}, but more specifically that the coherent functional domains extend beyond a single of the discussed traditional cross-linguistic categories. As mentioned above, the validity of describing these broader sets of functions as following a coherent functional domain is argued throughout the rest of this chapter.

\subsubsection{Scattered Systems}\label{sss:Discussion:Scattered}
In the previous discussion on paradigmatic systems and the defintion of paradigm, situations were discussed where a set of forms occupied a single grammatical slot but did not fall under a coherent functional domain. Scattered systems can be seen as the alternative to this, where forms do not occupy the same grammatical slot but can nonetheless be grouped according to their shared functional domain. It is argued above that the selection of forms within an epistemic system (or any set of functionally contrastive forms) is informed by an assessment of all conditioning factors relevant to the system. Part of the argument for the coherence of these sets of forms, and part of the justification for even considering them a single functional system is their functionally contrastive meanings. This is easy to see in paradigmatic systems, where forms exist contrastively both functionally and formally, in that they occupy the same grammatical slot and as such are more literally formally contrasted, even if strict mutual exclusivity is not being considered a necessary trait of a grammatical paradigm. This poses the question as to whether or not this functional cohesion applies also to scattered systems, where epistemic contrasts are marked with formally disparate strategies. I argue that it does, as the functional content of the various epistemic-marking strategies in a scattered system is not inherently tied to the literal form of the marking. That is to say that, for example, a direct evidential suffix, a reportative evidential enclitic, and a non-shared information sentence-final particle are not influenced functionally, at least not in terms of their epistemic content, by their literal form. It could be argued that the difference in scope of, for instance, a suffix which attaches directly to the verb compared to an enclitic which attaches to a clause level, means that it would exist higher on the theroetical syntax tree that would be drawn of a speech act, creating some functional difference. It is not clear to me that this would have any impact specifically on the epistemic meaning of the forms. As such, there is no reason, especially in cases where the epistemic-marking strategies of scattered epistemic systems do not cooccur, that they should not be considered equally as functionally contrastive and in turn part of the a functionally cohesive domain as paradigmatic systems. A speaker producing an epistemically marked speech act, whether their language has a more paradigmatic or more scattered system, still needs to consider every applicable conditioning factor across the possible epistemic marking for that speech act in order to select the most correct form at the exclusion of others. The necessity of this process is not dependent on the formal similarity of the epistemic marking, the meanings are contrastive and therefore exist within a cohesive system regardless. In any given speech act, the speaker is (though typically not consciously) considering every available communicative tool and selecting the most relevant ones.

There are also languages with scattered systems where the use of epistemic marking is not obligatory. In some cases, unmarked speech acts are assumed to be high epistemic authority on the \textsc{speaker/non-speaker} gradient discussed in Section \ref{sss:Description:SpeakerNonSpeaker}. Here, where some epistemic base is considered the default, the question regarding the difference between a non-obligatory marking and an obligatory marking with a null morpheme constituent again arises. For the sake of some brevity, they will be considered equivalent in this discussion. In Yongning Na (Naic: PRC), \citeA{Lidz2010} reports a four-way distinction between in one domain of the grammar of the language between direct, inferential, reportative, and quotative evidence (C3 per \citeA{Aikhenvald2004}). The direct evidential base, according to Lidz, is unmarked both formally and functionally. Presumably, the description of the direct evidential base as functionally unmarked is suggesting that it is considered the default meaning, and as such is presumed when no other epistemic meaning is marked. This follows the tendency for the unmarked or default epistemic base to be the one closest to the speaker in epistemic terms. In this sense, this epistemic marking in Yongning Na, which can be seen as either non-obligatory or obligatory with a functionally and formally unmarked direct evidential base, fits into the theoretical framework established in this section regarding the construction of speech acts. This is because, in any case where a formally epistemically unmarked speech act still carries a specific epistemic meaning, the decision to not use any epistemic marker is equally as meaningful as the decision to use a specific one.

The alternative to these situations would be ones where a lack of epistemic marking is genuinely epistemically neutral. For this to be possible, the speech acts that would fulfil the conditions of a marked epistemic base must also be grammatical or not unusual to native speakers without said marking. For example, a mirative marker might be seen to add flair to a story but might not be totally compulsory, as noted for Khroskyabs by \citeA[46]{TaylorAdams2020}. In Dhimal (Dhimalish: Nepal), the mirative particle \textit{la} is used in narratives to highlight information for the addressee, though it is not specifically described how obligatory the form is \cite[254]{King2009}. Specific description of non-obligatory forms such as these does not appear to be widespread in the literature, but in theoretical terms this does present a situation where it is harder to argue that all possible epistemic bases are being considered in every speech act, as there is a decision (not marking any epistemic meaning) that truly does not involve reference to any epistemic base by the speaker. \todo{look harder for an example here}

There is a challenge in terminology here, specifically in the use of the term `system'. Thus far, the `epistemic system' or `epistemic-marking system' has been used to refer to the whole set of grammatical epistemic-marking strategies across a language. While this generally involves specific markers, in some languages such as Milang (Siangic: India) these strategies might instead be syntactic \cite{Modi2017}, though not periphrastic or lexical, as this project is specifically limited to grammatical marking. The epistemic-marking system of a language is not necessarily a single entity per se, and there can be multiple subsystems serving different domains of the grammar of a language. This is exemplified below in Kurtöp, which shows a predominantly paradigmatic epistemic-marking system across. The overall system can be divided into a number of subsystems, each serving in this case a different tense/aspect. That is, there is one paradigm of epistemic marking for each of the future, the imperfective, and the perfective domains. As is discussed below, these paradigms are limited to different areas of the grammar and do not interact. They also do not mark the same set of epistemic contrasts, shown in Table \ref{t:Discussion:KurtopComparison}. This is not difficult to describe with paradigmatic systems, as the so-called subsystems exist in discrete paradigms which can be described as such. With more scattered systems, however, this use of terminology is not so easy, as a domain-limited set of epistemic markers such as the various paradigms of Kurtöp would not exist in such an easily described set. This is less an issue of theory than of terminology, but it remains that a term is needed to describe domain-restricted sets of scattered epistemic markers below the level of the system. Other literature has already used the term `subsystem', namely \citeA{Aikhenvald2004}, and as such the term will also be used here.

\subsection{Case Studies}\label{ss:Discussion:MixedCases}
This section presents two\todo{three?} case studies of languages with mixed epistemic systems: Kurtöp (East Bodish: Bhutan) and Eastern Geshiza (rGyalrongic: PRC). These epistemic-marking systems in these languages are described in detail, both in terms of the traditional cross-lingusitic categories and the typological observations presented in Chapter \ref{c:Description}. For each case study language, the challenges that arise when attempting to describe these mixed epistemic systems solely within the framework of a single traditional cross-linguistic cateogory are discussed, as is the usefulness of the more general approach for which this thesis advocates. These languages have been selected for their detailed descriptions of their epistemic marking systems, in particular for the clarity they provide to this discussion of mixed systems. In this sense, they are at the far end of the spectrum in terms of how clear and demonstrable the issues at hand are, though the theoretical arguments made are no less applicable to the wider family.
\subsubsection{Kurtöp}
Kurtöp is an East Bodish language spoken in Central to Eastern Bhutan \cite{Hyslop2017}. The East Bodish family is a relatively small subfamily spoken almost entirely within Bhutan, with some speakers spilling out into the Arunachal Pradesh (specifically Tawang) and Tibet to the North-West. While, as was discussed in detail in Section \ref{ss:Methods:Bayesian}, it is not currently possible to establish a clear phylogeny to account for the development of the subgroups of the \lfam\ family, there appears to be a closer relation between East Bodish and Tibetic languages. \citeA{Hyslop2014a} suggests that, while there is a great deal of shared vocabulary and grammatical forms between individual East Bodish languages and Tibetic languages, much of this similarity does not reconstruct within the East Bodish family and is better attributed to widespread borrowing. Hyslop also notes that, as is common across the family, detailed description is very limited. While there is a substantial amount of material published on Kurtöp by Hyslop, other literature on the family is limited to sketch grammars or small numbers of articles on specific features of languages (see \citesA{Tombleson2020}{Donohue2019} for two examples on Tawang Monpa and Bumthang) as well as Honours\footnote{Australian equivalent of an integrated MA} theses from Hyslop's students \cites{Bosch2016}{Hewitt2020}, on Upper Mangdep and Khengkha, the latter of which is being developed into a Doctoral thesis at current. With this limited description, more confident reconstruction and subsequent clear establishment of the relationship between the East Bodish and Tibetic families is not currently possible. The language family has and continues to be in close contact with Tibetic languages, in particular with Dzongkha as the national language of Bhutan and contact language to the East, as well as historically Kham Tibetic varieties to the North. This close contact and its possible implications will be further discussed in Chapter \ref{c:History}, particularly in terms of the possibility that the epistemic system of Kurtöp discussed here, and others nearby, may have developed under influence from Tibetic languages through this contact.

The epistemic system in Kurtöp is deeply entrenched in a number of areas of the grammar of the language. There are a number of grammatical paradigms, some clearly related and others less so, carrying epistemic marking in different grammatical domains. This paradigmatic structure of multiple paradigms across the grammar is distinct from scattered epistemic marking in that in any individual speech act, the choice of epistemic marking exists within a single paradigm or grammatical domain. This is, for instance, in the perfective aspect which will be specifically discussed here, the epistemic marking system exists within the verbal morphology, though in copulative clauses, a different set of forms exists within the set of copulas in the language. This is contrasted with scattered systems, in which, for example, a single speech act could, all other conditions being equal, see a speaker selecting a verbal suffix, copula, or nominaliser to mark different epistemic meanings. The perfective, copular, and imperfective epistemic paradigms will be discussed here.

The perfective apsect in Kurtöp has five bases, presented in Table \ref{t:Discussion:KurtopPerfective} along with the category or label that best fits the form. Each of these forms and labels will be specifically discussed.
\todo{Should i specifically reference that this table also appears in my book chapter}
\begin{table}
    \caption{Perfective paradigm in Kurtöp (East Bodish: Bhutan, \citeNP{Hyslop2017}) with the closest cross-linguistic category.}
    \label{t:Discussion:KurtopPerfective}
    \begin{tabular}{r l l}

        Form            & Meaning                             & Cross-linguistic Category    \\
        \hline
        \textit{-shang} & Direct evidence non-shared          & Evidentiality and engagement \\
        \textit{-pala}  & Direct evidence                     & Evidentiality                \\
        \textit{-na}    & Mirative                            & Mirativity                   \\
        \textit{-mu}    & Indirect evidence                   & Evidentiality                \\
        \textit{-para}  & Low certainty/Low epistemic support & Epistemic modality           \\
        \hline
    \end{tabular}
\end{table}

These forms are mutually exclusive and obligatory in the perfective aspect, all also carrying the perfective meaning alongside their epistemic one. \cites{Hyslop2014}{Hyslop2017}{Hyslop2018} describes the meanings encoded by the forms in Table \ref{t:Discussion:KurtopPerfective} in opposition to each other using a binary tree, reproduced in Figure \ref{f:Discussion:KurtopPerfective}. Notably here, the forms are not assigned to specific cross-linguistic categories, but rather are defined in terms of the differences between forms and their specific meanings. Hyslop does, however, use specific terminology when describing the forms in greater detail and in glossing. Each form will be discussed specifically below, with a consideration of the terminology used by Hyslop as well as the specific meanings and conditions of use both in Hyslop's analysis and in the examples provided.
The form \textit{-shang} is defined against the functionally similar form \textit{-pala} as marking exclusive knowledge on the part of the speaker. It is not specifically described as marking engagement, given these publications all predate \citesA{EvansBergqvistSanRoque2018a}{EvansBergqvistSanRoque2018b} and the introduction of the term engagement in reference to the more formalised cross-linguistic phenomenon. Rather, Hyslop describes the form as egophoric, referencing the initially observed tendency of the contrast between \textit{-shang} and \textit{-pala} to follow the archetypal egophoric distribution, noting that in naturalistic data recorded in situ this tendency does not reflect the actual use of the forms. This is to say that, typically speaking, the higher speaker-authority form \textit{-shang} occurs more commonly in elicitation in first-person statements and second-person questions, while the lower form \textit{-pala} occurs more commonly in other cases \cite[129]{Hyslop2018}. The challenge with the sole usage of elicitation as a methodology for the analysis of epistemic forms is that the utterances are potentially bleached of context. That is, the lack of real-world context surrounding the elicited speech act means that the selection of obligatory epistemic marking, such as here in Kurtöp, is not naturally possible. This is further confounded by the observation that speakers tend not to have a conscious awareness of the exact meanings or usage conditions of epistemic forms \cite{Grzech2020}, and as such cannot necessarily readily simulate the usage of these epistemic forms in direct elicitation, even if an example context is given. Rather, in usage, the contrast between \textit{-shang} and \textit{-pala} is conditioned by the exclusivity of the knowledge to the epistemic origo. As such, in contextually bare speech acts, the usage of the forms does follow the archetypal egophoric distribution, but in contextually rich usage, this is not necessarily the case. In referencing both the access of the speaker and the addressee to the given piece of information, this contrast could be better described as in fact marking engagement were it to be ascribed a label of this sort.

Hyslop contrasts \textit{-shang} and \textit{-pala}, which mark expected or unsurprising knowledge, against \textit{-na}, the mirative. \citeA{Hill2012} argues against the existence of the mirative as a valid cross-linguistic category, and attempts to explain mirative meanings as alternatively construed direct evidential markers. He argues that the construal of mirative meaning is an artefact of the necessary immediacy of the source of information in speech acts marked with direct or visual evidentials. While he has reasonable success in refuting DeLancey's (1997)\nocite{DeLanceyMirativity1997} analysis of Lhasa Tibetan \textit{'dug} as mirative, the same counterargument does not seem to hold here. This is because the form is specifically contrasted against forms with a direct or visual evidential meaning that do not mark the same mirative meaning. That is, if the form \textit{-na} does not mark the mirative function, marking a given piece of information as new or unexpected, then there is nothing to distinguish it from specifically the form \textit{-pala}, which marks direct visual evidence unmarked of engagement. Interestingly, the mirative \textit{-na} does not appear to be limited to speaker-origo, but rather is also attested in narrative. Here, the necessary prior knowledge by the speaker of the events in the narrative they are telling preclude them from reflecting their own perspective as new or unexpected. As such, the origo cannot logically lie with the speaker in these cases. The presence of mirative marking in narratives is not unique to Kurtöp, also being attested for example in Atong \cite[Brahmaputran: India,][]{Breugel2014}, Munya \cite[Qiangic: PRC,][]{Bai2019}, Poumai Naga \cite[Angami-Pochuri: India,][]{Veikho2021}, Yakkha \cite[Kiranti: Nepal,][]{Schackow2015}, and Lepcha \cite[Internal Isolate: India,][]{Plaisier2007}. In some of these cases, it is clear that the use of the mirative in narrative reflects a character origo - the character in the story is surprised by information newly presented to them within the context of the story. This is a common function of epistemic marking in narratives, also being seen in analyses of evidential and egophoric marking, to be further discussed in Section \ref{ss:Discussion:Origo}. In Kurtöp, as in some other cases, however, this does not appear to be the case. Rather, the mirative marker appears to mark a moment in a narrative as new or unexpected for the audience or the addressee.

The three forms \textit{-shang}, \textit{-pala}, and \textit{-na}, all appear to also carry a direct evidential meaning, though \citesA[113]{Hyslop2014} describes this in terms of the presence of ``personal knowledge''. The direct evidential meaning can more specifically be construed from their contrast with the expressly indirect or inferential marker \textit{-mu}. This form is described briefly as carrying only this evidential meaning, and is presented as doing so in contrast with the three forms discussed above. Finally, the form \textit{-para} marks, in contrast with all other forms, low epistemic support from the perspective of the speaker.

This paradigm, given in full in Table \ref{t:Discussion:KurtopPerfective}, is a good example of the \textsc{speaker/non-speaker} gradient discussed in the typological overview of this project in Section \ref{sss:Description:SpeakerNonSpeaker}. The five forms range from a high level of epistemic authority on the part of the origo in \textit{-shang} and its non-shared information meaning, with each subsequent form in Table \ref{t:Discussion:KurtopPerfective} reflecting a lower claim\footnote{Of course, in cases where the origo is not the speaker, there is no agentive `claim' being made by said origo, but rather the status of the authority is being projected and granted by the speaker. This is further discussed in Section \ref{ss:Discussion:Origo}.} over the epistemic authority by the epistemic origo. The form \textit{-pala} reflects direct, previously assimilated knowledge but without the exclusive non-shared information meaning of \textit{-shang}, \textit{-na} reflects this direct evidence but without the previous assimilation, \textit{-mu} reflects certainty but without direct evidence, while \textit{-para} does not reflect the certainty of the other forms.

These forms show a degree of variation within the paradigm in terms of the reflection of perspectives. All forms in declarative structures reflect some aspect of the perspective of the speaker, including \textit{-shang} which carries a direct evidential meaning. The forms are attested reflect the perspective of the addressee when used in interrogative structures \cite{Hyslop2018}, exhibiting the common shift of origo from speaker to addressee in questions referred to by \citeA{Hill2020} as the conversational presumption. In addition to this, however, there are also forms which necessarily reflect the perspective of both the speaker and listener in declarative structures. Reflection of the perspective of the addressee is typical of engagement marking per \citeA{EvansBergqvistSanRoque2018a}, and can be seen here in all cases in \textit{-shang}, as well as potentially in \textit{-na} in narratives where the mirative meaning is reflective of an expectation that the marked event will be new or unexpected to the addressee. A more detailed discussion of perspective and the origo can be found in Section \ref{s:Discussion:Perspective}, but for the sake of the discussion it is clear that there is substantial variation in the reflection of perspective across the paradigm.

This paradigm exists alongside two other epistemic-marking paradigms, along with a small number of other forms to be discussed below. The first of these two other paradigms is the similarly complex copula system. As is common in many langauges in the \lfam\ family, Kurtöp has two sets of copular morphemes, an existential and an equative set. The existential set are used in sentences marking the existence or presence of some referent, potentially at a given location or in an individual's possession, while the equative set are used to equate two referents, being nouns or adjectives.

In Kurtöp, a small set of base copulas are complemented by a much larger set of complex or composed forms, which can be reconstructed with varying success to the grammaticalisation of a base copula and some other morphology. These base copulas are \textit{wen} and \textit{min} as the equational positive and negative forms and \textit{nak} and \textit{mut} as the existential equivalents \cite[120]{Hyslop2014}. Notably the forms \textit{nak} and \textit{mut}, while reconstructable, are not attested in their base forms. Many of the contrasts in the larger set of complex copulas reflect the contrasts already discussed for the perfective-marking paradigm, including some which appear to be morphologically related. The form \textit{nawala} is used with knowledge that is certain on the part of the speaker but is not new or unexpected, largely equivalent to the perfective \textit{-pala} and seeming to be a grammaticalisation of the base copula \textit{nak} and \textit{-pala}. Similarly, there is a form \textit{nawara}, likely composed of \textit{nak} and the perfective suffix \textit{-para}, carrying the same low epistemic confidence meaning as the perfective suffix. Not all forms have this clear etymology. The mirative copula \textit{nâ} appears to be the descendent of the original base copula, and is in fact the least marked of the copulas. Finally, not all of the existential copulas are so clearly equivalent to bases in the imperfective paradigm. The form \textit{naki} carries a similar meaning of uncertainty to \textit{nawara}, contrasting in evidential terms in that it requires direct visual evidence while \textit{nawara} does not. There is a final form \textit{naksho}/\textit{nakshu} which seems to remain uncertain \cites[122]{Hyslop2014}[311]{Hyslop2017} but may be related to the similarly composed equative copula \textit{yincok} in the geographically neighbouring Tibetic language Chocangaca \cite{Bodnaruk2023a}.

The imperfective paradigm only contrasts two epistemic bases, between information that is or is not unexpected. While the meaning of the mirative form \textit{-ta} is similar to that of the perfective form \textit{-na}, it is used much more widely. This is potentially a logical outcome of the difference in temporality. \citeA[307]{Hyslop2017} notes that the mirative \textit{-ta} is more commonly used with third-person agents. The actions of these third-person agents is inherently less accessible to the speaker than their own actions specifically when speaking in the imperfective aspect - about actions that have not yet taken place, and as such the more common usage of the mirative form in the imperfective is not surprising. Table \ref{t:Discussion:KurtopPerfective} compares the epistemic bases marked by the perfective, imperfective, and both copular paradigms. While these epistemic bases do not align, in all bases there is a gradient that can be analysed from meaning closest to the speaker to furthest from the speaker, and in many cases these bases cross the boundaries of traditional cross-linguistic categories.

\begin{table}
    \caption{A comparison of the epistemic bases and their meanings across four paradigms in Kurtöp drawing data from \citesA{Hyslop2014}{Hyslop2017}.}
    \label{t:Discussion:KurtopComparison}
    \scalebox{0.69}{
        \begin{tabular}{llll|c|c|c|c|}
            \cline{5-8}
                                                                   &                                                       &                                                &                          & Perfective               & Imperfective                    & Existential Copular              & Equative Copular              \\ \hline
            \multicolumn{1}{|l|}{\multirow{4}{*}{High confidence}} & \multicolumn{1}{l|}{\multirow{3}{*}{Direct evidence}} & \multicolumn{1}{l|}{\multirow{2}{*}{Expected}} & Origo exclusive          & \textit{-shang}          & \multirow{2}{*}{\textit{-taki}} & \multirow{2}{*}{\textit{nawala}} & \multirow{2}{*}{\textit{wen}} \\ \cline{4-5}
            \multicolumn{1}{|l|}{}                                 & \multicolumn{1}{l|}{}                                 & \multicolumn{1}{l|}{}                          & Non exclusive            & \textit{-pala}           &                                 &                                  &                               \\ \cline{3-8}
            \multicolumn{1}{|l|}{}                                 & \multicolumn{1}{l|}{}                                 & \multicolumn{2}{l|}{Unexpected/mirative}       & \textit{-na}             & \textit{-ta}             & \textit{nâ}                     & \textit{wenta}                                                   \\ \cline{2-8}
            \multicolumn{1}{|l|}{}                                 & \multicolumn{3}{l|}{Indirect evidence}                & \textit{-mu}                                   & \cellcolor[HTML]{C0C0C0} & \cellcolor[HTML]{C0C0C0} & \cellcolor[HTML]{C0C0C0}                                                                           \\ \hline
            \multicolumn{1}{|l|}{\multirow{2}{*}{Low confidence}}  & \multicolumn{3}{l|}{Direct evidence}                  & \textit{-para}                                 & \cellcolor[HTML]{C0C0C0} & \textit{naki}            & \cellcolor[HTML]{C0C0C0}                                                                           \\ \cline{2-8}
            \multicolumn{1}{|l|}{}                                 & \multicolumn{3}{l|}{Indirect evidence}                & \cellcolor[HTML]{C0C0C0}                       & \cellcolor[HTML]{C0C0C0} & \textit{nawara}          & \textit{wenpara}                                                                                   \\ \hline
            \multicolumn{4}{|l|}{Lower confidence}                 & \cellcolor[HTML]{C0C0C0}                              & \cellcolor[HTML]{C0C0C0}                       & \cellcolor[HTML]{C0C0C0} & \textit{winim}                                                                                                                \\ \hline
        \end{tabular}}
\end{table}

There are also a number of epistemic forms that are not part of these paradigms, such as the reported speech clitic \textit{=ri} and the counter expectation clitic \textit{=sa}. Unlike the other independent paradigms within the epistemic system, these forms do cooccur forms in the other paradigms, such as the perfective markers. An example of this is given in \ref{e:Discussion:KurtopCExp}. Here, the speaker is asking the location of a third party, a question to which he thought he had heard the answer but has, counter to his expectations, realised he does not in fact know the answer. Firstly, the copulative clause asking the individual's location marked with the reportative clitic \textit{=ri}, marking that the speaker expects this information to have been acquired second-hand. Secondly, the the perfective clause is marked with the high speaker authority, non-shared information perfective suffix \textit{-shang}, as the speaker has sole access to his experience, as well as the counter expective clitic \textit{=sa}, as he had expected to have remembered.

\begin{exe}
    \ex
    \glll 'au nawori... ngai koshangsa \\
    'au nak-po=ri ngai ko-shang=sa \\
    where \textsc{cop.exis-qp:perv=rep} \textsc{1.erg} hear-\textsc{perv.ego=cexp} \\
    \glt `Where (did I hear) (he) was? I thought I heard (to self).' \\
    Kurtöp \cite[East Bodish: Bhutan,][126]{Hyslop2014}

\end{exe}

The typologies presented in the currently available literature on topics such as the more traditional categories of evidentiality, epistemic modality, egophoricity, mirativity, and engagement struggle to capture and describe the epistemic-marking systems in Kurtöp. \citeA{Aikhenvald2004} presents one of, if not the, earliest major attempt to develop a typology of evidentiality marking across the world. She presents a categorisation of evidential systems primarily by the number of bases contrasted in the system, and then by the specific functional content of these contrasts. This is not dissimilar to the typology presented in Chapter \ref{c:Description}, grouping epistemic systems both by size and by function, but is more specific in terms of the number of contrasts and only assesses evidential functions. While very functional for describing systems that only mark evidential contrasts, it is readily clear that a system such as that in Kurtöp, or even just the perfective paradigm, cannot be accurately classified within Aikhenvald's framework despite marking some evidential meanings. Kurtöp only contrasts on two evidential bases - direct and indirect, in this case more specifically comprising experiential, visual, and other sensory sources of information contrasted against inference and hearsay. It could perhaps also be argued that the \textit{-shang}/\textit{-pala}, analysed here as falling more under the umbrella of engagement, could represent a separation of the experiential evidential base as argued for egophoric systems by \citeA{Gawne2017}. This can be fairly readily disproven, however, with the attestation of \textit{-shang} with non-first-person statements where the source of evidence cannot be experiential, as in \ref{e:Discussion:KurtopShang}.

\begin{exe}
    \ex
    \gll zhang pep-shang \\
    heaven arrive.\textsc{hon-perv.ego} \\
    \glt `(the lama) passed away (lit. arrived in heaven).' Kurtöp \cite[East Bodish: Bhutan,][114]{Hyslop2014}
\end{exe}

This would classify this paradigm as an A1 evidential system, one that marks only a firsthand vs. non-firsthand distinction. Aikhenvald also does not account for personal experience as a source of evidence, an omission noted previously by \citeA{Hill2020}, separating egophoricity from evidentiality completely.

Similar challenges exist for other categories. Only \textit{-shang} marks any explicit engagement-like meaning, in that it references the mental access of the speaker and lack of access of the addressee to the information at hand, only \textit{-na} marks explicit mirative meaning, and only \textit{-para} marks explicit epistemic modality. It could be analysed that each of these in fact represents a two-way distinction between marked and unmarked, but regardless in no case does an analysis of Kurtöp within any singular framework lead to anything close to a complete picture of the actual system. This is perhaps why Hyslop generally seems to have avoided using such terms in much of her labelling of these forms. While they are still glossed as ``\textsc{ego} or \textsc{mir}'' and so on, \citesA{Hyslop2014}{Hyslop2017}{Hyslop2018} describes the actual contrasts with non-jargon terms such as ``Certainty'', ``Shared Experience'', and ``Personal Knowledge'' \cite[113]{Hyslop2014}.

As discussed in Section \ref{ss:Discussion:MixedFoundation}, there is also a functional reason that these more siloed approaches to analysis fall short. In choosing a given form from, to continue the example, the perfective paradigm, a speaker is not only choosing to use a given form, they are also choosing to not use any other one. That is, in order to select the most correct form given the epistemic context of the speech act at hand, the speaker needs to consider the functions of all forms within the paradigm. Rather than thinking of each form as carrying its meaning and as being independent from all others in a paradigm, it is in this case perhaps better to consider the paradigm as a whole of having a set of conditions to be assessed by the speaker in order to select the correct form, all of which must be considered every time a form is selected.

In viewing this epistemic-marking paradigm as such, a set of conditions can be developed for the paradigm that denote the various assessments of the context of the speech act undertaken by the speaker. For instance here, while only \textit{-shang} explicitly marks information about the speaker's assessment of the addressee's perspective, it can be presumed that at the very least such an assessment is also happening in cases where the alternative \textit{-pala} is used instead, if not in every perfective speech act along with other conditions such as the source of information and the speaker's own confidence and expectations surrounding the information at hand. The implication of this is that rather than a single form being a ``evidential marker'' or ``engagement marker'' independent of other forms, the entire paradigm needs to be seen as being conditioned by all the factors described above, regardless of which form is ultimately used. With this, it is not readily possible to describe the paradigm, and subsequently the entire system as a sum of the aforementioned paradigms and outlying forms, as any single category. Rather, a single term to denote all of these traditional cross-linguistic categories as they are marked together is necessary. Here, the term ``epistemic'' has of course been used to represent grammatical subsystems which mark these sets of meanings, or the usage of which are informed by these conditioning factors.


example of mirative contrasted (though not perfectly) with the visual evidential - non-narrative if possible
not many examples of the -na perfective form but maybe could use the copula example on p 123 of hyslop2018


\subsubsection{Eastern Geshiza}
Eastern Geshiza is a rGyalrongic language spoken in Sichuan Province, PRC \cite{Honkasalo2019}. It is a member of the rGyalrongic subgroup often referred to as the Horpa languages or lects (though \citeA[3]{Honkasalo2019} discusses potential issues with this term), referred to by Glottolog \cite{glottolog} as West rGyalrongic in contrast to Core rGyalrongic, following \cite{Gates2012}. Literature on these languages is fairly limited - \citeA{Gates2021} provides a clear overview of the fairly disparate sketch grammars which have been published in Chinese and English over the last 50 years. In terms of comprehensive descriptions, however, the West rGyalrongic Horpa languages are much better described than the East Bodish languages discussed above, with comprehensive grammars of varieties of Khroskyabs \cite{Lai2017}, Stau \cites{Gates2021}{Tunzhi2019}, and of course Eastern Geshiza \cite{Honkasalo2019} all written as PhD theses, among other non-English or shorter descriptive works. While \citeA{VanDriem2014} places rGyalrongic as a leaf on its own, it has also been positioned as a subfamily within a larger Qiangic group \cite[15]{Honkasalo2019}, though both this positioning and the Qiangic subfamily itself have been questioned for their validity as actual genealogical groups as opposed to potential examples of widespread areal diffusion \cites{Honkasalo2019}{Chirkova2012}.

The epistemic-marking system in Eastern Geshiza predominantly appears as a paradigm of verbal suffixes, presented in Table \ref{t:Discussion:Geshiza}. \citeA[584]{Honkasalo2019} notes that the marking of epistemic meaning is obligatory, though that the ``ego-oriented'' form is unmarked or marked with a null morpheme. This is in line with the wider trend for unmarked forms to represent meaning closest to the speaker. This egophoric base is conditioned by volition on the part of the speaker. Actions directly experienced by the speaker but without their control, such as a dream, are marked instead with the inferential \textit{-sʰi}. This is at odds with similar distinctions in, for instance, Lhasa Tibetan, where non-volitional first-person actions would be marked either with the dedicated speaker-patient form \textit{-byung} or with the direct evidential. The extensions of egophoric meaning and claims of highest epistemic authority by speakers seen in languages such as Amdo Tibetan \cite{Tribur2019} are also attested in Eastern Geshiza, in that the egophoric base can be used with non-first-person agents in cases where the speaker has a deep intimate knowledge about the information at hand. This includes a lasting state of being or habitual action of, for instance, a family member.

The next form moving further from the highest level of epistemic authority is the sensory evidential \textit{-ræ}, which marks information from any sensory source including non-visual sources such as taste and physical sensation. It is potentially cognate with the Mazur Stau sensory evidential \textit{-rə} \cite[347]{Gates2021}. Generic information or general world knowledge are also marked with the sensory evidential rather than the egophoric form. The inferential marker \textit{-sʰi}, beyond its use above in some non-volitional first-person actions, acts more or less as would be expected, marking information gained without direct sensory or participatory evidence. This is extended to sources of information for which the speaker was present but does not mentally have access to, such as their own birth. \cite[589]{Honkasalo2019} suggests a possible diachronic route for the development of this form, though notes some possible challenges with the development of a Proto-\lfam\ \textit{*-s} into \textit{-sʰi} within Eastern Geshiza. Not noted, however, is the formal similarity between the form and functionally similar forms in varieties of the related Khroskyabs language. \citeA{TaylorAdams2020} give a mirative marker \textit{-(t)sʰi} for the variety spoken in Siyewu, while \citeA{Lai2017} gives the form \textit{-si} in the Wobzi variety as carrying both inferential and mirative meaning. This seems to lend some support for areal diffusion of the form given the similarities of these forms and their functions, as well as the geographical proximity of the speaker groups being spoken in neighbouring counties within Sichuan province. Interestingly, despite the documented function of the form in the Wobzi variety of Khroskyabs appearing closer to Eastern Geshiza than the Siyewu variety, geographically speaking Siyewu appears to be between Geshiza and Wobzi, assuming the Dajin River valley is used as a passage through the mountains. Running downstream along the Dajin River instead of upstream, \citeA{Gates2021} describes another formally similar form \textit{-sə}, again an inferential evidential with some mirative uses, and notes a further cognate form in Tangut of \textit{sji²}, also an inferential \cite{Lai2020}. The likely cognacy of these forms, along with the age of Tangut (which is attested only in historical record and was spoken circa 1038-1227 \cite{Lai2020}), suggests that they are a common inheritance rather than an areal feature. \citesA{Gates2021}{Honkasalo2019}{Zhang2022} all note that these forms are polyfunctional, also being used as nominalisers among other things. Regardless of the exact development of these forms, the inferential meaning of the cognate form both in the historical Tangut and in the modern-day languages of Mazur Stau, two varieties of Khroskyabs, and Eastern Geshiza suggests that such meaning could have applied to the ancestral form. The similarity between sensory evidentials \textit{-ræ} in Eastern Geshiza and \textit{-rə} in Mazur Stau is also present possibly in Khroskyabs as the prefix \textit{rə-}, which can have sensory evidential meaning \cite{Lai2017}. If these forms can in fact be reconstructed with epistemic meaning to a shared ancestor, it may be one of, if not the only case where such is possible for epistemic marking.

The next two forms are the reportative and quotative, which respectively mark indirect reported speech (information gained through someone else) and direct reported speech (actual quotation). The reportative \textit{-jə} is a clear and fairly recent grammaticalisation of the verb \textit{jə} `to say'. The verb as a lexical item is still able to be used to mark directly quoted speech, often with the dedicated quotative \textit{-wo}, which is only able to be attached to \textit{jə} and other \textit{verba dicendi}.

\citeA{Honkasalo2019} separates the above for markers, labelled as ``evidentiality'', and the final two, labelled as ``engagement'' in his presentation and description of the system. Honkasalo notes that these final two forms do not fit within the narrow scope of evidential marking in functional terms and as such groups these forms, which occupy the same grammatical slot as the others and share a clear functional domain, as separate. This separation appears to only exist in definitional terms - evidentiality has a limited functional scope, and as such part of the paradigm must be analysed within a different framework. This is the core of the argument being presented here. There is little linguistic reason not to treat the paradigm as it really appears, being a single system marking a broader set of epistemic meanings, with a much broader set of conditioning factors.

The first of these two ``engagement'' markers is the non-shared-information marker \textit{-go}. This form marks information which the speaker knows confidently, but does not expect the addressee to have yet have access to. Functionally, this forms appears similar to the Kurtöp non-shared-information marker \textit{-shang}, but shows a key distributional difference. While Kurtöp \textit{-shang} is used in first-person statements as a default, marking that a speaker tends to have exclusive access to their own internal awareness, Eastern Geshiza \textit{-go} is generally limited to third person statements. Rather, first person statements are unmarked, taking the default ego-oriented epistemic base. It is also rarely used, if ever, with second person subjects in declaratives, as it is difficult for a speaker to have greater knowledge over their addressee's internal state than the addressee themself. This suggests that, while in Kurtöp the non-shared-information marker represents the highest possible claim of authority, the separate ego-oriented form in Eastern Geshiza sits higher in the hierarchy.

The final form described in \citeA{Honkasalo2019} is a form \textit{-mə} of unclear epistemic meaning. It might represent knowledge that is new to the speaker, in contrast with the previously described \textit{-go}, though Honkasalo notes that this does not account for every attested use. Given the brief description of this form and the lack of clarity in the analysis, at least at the time of submission of this doctoral thesis, this form will not be further discussed.

Table \ref{t:Discussion:EasternGeshiza} provides all of these forms in order from closest to furthest from the speaker in terms of epistemic authority. A decision has been made here to place the reportative above the quotative. A core differece between the two in Eastern Geshiza, functionally speaking, is the presence of a specified third-party source of the information at hand. That is, the use of the reportative \textit{-jə} marks simply that the information was at some point said by a third party, while the direct quotations marked by the quotative \textit{-wo} attaches to \textit{verba dicendi} forming a matrix clause over which a specified information source is the subject. While this not necessarily a named or explicitly marked individual, verbs in Eastern Geshiza unmarked for person take third-person meaning, and still reference a specific third-party \cite[592]{Honkasalo2019}. In the larger scheme of this typology of funtional cline from high to low epistemic authority, this distinction is arguably less marked or clear, but the distinction made here to place the reportative as higher than the quotative is based on this reference to a specified individual information source. That is, in the use of the quotative, the speaker is referencing a specific individual as their information source and is subsequently passing off some responsibility over the information to this individual, whereas with the reportative the speaker can be understood to be maintaining responsibility over the information themselves, in turn claiming a higher level of authority. This idea of responsibility is in reference to the Gricean maxim of quality. There is an expectation in cooperative conversation that what a speaker says is, to the best of their knowledge, true. As can be seen most clearly with epistemic modality, more or less the confidence of the epistemic origo in the truthfulness of the information at hand, but also more generally in epistemic marking when considering the common links between low confidence and inferential marking, for instance, a claim over epistemic authority is strongly tempered by confidence. Lower claims of epistemic authority often (though not necessarily, as is discussed in Section \ref{s:Discussion:Social}) reflect a lower level of confidence in the truthfulness of the information at hand. As such, the reference to an explicit information source in the quotative can be seen to temper confidence of the epistemic origo in essentially citing another source. This is compared to the reportative, which while claiming less authority than, for instance, the sensory evidential form \textit{-ræ}, still involves the epistemic origo as the sole specific individual responsible for the truthfulness of the information being presented.

\begin{table}
    \caption{Epistemic marking paradigm in Eastern Geshiza, forms reproduced from \cite[584]{Honkasalo2019}, with order changed to reflect \textsc{speaker/non-speaker} typology.}\label{t:Discussion:EasternGeshiza}
    \begin{tabular}{|l|l|l|}
        \hline
                                 & Function               & Form          \\ \hline
        Higher speaker authority & ego-oriented           & unmarked      \\
                                 & non-shared information & \textit{-go}  \\
                                 & sensory                & \textit{-ræ}  \\
                                 & inferential            & \textit{-sʰi} \\
                                 & reportative            & \textit{-jə}  \\
        Lower speaker authority  & quotative              & \textit{-wo}  \\ \hline
                                 & \textit{unclear}       & \textit{-mə}  \\ \hline
    \end{tabular}
\end{table}

The conclusion drawn here is not dissimilar to that from Kurtöp above. To fully describe and analyse the system of epistemic marking in Eastern Geshiza, a more general, unified framework is necessary to avoid functional relationships between the various forms being obscured. This system in particular, while showing a lower level of variety in terms of the traditional categories represented than in Kurtöp, does also contain forms marking engagement-like meanings, in that their use is conditioned by the perspective of both the speaker and addressee in declarative structures. While this does not in and of itself support the argument that speakers are necessarily considering the relevant aspects of the perspectives of both themself and their addressee in any given speech act, Example \ref{e:Discussion:GeshizaDialogue} shows an interesting example of the non-shared information marker \textit{-go} in use.

\begin{exe}
    \ex\label{e:Discussion:GeshizaDialogue}
    \begin{xlist}
        \ex\label{e:Discussion:GeshizaDialogue:A}
        \gll \textit{e} \textit{smæŋa} \textit{gæ-mdze} \textit{æ-lə} \textit{ŋuə-go} \\
        \textsc{dem} girl \textsc{adjz}-beautiful one-\textsc{clf.indef} \textsc{cop.3-nsi} \\
        \glt `That girl is beautiful'

        \ex\label{e:Discussion:GeshizaDialogue:B}
        \gll \textit{ŋuə-ræ}. \textit{ŋuə-ræ}. \textit{mdze-ræ}. \\
        \textsc{cop.3-sens} \textsc{cop.3-sens} be.beautiful-\textsc{sens} \\
        \glt `Yes, yes. She is beautiful.'
    \end{xlist}
    Eastern Geshiza \cite[rGyalrongic: PRC,][593]{Honkasalo2019}
\end{exe}

The speaker in \ref{e:Discussion:GeshizaDialogue:A} sees a woman enter the room in which both speech act participants are standing, but behind the addressee. As such, in selecting which form to use, he uses the non-shared information marker \textit{-go} based on an assessment of his persective (he can see this directly) as well as that of his addressee, who he does not believe can see the woman. In the addressee's response in Example \ref{e:Discussion:GeshizaDialogue:B}, they agree that the woman is indeed beautiful, now using the sensory evidential marker \textit{-ræ}. The use of this form is clearly as they have now seen the woman in question and as such have direct sensory evidence for their statement. It perhaps seems evident and unimportant, but the use of this form here is also necessarily informed by the fact that the addressee is clearly aware that the original speaker knows this, given he first introduced the topic. While this seems a given, it is still a reflection of the fact that, even when using a form that would be traditionally categorised as evidentiality and not engagement, the second speaker is still making an assessment of the perspective of their interlocutor. In this case it is a straightfoward assessment given the context, it is still reflected in the choice of \textit{-ræ} over \textit{-go} in Example \ref{e:Discussion:GeshizaDialogue:B}.

This epistemic-marking paradigm is marked across multiple aspects, including on copulative and verbal clauses. This is unlike Kurtöp, which has distinct set of markers for each aspectual distinction. In Kurtöp, each of the epistemic markers also carry the aspectual meaning themselves (such that every form in Table \ref{t:Discussion:KurtopPerfective} could in fact be glossed as its epistemic meaning and perfective), whereas in Eastern Geshiza this aspectual meaning is encoded through a separate prefix. There are, however, some strategies for marking epistemic meaning outside of this paradigm, mostly involving lexical or periphrastic constructions. These include the use of verbs of perception encoding a sensory evidential meaning, as well as a nominalised construction with a specific copula, which neutralises any ego-oriented meaning and rather references an indirect base \cite[596]{Honkasalo2019}. As is also seen in other rGyalrongic languages \cites{Lai2017}{Gates2021}{Zhang2021}, this nominaliser shares a form with the inherited inferential marker \textit{-sʰi}. The polyfunctionality of this form is common across the family, and may have contibuted to its retention across the family from the family's common ancestor. This reduction of a claim over epistemic authority in nominalisation (or at least the similar forms) is interestingly similar to the ego neutralisation in Milang nominalised constructions, introduced in \ref{sss:Description:SpeakerNonSpeaker} and discussed further in Section \ref{ss:Discussion:SocialCases}.


\section{Social Conditions}\label{s:Discussion:Social}
The discussions above have primarily been focussed on meanings of forms in terms of conditioning factors related to the perspectives of the speaker and addressee, specifically their relationship to the information being presented in epistemic terms. That is, the areas of evidentiality, egophoricity, epistemic modality, mirativity, and engagement all refer to metapropositional information about the information presented in a speech act specifically from the perspective of the epistemic origo. They do not, at their core, reflect any assessment on the part of the speaker of their own relationship to either the addressee or a third party. This section argues that this is a shortfall of many previous (though, of course, not all) analyses of epistemic-marking systems, and that in many more cases than has previously been expected the interpersonal relationships of the parties relevant to a speech act also condition the use of epistemic marking. These are being labelled `social conditions'. This section presents an in-depth discussion of these social conditions, including the challenges that might be faced in analysis of data from the field, as well as a number case studies of documented cases of social conditions on epistemic marking in Amdo Tibetan, Ladakhi, and Milang.

The concept of the epistemic origo, discussed above in Section \ref{ss:Discussion:Origo}, assumes to some extent a single point of reference for the epistemic content of a given proposition. The origo is the individual, either a speech act participant or occasionally character in a narrative, from whose perspective any epistemic meaning is construed. As is discussed in Section \ref{ss:Discussion:Origo}, the origo is most commonly the speaker in declarative constructions and the addressee in interrogatives, though it can also reference the addressee in declarative structures or speaker in interrogative structures. The concept, however, struggles to handle situations with multiple perspectives being encoded or considered in a single speech act, a feature of the case studies discussed in Section \ref{ss:Discussion:MixedCases}. The social conditions being discussed here are handled about as well as other epistemic marking by the origo concept, with a key difference. While there is initially a clear origo distinction when the speaker is attentive to their own relationship to others, or when the speaker is referencing the relationships of the addresee, cases where the relationship between the speaker and addressee is relevant present a challenge in that relationships are an inherently two-way process\footnote{This is excluding the so-called ``parasocial'' relationships which have become a feature of contemporary media in which there is a perceived close, but one-way relationship between an individual and a public figure who does not know them. Given I am referring here to conversations between two individuals with a given relationship, the low likelihood of an interaction occurring between such an individual and public figure means I do not believe I need to account for this edge case.}, and as such it is not immediately possible to apply the origo to either of the speech act participants. It could be argued that a given relationship between two individuals can be viewed from the perspective of either individual, and that two individuals might not have the same perception of a relationship. It is nonetheless difficult to see how this could be represented in speech, given that, regardless of the epistemic origo or perspectives being represented, any information encoded in language is from the speaker, and any assessment of the perspective of the addressee is still necessarily filtered by the perspective of the speaker. While this problem of the speaker lens can be dismissed readily enough when discussing addressee perspective in specific reference to a piece of information, or in reference to an addressee's relationship with others, it cannot readily be dismissed here where the metapropositional information at hand still involves the speaker. That is to say that any reference to the relationship specifically between the speaker and addressee must be analysed as speaker origo. This is of specific relevance to the data presented below from Ladakhi.

As with other conditions of epistemic marking as discussed in Section \ref{sss:Description:SpeakerNonSpeaker}, the social conditions presented in the three case studies all appear to work to condition the appropriateness socially of a claim over epistemic authority by the epistemic origo. If the unmarked default of epistemic marking is generally the highest (or in some cases near highest, such as in Eastern Geshiza in Section \ref{ss:Discussion:MixedCases}), then social conditions act to temper the ability of the origo to make such a claim. This seems straightforward when considered as such - a claim over epistemic authority is reliant not only on the access of the origo to the information, but also of the access of the other speech act participant, and the relationship of the origo to their counterpart in terms of social hierarchy. Despite this, potentially for reasons discussed in Section \ref{ss:Discussion:SocialChallenges}, little has been published specifically on these social conditions. With all of this in mind, these social conditions are of interest as they add an extra dimension to the functions of epistemic marking, in turn adding further dimensions to the discussion on the ultimate functional motivations for the grammaticalisation of epistemic marking. Specifically, they provide a further scope of deictic reference, still relating to the perceived relationship of the speaker to the context of the utterance, but in this case with specific reference to their relationship to other individuals involved in the speech act in some way (participant or referent) rather than to the information itself. Of course, these conditioning factors are present alongside the relationship of the epistemic origo to the information itself in more established terms, and both of these are relevant to the claim of epistemic authority being exercised (or granted to) the origo.

\subsection{Challenges}\label{ss:Discussion:SocialChallenges}
There are a number of challenges associated with the documentation and subsequent analysis of epistemic marking. As discussed in Section \ref{s:Methods:FieldMethods}, perhaps due to their highly deictic and internal nature, the meanings of epistemic markings are often not consciously available to speakers, especially who have not either been formally education in their language (something highly limited a handful of languages with substantial institutional backing such as Lhasa Tibetan) or spent a substantial amount of time actively and analytically thinking about language (something which is essentially limited to linguists). Additionally, there are challenges in the analysis of naturalistic speech in the ability of the analyst to actually ascertain what the deictic context of a given speech act was in epistemic terms. That is, outside of controlled environments such as the activities used in Section \ref{s:Methods:FieldMethods} in studying Lhokpu, the analyst needs so know what the speaker already knows and how. This can be more easily accessed in the moment, with the linguist present for said context themself.

These two challenges apply as much, if not more, to the analysis of social conditions. Given they are very much secondary conditioning factors on the use of epistemic marking, and not the primary meaning, it can be expected that the presence of these conditions will be even less consciously available to speakers, and as such can only be identified alongside other epistemic contrasts and conditions. However, the identification and analysis of these conditioning factors requires two further pieces of knowledge on the part of the analyst, knowledge of both the relationship of the speech act participants and referents to each other, as well as any cultural aspects that might affect these relationships. There is an ongoing discussion about the role of colonialism in documentary linguistics and the inclusion of community and indigenous linguists in all processes, and I do not seek to contribute to it here in this setting, however this is a clear example of the necessity of deep cultural knowledge in linguistic description, and does give value to the idea that there are inherent limitations on a very removed and dis-personal approach to linguistic fieldwork. It is possible that the lack of documented cases of clear social conditions on epistemic marking is a result of the young age of the field and of widespread and in-depth description work. With this being said, the secondary nature of these social conditions means that descriptions of epistemic-marking systems that are not yet able to consider the social context of a given speech act are not inherently \textit{wrong}, but rather are not necessarily capturing the entire picture of the factors conditioning the use of the various forms.

\subsection{Case Studies}\label{ss:Discussion:SocialCases}
This section comprises three case studies of documented social conditions on the use of epistemic marking in Amdo Tibetan, Ladakhi, and Milang. Both Ladakhi and Amdo Tibetan are Tibetic languages with some etymologically related forms in their epistemic systems, though are descended from likely very distant branches of the Tibetic family. \citeA{Tournadre2023} place them in the North-Western and North-Eastern groups respectively, though do not provide any potential phylogeny for the Tibetic family. \citeA{Bialek2018} suggests that Western Archaic Tibetan, a parent language or subgroup that would include Ladakhi, may exist as the outgroup and earliest separated subgroup of the Tibetic languages. This is to say that, while both clearly Tibetic, and while both do tend to be more phonologically conservative \cites{Tribur2019}{Zemp2018}, it is unlikely that they are members of branches within the Tibetic subfamily that are closely related.

Milang is classified by \citeA{VanDriem2014} as being one of two members of the Siangic subfamily, along with the geographically non-contiguous Koro. This subgroup is one of the less confident ones, as an undetermined substrate and large amounts of contact with neighbouring Tani languages have made historical analysis more difficult \cite{Modi2017}. Regardless, Milang is, phylogenetically speaking, likely very far removed from the Tibetic languages discussed above.

All of these case studies show quite different references to social structures and relationships in epistemic marking, suggesting that there is a great deal of breadth in how these conditions can manifest, as well as how widespread they are. As is discussed above in Section \ref{ss:Discussion:SocialChallenges}, the identification of such systems is mired by the often opaque or less readily accessible nature of social structures to researchers. This is reflected in the fact that two of the researchers responsible for the analyses discussed here are able to provide a greater deal of insight into cultural practises than might otherwise be possible for many other researchers. Namely, Bettina Zeisler has spent multiple decades working with Ladakhi speakers and communities, and Yankee Modi is an indigenous linguist and Adi community member for whom Milang is a heritage language.

\subsubsection{Amdo}\label{sss:Discussion:AmdoCase}
The core of the social conditions affecting the use of epistemic marking in Amdo Tibetan is the flexibility of the use of the egophoric epistemic base beyond the archetypal egophoric distribution of first-person in declaratives and second-person in interrogatives. This is by no means new or unique to Amdo Tibetan, and has been discussed in Section \todo{reference to unwritten introduction i think}. In Amdo Tibetan, egophoric marking can be extended to third-party subjects in certain conditions, in part governed by the relationship between the epistemic origo and the third-person referent \cite{Tribur2019}.

Amdo Tibetan copulas mark a primary two-way distinction between egophoric and non-egophoric, along with a large number of compund forms marking further epistemic distinctions. As with many Tibetic languages, these copulas further exist in two sets, equative and existential. As such, there are four copulas whose distribution, along with their analysis in \citeA{Tribur2019} is of interest here. These are the equative egophoric and non-egophoric\footnote{\citeA{Tribur2019} uses the term \textsc{allophoric}.} copulas \textit{jɪn} and \textit{ʐɛ}, as well as their existential equivalents \textit{jot} and \textit{jokə}. The various compound copulas are comprised of the two egophoric copulas \textit{jɪn} and \textit{jot} some other morphological components, including but not limited to the non-egophoric equative copula \textit{ʐɛ} and various nominalisers.

For the most part, this egophoric distinction follows the archetypal distribution, but there are some cases where this is not necessary. Interestingly, \citeA{Tribur2019} notes variation in the strictness of this distribution between varieties. Namely, she notes that while there are cases in both the Gcig.sgril (Jigzhi) and Rnga.ba (Ngawa) counties, which are neighbouring across the border between Sichuan and Qinghai provinces, constructions are allowed in Rnga.ba which are strongly dispreferred in Gcig.sgril. A number of examples are given in \ref{e:Discussion:Amdo}, presenting data on both varieties from \citeA{Tribur2019}.

\begin{exe}
    \ex\label{e:Discussion:Amdo}
    \begin{xlist}
        \ex \label{e:Discussion:Amdo:A}
        \gll ŋɑ sɨ rɛt? ŋɑ ɑʑɑŋ rɛt. \\
        \textsc{1s} who \textsc{eq.allo} \textsc{1s} uncle \textsc{eq.allo} \\
        \glt `Who am I? I am Uncleǃ' Gcig.sgril variety (p.311)
        \ex \label{e:Discussion:Amdo:B}
        \gll tə ŋi nəwu jɪn \\
        \textsc{def} \textsc{1s.gen} younger.brother \textsc{eq.ego} \\
        \glt `That is my younger brother.' Rnga.ba variety (p.312), not allowable to speakers in Gcig.sgril.
        \ex \label{e:Discussion:Amdo:C}
        \gll ɑtɕe jɪɖoŋ yu-gə ɸɕɪmtsʰo-na jo \\
        elder.sister Ye.Sgron up-\textsc{gen} 'Phyi.mtsho-\textsc{loc} \textsc{exist.ego} \\
        `Sister Ye.Sgron is up at 'Phyi.mtsho Lake.' Gcig.sgril variety (p.313)
    \end{xlist}
    Amdo Tibetan varieties \cite[Tibetic: PRC,][311-313]{Tribur2019}
\end{exe}

Example \ref{e:Discussion:Amdo:A} is, contextually speaking, baby talk. Here, an uncle is speaking to his young nephew. Of note is the non-egophoric equative copula \textit{rɛt} in the first person statement `I am Uncle!'. \citeA[311]{Tribur2019} notes that this construction is specifically grammatical because the addressee is a young child, and in fact it would be unusual here to use the standard egophoric distribution of egophoric in first person declaratives. Tribur's explanation for this rings true, adults are not attempting to hold a serious and legitimate conversation with young children, especially those still pre-verbal, as the one being spoken to in \ref{e:Discussion:Amdo:A} was. Rather, they are modelling speech to encourage the babies to learn, an as such are speaking entirely within the addressee's perspective in epistemic terms, despite still referring to themself in the first person. That is, the epistemic origo is firmly anchored with the baby addressee. As such, the key factor conditioning the selection of the epistemic marker here is not any factor in the actual information being presented, or the relationship of either speech act participant to said information, but rather an trait of actual individual taking the role of addressee, along with the relationship of the speaker to this addressee. This is clearly not an example of a highly nuanced distinction where the speaker is judging their position in a social hierarchy relative to the addressee, but it is still an example of epistemic marking being conditioned by the speech act participants themselves, along with the social desire to model speech for babies.

Example \ref{e:Discussion:Amdo:B} shows a construction that has been noted more widely in egophoric marking \cite{EgoIntro}, in which egophoric marking can be extended to close relations such as family members. Here, the speaker uses the egophoric equative copula \textit{jɪn} to mark a level of epistemic authority more or less equivalent to knowledge they would have over themself, the fact that a given individual is their brother. This construction is also seen with habitual actions of family members. This example comes from a different variety of Amdo Tibetan to the other two examples however, from the Rnga.ba county to the North-West of Gcig.sgril, and was considered ungrammatical in Gcig.sgril. This judgement was not limited to this cosntruction, but extended to others with the equative copula. This restriction does not apply in Gcig.sgril to similar familial egophoric constructions using the existential copula, however, as seen in Example \ref{e:Discussion:Amdo:C}, in which the egophoric existential copula \textit{jo} can in fact be used with non-first person statements referring to close relations, here such as the speaker's elder sister. It is clear here that there is a social assessment being made of whether or not the subject of the statement is close enough to the speaker to use the egophoric marker and make such a high claim of epistemic authority. Tribur does herself note this, suggesting that the construction is equating the authority of knowledge over one's own self to knowledge over one's own family. It is not clear at this stage why the existential copula shows less restriction over the use of the egophoric copula than the equative, but there is potentially some factor surrounding the idea that the existential copula is simply denoting the very existence of something (or in this case someone), which can be directly observed by the speaker, while the equative copula can denote information not directly outwardly available to the speaker. Further North still in the Rebkong area of Amdo Tibetan, this non-first-person egophoric marking is covered by an entirely separate form \textit{jənnəre}, described by \citeA[300]{Simon2021} as the \textsc{ego-authoritative}\footnote{L'égo-autoritatif}. This form marks specifically marks exclusive information that is assessed as exclusive to the epistemic origo. While the exclusitivity of information is not necessarily linked to any social factors, \citeA{Simon2021} provides an interesting example of the form which does, at least in one specific case, suggest a secondary influence of social structures on the use of the form.

\begin{exe}
    \ex \label{e:Discussion:AmdoAuth}
    \gll \textit{təxe} \textit{niɕe\textsuperscript{F}-gə} \textit{kore} \textit{nakko} \textit{nakko-sək} \textbf{\textit{jənnəre}} \textit{jaː} \\
    so barley.flour-\textsc{gen} bread black black-\textsc{indf} \textbf{\textsc{equ.ego.aut}} \textsc{disc} \\
    \glt `Barley flour bread, like this, is a black, black one.' \\
    Rebkong Amdo Tibetan \cite[Tibetic:PRC,][300]{Simon2021}
\end{exe}

In Example \ref{e:Discussion:AmdoAuth}, the speaker, a farmer, is describing traditional breads to the author, who is a foreigner. The speaker is making an assessment of his addressee's knowledge surrounding traditional bread-making practises. It has not been discussed in detail in this thesis that these assessments of the perspective and awareness of the addressee need themselves to be informed by the knowledge of the speaker. These assessments could be informed by any number of factors, but the key here is that this assessment on the part of the speaker appears to be based on the addressee's position as a foreigner. That is, the speaker is able to claim a higher level of epistemic authority in the use of this non-shared-information ego-authoritative marker because of the addressee's social position as a foreigner. This is admittedly not as straightforward a case of social conditions on the use of epistemic marking than the other examples, though it does begin to suggest that even where social status is not a clear and direct factor in the selection of epistemic forms, assessments of the perspective of the addressee still may be informed by social relations in ways that do not appear to have been widely explored.

In Amdo Tibetan, the factor conditioning the use of the egophoric form is, at least in part, the relationship between the speaker and the third party referent. It is not clear, however, if the relationship of the addressee is relevant here. That is, could the speaker in \ref{e:Discussion:Amdo:B} or \ref{e:Discussion:Amdo:C} use the egophoric copula if their addressee was also a family member of the referent, or at least had a similarly close relationship. That is, is the important factor the closeness of the epistemic origo to the referent, or the relevant closeness of the epistemic origo to the referent compared to the interlocutor? This is, unfortunately, not clear at this stage. While there are no shortage of clear cases where the relative access of the speaker and addressee in terms of knowledge is important, inlcuding in the Rebkong variety of Amdo Tibetan, Kurtöp and Eastern Geshiza discussed in Section \ref{ss:Discussion:MixedCases}, as well as cases such as Bunan (West Himalayish: India) where this same factor conditions the use of the egophoric \cite[469]{Widmer2014}, these cases are all limited to the relative access of the speech act participants to the knowledge at hand, rather than any social factor. It can perhaps be expected that this will extend to social factors in some cases. That is, there is very possibly a situation in which the epistemic origo need not only be socially close to the referent, but specifically closer than their interlocutor in order to claim the higher level of epistemic authority. If this does exist, or if this is a factor in systems such as the one presented here for Amdo Tibetan, is not yet clear to me from the available literature. Regardless, it is clear from the data presented here from \citesA{Simon2021}{Tribur2019} that the use of epistemic marking in Amdo Tibetan varieties is conditioned by social conditions, specifically in this case the nature or closeness of the relationship of the epistemic origo to a third-party (animate) referent in social terms, as opposed to the more widely discussed relationship in terms of knowledge.

\subsubsection{Ladakhi}\label{sss:Discussion:LadakhiCase}
In Amdo Tibetan, the social conditions on the use of epistemic marking were limited to the relationship between the epistemic origo and a third-party referent, as well as only to the closeness of the relationship and the ability that granted to claim epistemic authority. In Ladakhi (Tibetic: India), however, the scope of these social conditions appears to be much broader.

Relevant to this case study is a distinction between two epistemic bases, which \citeA{Zeisler2018} describes as the \textsc{General Evidential Marker} (GEM), and an \textsc{assertive} form. This assertive base does not mark any specific evidential meaning, but rather mark a claim of authority on the part of the origo, typically the speaker \cite{Zeisler2018a}. As a result, and as is seen with much egophoric marking, the assertive base is often used in first-person constructions, though it is not entirely restricted to this egophoric distribution. Notably, however, this is not an egophoric marker, and it does not mark information as held by the origo with any sort of superiority. Rather, it marks a higher level of confidence on the part of the origo, or that the information is clearly true, widely known, or not at issue or able to be questioned. While this does not mark any asymmetrical access to the information by the origo as an egophoric or non-shared information marker might, it still involves a highers claim over epistemic authority by the origo. This is as they are making a stronger claim over the validity of the information, as well as making a strong claim over the perspective of the addressee with regards to the information, as opposed to forms marking direct evidence, which provide a specific source of the information as a means of hedging the necessarily claim of authority.

\citeA{ZeislerForthcoming} reports a small number of attested in which the ability of a speaker to claim epistemic authority is conditioned not only by the standard factors discussed in epistemic marking as discussed in Section \ref{ss:Discussion:MixedFoundation} and the social relationship between the epistemic origo and the third-party referent, but also by the social status of the epistemic origo in hierarchical terms. That is, the ability to claim or make judgements on epistemic authority is also conditioned the wider ability of the origo to hold authority in social terms. The specific example given by Zeisler discussed here is not, notably, speech that occurred in situ, but rather is a hypothetical interaction which was reported to be highly typical by a speaker.

In this interaction, a young member of a village has assisted in organising a village meeting with the support of the village head. The village head announced the time and place for the meeting, but when the time came, few village members were on time and some did not arrive at all. The young village member who has organised the meeting then confronted these people who did not show up, one of whom in particular reacted negatively to his confrontation, and reprimands him, suggesting that he is too young to be speaking to her as such. In reponse to this reprimand, the village head then intervenes and repeats the same confrontation, this time with no recourse from the addressee. The interaction, while not an actually attested interaction, was proposed to be highly typical and believable. The interaction itself is quite long and the majority of the actual language is not so relevant to this analysis. As such, I have only included the actual relevant clause using the assertive marker in Example \ref{e:Discussion:Ladakhi}, with the overall translation given below\footnote{``Following yesterday's meeting, \textbf{all of us know it well (authoritative)}: today [we were supposed] to meet at ten, but nobody came on time.'' Then one lady became angry [and said]: ``Who are you to tell us that \textbf{we all know it well}?! You are, as it appears, still wet behind the ears! What [kind of manner] is this, talking to us in this way?! If the village head speaks like this, it is okay. But who, [do you think], are you?!'' \textit{Later, the village head confronts the lady.} ``Following yesterday's meeting, \textbf{all of us know it well (authoritative)}: it was agreed to meet at ten today, so why didn't you come on time? And why did you wrongly scold the youngster?'' Then that lady couldn't say anything any more (lit. was left with the mouthopen wide).\cite[77]{ZeislerForthcoming}}.

\begin{exe}
    \ex\label{e:Discussion:Ladakhi}
    \gll ...oɣo tshaŋma+(ː) gju ɦot... \\
    we.incl all+\textsc{aes} knowledge \textsc{ass}.have \\
    \glt `...all of us know it well...[that the meeting was at ten]' (authoritative)
    Ladakhi \cite[Tibetic:India,][77]{ZeislerForthcoming}
\end{exe}

Here, the young man who organised the meeting uses the assertive form \textit{ɦot} in the statement that the details of the meeting were widely known. He is not providing any source for this knowledge or marking the information as coming from the village head, but is stating as an incontestable fact. This high claim of epistemic authority, specifically here over the perspective of the addressee (an older village member) is not taken well. Specifically, the older addressee's angry response notes that the use of \textit{ɦot} is inappropriate precisely because the original speaker is young. That is, there is no issue with the actual epistemic content of the marking, or the truthfulness of the statement. On the basis that the village head did in fact notify the community that the meeting was happening, this information was known to the addressee, and it ought to be fairly general knowledge. It is not explicitly stated here whether or not the statement would have been seen as less improper if the GEM form was used to avoid making such a strong claim over the epistemic authority, nor is an alternative more acceptable construction given, however the same construction when used by the village head in Example \ref{e:Discussion:Ladakhi:Head} does not provoke the same negative response from the addressee. Given all else is equal here, it is clear that it is the social status of the speaker in relation to the addressee that is governing the ability of the speaker to use this assertive form, and in turn the degree to which they can claim epistemic authority. Unlike in Amdo Tibetan, where the relevant relationships were between the speech act participants and other third-party referents, here the relevant relationship can be seen as being that between the speech act participants. Arguably, the direct relationship between the two speech act participants is not actually reflected here, but rather both of their positions within the relevant social hierarchy, though given these statuses are measured relative to each other (hence the ability of the village head to use the assertive form), I do not believe that these are different enough to warrant differentiating them in this analysis.

\subsubsection{Milang}\label{sss:Discussion:MilangCase}
The epistemic system in Milang is described in some detail in Section \ref{sss:Description:SpeakerNonSpeaker}, but will be briefly introduced here again. The core contrast that will be discussed here is on described by \cite{Modi2017} as an egophoric distinction. Notably, this distinction is not marked by any dedicated morphology, but rather all unmarked clauses carry a strong egophoric meaning unless actively neutralised through the use of nominalisation constructions.  While Modi uses the term egophoric to mark this distinction, it is not restricted to marking first-person statements and second-person interrogatives, nor are its third-person uses restricted to cases where the speaker is socially close to the third-party referent as in Amdo Tibetan. Rather, the use of the unmarked construction marks a strong claim over epistemic authority by the speaker, regardless of their actual participation or social proximity to the statement in question. Example \ref{e:Description:MilangEgo} is reproduced in Example \ref{e:Discussion:MilangEgo}, showing both a first person and third person statement in the default, unmarked, high epistemic authority construction. If the speaker is not in a position to make such a claim over epistemic authority, that is, if they do not have the expected direct knowledge over the event, they must neutralise this meaning with the nominalisation construction seen in Example \ref{e:Discussion:MilangNonEgo} reproduced from \ref{e:Description:MilangNonEgo}. Here, the nominaliser \textit{ɲi} neutralises the authoritative meaning, after which further epistemic distinctions can be made in the forms \textit{la} \textsc{reportative} and \textit{pɨ} \textsc{uncertain}, among others.

\begin{exe}
    \ex\label{e:Discussion:MilangEgo}
    \begin{xlist}
        \ex \label{e:Discussion:MilangEgo1}
        \glll ŋa tutu. \\
        ŋa tu-tu \\
        1.\textsc{sg} eat-\textsc{pfv} \\
        \glt `I ate.' (p. 455) \\

        \ex \label{e:Discussion:MilangEgo2}
        \glll joon bozar yitu. \\
        joon bozar yi-tu \\
        John market go-\textsc{pfv} \\
        \glt `John went to the market.' (p.456) \\
    \end{xlist}
    Milang \cite[Siangic: India,][]{Modi2017}
\end{exe}
\begin{exe}
    \ex \label{e:Discussion:MilangNonEgo}
    \glll joon bozar yituɲila | yituɲipɨ \\
    joon bozar yi-tu-ɲi-la | yi-tu-ɲi-pɨ \\
    John market go-\textsc{pfv}-\textsc{nzr:subj}-\textsc{rep} | go-\textsc{pfv}-\textsc{nzr:subj}-\textsc{ucrt} \\
    \glt `John went to the market. (I am told) | (I am not sure)' \\
    Milang \cite[Siangic: India][457, given as two examples in source and combined here]{Modi2017}
\end{exe}

In Milang, it is not so much the availability of this higher claim over epistemic authority that is conditioned by social factors as with Ladakhi and Amdo Tibetan, but rather that the use of the unmarked authoritative form influences the use of other forms by other speakers in social terms. In Milang, the claim over epistemic authority seen in unmarked clauses extends socially in that it is considered impolite (though still grammatical) to question such claims. It would be highly inappropriate, if not blatantly rude, to ask a follow up question such as ``how do you know?'' or ``really?'' to a statement such as \ref{e:Discussion:MilangEgo2}, even though the speaker was neither a first-person participant, nor have they given any source for their knowledge. The origo for this claim is, interestingly strictly limited to the speaker. Interrogative constructions, where the epistemic authority is being passed to the addressee, must involve the neutralising nominalisation construction, seen in Example \ref{e:Discussion:MilangInter} \cite[457]{Modi2017}.

\begin{exe}
    \ex \label{e:Discussion:MilangInter}
    \glll joon bozar yituɲaa \\
    joon bozar yi-tu-ɲi aa \\
    John market go-\textsc{pfv-nzr:subj} \textsc{tag} \\
    \glt `Did John go to the market?' \\
    Milang \cite[Siangic: India,][457]{Modi2017}
\end{exe}

While this case study does not show to the same extent as the others that social factors can condition the selection and use of epistemic marking, it does show that there is a broader link between epistemic marking and social factors in both directions. In all cases, Milang included, the connection between epistemic marking and social factors has been centred on claims over epistemic authority, specifically on the rights of an individual to make such claims given either their relationship with the referent or their addressee, as well as the social implications of such claims on the acceptability of, in the case of Milang, any questioning of the information present authoritatively.

\section{Reference to Perspective}\label{s:Discussion:Perspective}
The importance of perspective as a concept in epistemic marking, if not in pragmatics as a while is well established \cite{Evans2005View}. Any deictic reference, epistemic or otherwise, that is any reference in speech to the world in which the speech act is taking place will necessarily come from the perspective of the speaker. While this is not a place for an in-depth discussion on subjectivity and objectivity of thought or truth, it is not particularly contestable that speech, or more generally language, in being a largely subconsciously produced phenomenon is reflective only of the world from the perspective of the speaker. At the most obvious end of this is the fact that an individual cannot speak about something they are not aware of. One cannot report the presence of an object one has not seen, nor can one, for instance, bring up the politics in a place of which they have not heard. This is not to say that speaker cannot pretend to know more than they do, but rather that the language is necessarily anchored to the actual awareness of the speaker, and not some omniscient narrator. Less obvious implications of this include the idea that two speakers can hold a conversation entirely within the restraints of the cooperative principle in terms of truthfulness, and yet still either disagree or draw differing conclusions as a result of their potentially varied perspectives. The cooperative principle itself can be a victim of this, in that, as discussed in \todo{ref chapter 1?, my chapter in bern book}, the maxims of relevance and quantity might suggest different conversational expecations for different speakers with different perspective. This is not a revolutionary, or even new, idea, but it is a useful foundation for the discussion of perspective here, and in this chapter as a whole.

Epistemic marking is inherently deictic, in that is references aspects of the context of the speech act. This deixis means that it is even more clearly reflective of the perspective of the speaker. While any given statement, as suggested above, is influenced by the perspective of the speaker to some extent, epistemic meaning reflects the persective of the speaker (or some other individual, discussed below) as its core meaning. In contrast to some other forms of deictic meaning, such as tense marking directly relative to the current moment (where the tense locus is the moment of speech) or spatial deixis, however, epistemic marking does not functionally form part of the actual meaning of the proposition to which it is attached. That is, statements such as `he went over there', `he went here' and `he will go over there' are describing three different events. They have different propositional content. Compare this to, for example `he went over there (I saw it)' and `he went over there (I was told)'. In both cases, the actual event being described, the proposition itself, is the same. Rather, the epistemic meaning is metapropositional, it provides information about the proposition and not actual content to the proposition.

I argue in \todo{reference chapter 1} that this underlying reflection of the perspective of the speaker across any language can be extended to an assessment by the speaker of the perspective of the addressee. This argument is largely based on the cooperative principle, and in particular the maxims of relevance and quantity \cite{Grice1989}. These state that a speaker will aim to limit information presented to that which is relevant to the speech act at hand, and that they will not say any more or less than is necessary to communicate their point, a judgement I argue requires a consideration of the perspective of the addressee and more specifically their state-of-mind and knowledge regarding the topic or referent at hand. For example, in a conversation about the traffic on the way to the university, it would not require an assessment of the perspective of the addressee for the speaker to determine that a comment about the speaker's mismatched socks is not relevant. However, in the same conversation, an assessment of the addressee's perspective by the speaker would be necessary to determine if a comment about a car accident nearby would fir the maxim of quality, or how to present such information. A bare, epistemically unmarked statement of a state-of-affairs already known to the addressee can be perceived as rude, or at the very least cause a breakdown in communication as it suggests the speaker thought the addressee did not know about, for instance, the car accident. Having made an assessment about the knowledge of the addressee of this state-of-affairs, however, the speaker can avoid this breakdown and properly mark the information as shared. This process appears to occur widely, either through periphrastic constructions such as English `as we both know, there was a car crash' or more succinctly `of course, there was a car crash', or grammatically, as has been discussed in this thesis. These references to the perspective of the addressee also occur more explicitly in interrogative structures, where authority over information is given to the addressee \cite{Hill2020}.

In sum, reference to perspective lies to an extent at the centre of language and conversation in general, but is intrinsically and inalienably linked to epistemic marking at a functional level. The importance and unavoidable nature of assessment of both the speaker and the addressee and their perspectives in relation to one another has previously been established \cite{Heritage2012}, and many analyses of the necessary asymmetry between the knowledge of the speaker and addressee have previously been undertaken (See \citesA{GonzalezPerez2023}{Kamio1997}, along with an in-depth literature review in \citeA[4]{Heritage2012}). \todo{more here, kinda got lost about what i was saying}
\subsection{The Origo}\label{ss:Discussion:Origo}
The term \textit{origo} refers to the reference point or anchor of deictic meaning \todo{cite}. While it is a useful term and concept, and is used widely throughout this thesis, there are a number of issues with the idea that arise in the discussion presented in this thesis. This section will discuss the concept of the origo, the theoretical differences between the different positions the origo can take, as well as whether or not the origo in any conceptualisation holds weight as anything more than a useful concept in the analysis of deixis.
\subsubsection{Theoretical Types of Origo}
The proposal of the conversational presumption suggesting that the shift from speaker to addressee-perspective in interrogative constructions is natural and unremarkable clearly holds in many cases, if not the majority. However, it does not always hold. There are documented cases of epistemic markers which do not undergo an origo shift in interrogatives, as well as cases which do not clearly fit into this model at all as they mark both speaker and addressee persective in both interrogative and declarative constructions, both of which have been discussed in this thesis. With these variations, a two-dimensional schema for categorising how a given form interacts with the epistemic origo and perspective marking can be developed, grouping forms by both whether or not the origo shifts in interrogative structures, as well as whether the form reflects a single perspective, or two.

\begin{table}
    \begin{tabular}{r c c}
        \hline
                             & Origo-Shifting                        & Non-Origo-Shifting             \\ \hline
        Single-Perspective   & archetypal evidentials and egophorics & some evidentials and miratives \\
        Multiple-Perspective & some engagement marking               & archetypal engagement marking  \\ \hline
    \end{tabular}
    \caption{Matrix allowing for the characterisation of perspective-marking forms according to the number of perspectives and shifting of perspectives. Reproduced from \citeA[77]{BodnarukForthcoming}.}
    \label{t:Discussion:OrigoShift}
\end{table}

The four resultant categories are given in Table \ref{t:Discussion:OrigoShift}, with some examples of the types of marking that would fit into the category in terms of the traditional cross-linguistic categories. The first cell shows forms which mark a single perspective and shift in interrogative clauses. I have not come across any forms in the survey in which the perspective of the addressee is marked in declarative clauses and the inverse in interrogatives. As such, this category represents forms marking speaker perspective in declaratives and addressee perspective in interrogatives. This includes archetypal evidentials and egophorics \cite{Aikhenvald2004}{EgoIntro} and represents the conversational presumption as described by \citeA{Hill2020}. The next cell, representing forms which mark a single perspective but do not shift in interrogative constructions include forms which can mark either speaker or addressee perspective. As was discussed in Section \ref{sss:Description:AddrPersp}, Meithei \cite[Internal Isolate: India,][]{Chelliah1997}, the inferential epistemic marker \textit{-ǰat} marks a counterexpective meaning with a speaker origo when combined with an interrogative marker, shown in Example \ref{e:Discussion:MeitheiInter}. It does not exhibit an origo shift from speaker to addressee.

\begin{exe}
    \ex \label{e:Discussion:MeitheiInter}
    \gll má ŋəraŋ skul čə́t-pə-\textbf{ǰat}-lə \\
    he yesterday school go-\textsc{nom-\textbf{type}-int} \\
    \glt `Could it be that he went to school yesterday!?' \\
    Meithei \cite[Internal Isolate:India,][296]{Chelliah1997}
\end{exe}

\citeA{HengeveldOlbertz2012} note that a typological characteristic missing from the original description of mirativity in \citeA{DeLanceyMirativity1997} is the ability for the markers to reflect addressee perspective in declaratives. An example of this can be seen in Duhumbi (Kho-Bwa: India), in which the copula \textit{le} marks information as new and recently acquired (i.e. mirative), specifically to the addressee \cite[405]{Bodt2020}. Rather than shifting to relfect the alternative (here, speaker) perspective in interrogatives, however, the form is simply never attested outside of the declarative. In this, the form does reflect a single perspective, and also does not undergo any shift of the origo from this perspective in interrogatives, simply because it does not occur in interrogatives. This does, interestingly, mean that these single-perspective, non-shifting forms are attested as marking either speaker or addressee perspectives, and not only speaker perspectives as might be expected if the perspective of the speaker is seen as the default.

Archetypal engagement marking, such as that of Andoke presented in \citeA{EvansBergqvistSanRoque2018a}, shows epistemic marking which reflects the perspectives of both the speaker and the addressee. In marking both perspectives, archetypal engagement marking does not formally or functionally change in interrogative constructions, placing it in the lower right cell of Table \ref{t:Discussion:OrigoShift}. This is not necessarily always the case, however. While I have not come across any specific analyses of engagement or engagement-like marking in \lfam\ languages which specifically clarify how the origo acts in interrogative structures, \citeA{SchultzeBerndt2017} gives data from Jaminjung and Ngaliwurru (Mirndi family: Australia) in which a non-shared information marker \textit{=ngarndi} can be used in declarative or interrogative structures. In declarative structures, the clitic marks that the speaker has a higher level of knowledge or epistemic authority over the information than the addressee, while in interrogatives it marks the opposite. Here, the perspectives of both speaker and addressee are always marked, but the specific meaning of each form in terms of these perspectives flips between declaratives and interrogatives, placing this form in the lower left cell of multiple-perspective, origo-shifting forms.

This final category and its description begin to highlight a potential issue with the concept of the origo, specifically in reference to systems in which multiple perspectives are marked. The construal of the epistemic origo as a theoretical entity representing the point of reference for epistemic (or more generally deictic) meaning which can be attached to a given speech act participant does not work so well with forms marking multiple perspectives. Does the origo here attach to both SAPs, or is the concept simply less useful. This is represented visually in \ref{f:Discussion:Origo}. In the general construal of the origo, the metapropositional epistemic meaning \textit{M} is assigned to the entity of the origo, which in turn can attach to either the speaker or addressee (or a character in a narrative). If the origo shifts to another individual, the perspective from which the metapropositional meaning is construed shifts with it. In single term systems where there is no change of perspective or epistemic meaning in interrogatives meaning, it it sufficient to simply state that the origo does not shift in this given form.


    \begin{figure}
        \centering
\underline{Single-perspective}
\[M \rightarrow O \rightarrow \begin{Bmatrix}
    \textsc{speaker} \\ 
    \textsc{addressee} \\ 
    \textsc{character}
\end{Bmatrix}\] \\

\underline{Non-shifting Multiple-perspective} \\
\(M_{\textsc{speaker}} \rightarrow{\textsc{speaker}}\) \\
\(M_{\textsc{addressee}} \rightarrow{\textsc{addressee}}\)
\caption{An illustration of the origo receiving some metapropositional meaning \textit{M}, and assigning this to some single perspective-holder, contrasted with a construal of the same process in non-shifting multiple-perspective forms without any origo entity, in which the component meanings are directly assigned.}\label{f:Discussion:Origo}
    \end{figure}

The marking of multiple perspectives is harder to explain with this model, however. There is no single individual to which the origo can be attached, and neither is there a single epistemic metapropositional meaning that can be assigned to a single origo. In archetypal engagement systems in which there are no changes in the assignment of given meanings to given perspectives in interrogative constructions it is also not useful to construe a system of two origos, each with its own meaning assigned. The epistemic meanings of forms in these systems are inherently and consistently tied to a specific speech act participant. That is, a form such as Andoke \cite[Isolate: Colombia][117]{EvansBergqvistSanRoque2018a} \textit{k-} always marks a lack of speaker knowledge and a presence of addressee knowledge, while \textit{kẽ-} always marks the inverse. There is no reason here to argue for a moveable meaning-carrying unit, but rather it appears more sensible to illustrate metapropositional meaning as being assigned directly to each perspective. In cases where there is a shift between declarative and interrogative uses, there is an arugment for a moveable origo which assigns some (higher authority) meaning to the speech act participant it is attached to, and a secondary meaning to the other SAP (e.g. \textit{\textsc{origo} knows this, \textsc{other} does not} as opposed to \textit{I know this, you do not}). This remains unproductive for the description of systems where this shift is not shown to occur. The usefulness of the origo as a theoretical entity overall is further challenges by the argument that reference to addressee-perspective is much more widespread than has previously been suggested, that there is a consideration of the perspective of the addressee by the speaker in many more situations than and grammatical forms due, as shown with mixed paradigms and social conditions in Sections \ref{s:Discussion:Mixed} and \ref{s:Discussion:Social} above.

Putting this thought aside for the time being, in the single perspective constructions where the concept of the origo is more readily applicable, it can be seen to take three distinct forms. There are clear differences in the reference to single perspective between the three most common positions for reference to perspective, namely declarative speaker, interrogative addressee, and declarative addressee. These differences, namely between reference to addressee perspective in declarative and interrogative clasuses, have previously been discussed by \citeA{Bergqvist2017}, who notes the two as separate though does not make any strong conclusions about the implications of this. Also attested, though less commonly, is the presence of explicitly marked speaker perspective in interrogative constructions, which will also be discussed briefly.

\paragraph{Declarative Speaker}
The declarative speaker origo can be seen as the default or unmarked position of the origo. It is, across both the data collected for this survey and the literature as a whole, the most common point of reference for both epistemic and other deictic meaning. It is understandable that reference from the speaker to their own perspective in declarative constructions is the most common. As was discussed above, speech is necessarily constructed in terms of the perspective and state-of-mind of the speaker, given the simple fact that an individual cannot speak outside of their own awareness. Given this, there is not a huge amount to be said about the nature and implications of the origo in this context, but rather that it acts as something of a definitional foundation against which the other coordinations of perspective will be contrasted.
\paragraph{Interrogative Addressee}
As has been mentioned throughout this thesis, the shift of the origo from the speaker to the addressee in interrogative structures can be seen as a natural outcome of the shift from declarative to interrogative. The expectation of authority over knowledge sits with the speaker in declaratives and with the addressee in interrogative. This can be seen as the core purpose of interrogatives, in that they are necessary when the speaker does not have some piece of knowledge but believes that the addressee does. This aligns this coordination of the origo with that of the declarative speaker, they are both logical outcomes of the underlying pragmatics of the conversational presumption per \cite{Hill2020}. They are still, however, fundamentally different in that only one reflects an actual and incontestable knowledge of the perspective of the origo holder. As discussed above, a speaker can only speak in terms of their own state-of-mind, and as such can only truly reference their own perspective. A shift of the perspective to any non-speaker referent requires the speaker to make an assessment of the perspective of this referent, an assessment which is still necessarily informed by the speaker's own perspective. That is, while reference to the perspective of the speaker is just that, reference to the perspective of the addressee is in fact reference to a projection of the perspective of the addressee via the perspective of the speaker. This is an important distinction as it carries with it implications for the truthfulness of epistemic marking and possible other discourse factors. At its core, reference to speaker perspective is incontrovertible. It is knowledge that is totally and inherently limited to the speaker, and its truthfulness cannot be assessed by others unless it is represented in a way that is at odds with the internal knowledge of another. For example, a speaker uses a visual evidential for an event that they were not actually present for, but the addressee was. Here, the addressee's own internal knowledge is at odds with the perspective presented by the speaker, and they might contest it. This aside, this incontrovertibility of speaker perspective is not in any way present in addressee perspective, as the addressee necessarily has a higher level of authority over their own perspective. This difference is reflected, for instance, in Milang (Siangic: India, presented in detail in Section \ref{sss:Discussion:MilangCase}), in which the use of the higher epistemic authority construction disallows the questioning of information socially. This construction cannot be used in interrogatives, to an extent reflecting the inability of reference to the perspective of the addressee to be presented with such a degree of confidence or authority.
\paragraph{Declarative Addressee}
The declarative addressee, in also referencing the perspective of the addressee, is similarly distanced from the declarative speaker origo coordination by the additional step of the projection of the addressee's perspective through that of the speaker. It does not, however, fit within the pragmatically natural distribution of perspective per the conversational presumption. Reference to the perspective of the addressee in declarative structures involves reference to a perspective other than that of the expected holder of the primary epistemic authority in pragmatic terms. With this, it is perhaps better to consider these coordinations in terms of whether or not they are pragmatically congruous rather than if they are declarative or interrogative. That is, the declarative speaker and interrogative addressee coordinations discussed above are both pragmatically congruous in that they are both results of the conversational presumption, whereas the declarative addressee and interrogative speaker are in opposition to this expected distribution. The declarative addressee can be split further into two groups in line with the distinctions presented in Table \ref{t:Discussion:OrigoShift}, being whether the addressee perspective is marked alongside that of the speaker, or is marked by itself. The latter, as is discussed in reference to Table \ref{t:Discussion:OrigoShift} above, appears most visibly in archetypal engagement marking. Here, it exists alongside the pragmatically congruous perspective of the speaker. Whether or not the concept of the origo is a particularly useful analytical tool in this context is discussed above, but in any case the otherwise incongruous reference to addressee perspective here can be seen as an extension of or addition to the perspective of the speaker, which is pragmatically congruous. This is contrasted with forms which reflect only the perspective of the addressee, such as some miratives, which are described as reflecting specifically the perspective of the addressee. These are particularly present in miratives which occur in narratives. The speaker of a narrative must have prior knowledge of the events being told, and as such cannot readily experience any of these events as unexpected. Rather, these miratives seemingly reflect the perspective of either the character within the narrative, or the addressee. 
\paragraph{Interrogative Speaker}
The final coordination of the origo is the reflection of the perspective of the speaker in interrogative constructions. While it has been argued above that there is an extent to which any reference to perspective is a reflection of that of the speaker, this in particular refers to direct reference that does not involve any projection of the perspective of the addressee. Non-shifting markers such as Meithei \textit{-ǰat} discussed in \ref{e:Discussion:MeitheiInter} fall into this category. While they do not follow the conversational presumption, the default expectation of reference to the speaker here seems to make these forms less analytically unusual.

\subsection{Analysing Perspectives}
The origo is a useful analytical tool, in particular for streamlining discussion of the deictic referent for epistemic meaning. That is, rather than specifying speaker or addressee, or constantly naming both, the term origo is useful to all possibilities. With suitable caveats, this can also be the case where there is more than one perspective marked by a given epistemic form, even though the conceptualisation of the origo does not lend itself so well to this situation. That being said, it is less compatible with the argument presented in Section \ref{s:Discussion:Mixed} on mixed systems that all perspectives potentially marked by an epistemic system will be assessed, even where the specific form selected does not reflect any aspect of one of these perspectives in its primary meaning. That is, there is a disconnect between the origo of the epistemic meaning marked by the form selected and the actual assessment and consideration of perspectives by the speaker, in which the selected form and its reference to perspective may be much more limited than the actual assessment of perspective by the speaker. As such, there are potentially two levels of attention to perspective to be considered when analysing epistemic marking: the reference to perspectives encoded by the form selected (that is, the specific conditions on the use of that form), and the assessment of perspective being undertaken by the speaker in the act of speaking. The prevalence of attention to the perspective of the addressee in the Gricean maxims as discussed above in Section \ref{s:Discussion:Perspective} and in the case studies discussed for mixed systems and social conditions in Sections \ref{ss:Discussion:MixedCases} and \ref{ss:Discussion:SocialCases} as well as the inherent necessity for the speaker to reference their own perspective mean that the former layer, that of the perspectives being considered by the speaker but not necessarily encoded by selected form, very often (if not always) involves the consideration of the perspective of speaker and addressee. This is to say that regardless of the meaning encoded by the selected form in terms of perspective, the speaker probably still considered their own perspective, as well as their assessment of that of their addressee. This does not mean that either layer here is more or less important or interesting than the other, and there is certainly scope for further research into the idea that speakers are always attentive to the perspective and state-of-mind of their addressee. Rather, it is important to note that these two layers are separate. The use of a form which only marks the perspective of the speaker, for instance a visual evidential marking that the speaker saw some information directly, does not mean that the speaker did not consider the state-of-min of the addressee. Rather, they may have done so and selected the most relevant form, which happens to not encode any meaning about the perspective of the addressee, or they may have done so and determined that the information at hand is relevant to the addressee and not previously known, and that this assessment informed their decision to make the statement at all, rather than their selection of a specific form.

\section{Functional Motivations for the Development of Epistemic Marking}\label{s:Discussion:Motivations}
The discussion of mixed systems in Section \ref{s:Discussion:Mixed} has shown that speakers can be attentive to a wide array of contextual or deictic factors covering both their own state of mind regarding the information being presented, as well as to the state-of-mind of their addressee even in situations where the chosen marking does not explicitly reference this addressee perspective. In Section \ref{s:Discussion:Social}, examples are given of the extension of these contextual factors to cover social features such as hierarchies and relationships, either together or to others. In all cases when discussing epistemic marking, these contextual factors only affect metapropositional meaning rather than the meaning of the proposition itself. That is, for a given topic, epistemic marking as a unified cross-linguistic category appears to be attentive to the knowledge of the speaker regarding the topic and their claim of authority over the information. This claim of authority is in turn informed and justified by a wide array of contextual factors which vary from language to language, including source of information, first hand experience, confidence, whether or not the information is new and surprising, and lastly an assessment of the relative relationship of the addressee to the topic at hand. This reduction of epistemic marking to a core function of making claims of epistemic authority does not, however, provide a clear functional motivation for the development and continued use of the system. While it may not be possible to completely prove any given motivation, such a motivation, or at least a discursive benefit to the use of epistemic marking, presumably does exist, and a brief discussion on the possibilities is worth presenting. 

In considering a possible functional motivation for epistemic marking, the limitation of this thesis to specifically grammaticalised marking of epistemic meaning is less applicable. While the grammaticalisation of epistemic meaning can be seen as a reification of such marking, and in particular the obligatory marking seen in many systems in the survey means that there is a much larger amount of epistemic information encoded in languages with grammaticalised marking, periphrastic strategies of encoding epistemic meaning are incredibly widespread\footnote{I hesitate here to claim anything as universal without having undertaken a much larger survey. That being said, when referring simply to the ability of a language to encode meaning about, for instance, source of information or level of confidence, it can be claimed that, assuming there is an equal ability to encode any thought in any language in some way, the ability to encode such meaning through some construction ought not be limited to any subset of languages.} and would sit atop the same functional motivations. This is to say that regardless of how a language encodes epistemic meaning, whether through grammaticalised marking, through periphrastic constructions, or simply lexically, the functional motivation for encoding such information is presumably universal.

It is difficult to discuss functional motivations for the encoding of epistemic meaning referencing only languages which have not been described in this regard. As such, examples and thoughts from English are also used in this discussion, despite clearly being neither a \lfam\ language, nor a language with any grammaticalised epistemic marking.

With this, epistemic marking, or more broadly the encoding of epistemic meaning, appears to establish a clear shared contextual foundation for communication to proceed more efficiently. In establishing a shared deictic ground of knowledge between speech act participants, the possiblity of a breakdown in communication and the need for repair structures is reduced. For an example in English\footnote{I assume in many other languages too, but am not prepared to make that claim of languages of which I am not a native speaker.}, a breakdown of communication can occur when information that is shared by both speech act participants is not presented as such, i.e. when information is reported to an addressee who is already aware of said information, necessitating a repair sequence and a delay in communication. Preemptive epistemic marking such as `as you know, ...' or `of course, ...' would have avoided this, as would any grammatical marking if it existed. Similar functions for the German modal particle \textit{ja} and its cognates in other Germanic languages have been noted by \citesA{Bergqvist2017}{Bergqvist2020b} Similar breakdowns in communication could occur with information presented with the appearance of higher epistemic authority than can rightly be claimed by the speaker, such as if information gained through hearsay was not marked with a lexical low confidence reportative marker in English like `apparently'. In such cases, while a breakdown in communication might not occur immediately, there is still a clear negative impact on cooperative communication from the dissemination of low confidence or reported information as if it were more supported than it is.

To an extent, this motivation of establishing a common contextual foundation between speech act participants is shared with other deictic functional domains. Recent research onto demonstrative selection in the \lfam\ language Phola (Ngwi-Burmese: PRC) suggests that many of the factors conditioning the selection and use of epistemic forms are shared with demonstratives, such as attention, shared knowledge, psychological proximity, as well as social access \cite{GonzalezPerez2023}. Additionally, their selection can also act to project either assumed or desired epistemic states of the addressee, reflecting their perspective through the perspective of the speaker in much the same way as was discussed in Section \ref{s:Discussion:Perspective}. These act to both establish common ground in terms of the awareness of referents, as well as to poll and subsequently align attention towards these referents. \citesA{EvansBergqvistSanRoque2018a}{EvansBergqvistSanRoque2018b} discuss engagement marking with demonstrative or nominal scope, in which demonstratives contrast not only by spatial proximity (to some deictic origo, typically but not necessarily the speaker) but also, as with verbal engagement marking, the epistemic access of both speaker and addressee to the referent, in this case in terms of either prior awareness or current attentional direction. Even more generally, spatial reference can be categorised as egocentric or otherwise, denoting the holder of the deictic origo as the speaker or some other reference point. If distinctions of definiteness are, at their core, a marker of prior awareness or attention by the addressee, then they too could be considered as marking engagement or addressee perspective in epistemic terms. This brief comment on the reflection of perspective in demonstratives is of course very shallow, and there is room for a much more in-depth study into the connections between epistemic and demonstrative deixis in \lfam\ languages that is outside the scope of this project and thesis.

The overlap in engagement marking between clausal and nominal or demonstrative scope suggests an alternative possible motivation behind markers such as the Eastern Geshiza non-shared information marker \textit{-go}. Example \ref{e:Discussion:GeshizaDialogue}, repeated as \ref{e:Discussion:GeshizaDialogue2} for ease of reference, shows a speaker referring to a girl unseen by his addressee. This is analysed above as a higher claim of epistemic authority over information that is held solely by the speaker, though it could potentially also be seen as a strategy to disambiguate between multiple women as a demonstrative would - this is the woman the addressee has not yet seen as opposed to the others. Whether or not such an extension of the analysis is reasonable is difficult to say without further insights into either the context of the dialogue or the language itself, but there is a more reasonable middle ground that the attention-orienting functions of demonstratives can be shared by epistemic marking, even though the marking is not directly governing a nominal referent.

\begin{exe}
    \ex\label{e:Discussion:GeshizaDialogue2}
    \begin{xlist}
        \ex\label{e:Discussion:GeshizaDialogue:A}
        \gll \textit{e} \textit{smæŋa} \textit{gæ-mdze} \textit{æ-lə} \textit{ŋuə-go} \\
        \textsc{dem} girl \textsc{adjz}-beautiful one-\textsc{clf.indef} \textsc{cop.3-nsi} \\
        \glt `That girl is beautiful'

        \ex\label{e:Discussion:GeshizaDialogue:B}
        \gll \textit{ŋuə-ræ}. \textit{ŋuə-ræ}. \textit{mdze-ræ}. \\
        \textsc{cop.3-sens} \textsc{cop.3-sens} be.beautiful-\textsc{sens} \\
        \glt `Yes, yes. She is beautiful.'
    \end{xlist}
    Eastern Geshiza \cite[rGyalrongic: PRC,][593]{Honkasalo2019}
\end{exe}

As is mentioned above, a key difference between demonstrative deixis and the epistemic deixis being discussed here is the location of the meaning within the speech act. That is, the demonstrative marking assists to disambiguate potential uncertainties in nominal reference by aligning the shared knowledge and attention of the speaker and addressee. This nominal reference is directly involved in the core meaning of a given proposition - its target is an actual part of the event or fact being relayed by the speaker. Epistemic meaning, on the other hand, while deictic and as such sharing many similarities in terms of conditioning factors and reference to perspective, broadly marks metapropositional meaning. This is meaning outside of the actual content of the proposition which will not change the actual event of fact being described if changed itself. That is, in a simple transitive sentence, a change in demonstrative will likely change the argument of the verb itself, fundamentally changing the meaning of the transitive sentence in terms of the event it describes. A shift from, for instance, a direct visual evidential to an inferential on, however, will not change the nature of the described event, but rather only marks a different relationship between the epistemic origo and event. This distinction between propositional and metapropositional meaning is, admittedly, occasionally blurred. In particular, while reportative evidentials mark information as being gained through hearsay but do not grammatically mark direct speech, quotatives do mark direct speech . As such, there is a question as to which even or piece of information is the proposition: the event of speaking now being quoted, of the information contained within the speech. If the event of speaking being quoted is seen as the proposition, then the quotative (marking the direct speech) arguably is affecting the meaning of the proposition itself as it is marking that such a speech event (the proposition) occurred. \citeA[64]{Aikhenvald2004} contrasts the quotative and reportative in terms of reference to an overt or named source. That is, the reportative notes information as received from sme third party source, but does not name said source. Quotative marking, on the other hand, specifically names the source of information often in the form of a direct quote. The specification of the source of the information is in part why quotatives canbe seen as affecting propositional meaning, as they introduce a new subject or agent to the speech act, adding a new nominal referent to the clause and in turn the proposition. 

As such, while demonstratives and epistemic markings seem to share the core functional motivation of establishing shared ground between speech act participants, the specific methods of achieving this, either through disambiguation of referents or the avoidance of communicative breakdowns by the establishment of a shared knowledge by both speech act participants as to their respective relationships to the proposition respectively, differ. It is likely that this shared functional motivation could be further extended to other areas of meaning, though outside the scope of this thesis. This argument follows similar suggestions in regards to engagement marking in Duna (Trans-New-Guinea: Papua New Guinea) in \citeA{SanRoque2015}.


\section{Conclusion}
This chapter has taken the data and initial typological observations presented in Chapter \ref{c:Description} and begun to draw a number of theoretical conclusions on the use of epistemic marking and the reflection of perspective in \lfam\ languages, and begins to extend these conclusions to speech and language more broadly. Section \ref{s:Discussion:Intro} began by introducing the proposal that epistemic marking can be seen as a valid cross-linguistic functional supercategory, which was then argued with specific reference to the mixed systems described in Chapter \ref{c:Description} (Section \ref{s:Discussion:Mixed}) and to the consideration of social factors in the use of epistemic marking (Section \ref{s:Discussion:Social}). In terms of mixed systems, it was argued that there is a cohesive functional domain across the `traditional' cross-linguistic categories of epistemic modality, evidentiality, egophoricity, mirativity, and engagement in which they all reflect the relationship of the speech act participants to the information at hand, and can be described as existing along a gradient reflecting the strength of the claim made by the speaker over their own epistemic authority (or said authority granted to the addressee). It was also argued that, in particular regarding paradigmatic mixed systems, the contrastive meanings of the various forms in an epistemic system mean that the speaker will consider the conditions of every possible form, meaning the actual epistemic consideration of the speaker is much wider than whichever function is ultimately selected to be marked. Section \ref{s:Discussion:Social} presented a number of epistemic systems in which the selection of forms was conditioned not only by the relationship between the speech act participants and the information at hand, but also by the relationships between the speech act participants and external referents and each other in social terms. This is suggested to be more widespread than current literature suggests, given the challenges is observing such contrasts and conditions in field work. In consideration of the mixed systems and social conditions discussed, Section \ref{s:Discussion:Perspective} argued that the assessment of the perspective of the addressee by the speaker, the actuality of any reference to the perspective of the addressee, is also very widespread if not necessary in any cooperative conversation. This is, however, contrasted with the actual epistemic functions that are selected to be marked and the perspective they reflect, suggesting a two-tier approach to the analysis of perspective reference: the assessment of the perspectives of the speaker, addressee, and potentially narrative characters or third parties by the speaker, and the persective(s) referenced by the specific form they ultimately select. Lastly, Section \ref{s:Discussion:Motivations} discussed the possible functional motivations for the enduring presence of grammaticalised epistemic marking across the \lfam\ family and apparent spread of the marking (discussed in detail in Chapter \ref{c:History}). It suggests that there is a shared functional motivation across epistemic marking and other deictic functional domains such as demonstratives, that of establishing shared ground between speech act participants to aid in communication by eliminating possible ambiguities or disagreements, though notes some key differences between demonstratives and epistemic marking, specifically in terms of effects of the marking on the propositional and metapropositional meaning of a speech act.