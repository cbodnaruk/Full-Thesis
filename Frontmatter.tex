\begin{center}
This is to certify that to the best of my knowledge, the content of this thesis is my own work. This thesis has not been submitted for any degree or other purposes.

I certify that the intellectual content of this thesis is the product of my own work and that all the assistance received in preparing this thesis and sources have been acknowledged.

\vspace{3cm}
Carl Bodnaruk

\end{center}
\newpage
\begin{center}
    \large
    \textbf{Abstract}
\end{center}
Grammaticalised marking of epistemic meaning---that is meaning conditioned by the relationship between speech act participants and the information at hand---is widespread across the Trans-Himalayan (Sino-Tibetan, Tibeto-Burman) language family, and more generally the Himalayan region. This thesis provides a typological overview of epistemic marking across the Trans-Himalayan language family by sampling previous literature and analysis on 68 languages across the family in terms of an updated version of van Driem's (2014) Fallen Leaves model of the phylogeny of the family. In doing so, it develops a number of typological categories for the description of epistemic marking in terms of form and function in cross-linguistic terms, arguing that epistemic bases within a system can be meaningfully classified along a gradient with respect to the strength of claim over epistemic authority being made by the speaker. These typological observations are further utilised to draw more broadly applicable conclusions regarding the theoretical and cognitive foundations of epistemic marking in language. Specifically, this thesis argues for a functional supercategory of epistemic marking which extends epistemicity as described by Boye (2012) to include not only the functional categories of epistemic modality and evidentiality, but also egophoricity (per San Roque et al. 2018), mirativity (per DeLancey 1997, 2012), and engagement (per Evans et al. 2018a,b), and suggests that this marking, and in fact cooperative conversational in general, is inherently attentive to the perspectives of both speech act participants both in terms of relationships to the information at hand, as well as potentially in social terms. Finally, it compares the typological classifications of epistemic marking across the 68 languages sampled and extralinguistic factors, such as economic and sociocultural history, to investigate the possible routes for the development of such a widespread epistemic marking sprachbund, finding that the diachronic development of epistemic marking across the Trans-Himalayan family appears to have spread areally rather than vertically from Proto-Trans-Himalayan, but that any solid conclusions regarding the exact nature of this diffusion are limited by a lack of contemporary knowledge surrounding historical population movements, levels of contact, and records of language itself.
\newpage
\begin{center}
    \large
    \textbf{Acknowledgements}

\end{center}
\normalsize
As with any PhD project, none of this would have been possible without the enormous support of the people around me, both academic and personal. First and foremost, I of course need to thank my supervisors, Dr Gwendolyn Hyslop and Dr Lila San Roque, for their incredible and unceasing support over the last few years. I cannot imagine a better two supervisors, and I cannot thank them both enough for going above and beyond the expectations on them, meeting me so often, being happy to answer questions and give advice at any hour, advocating for me to administration, and providing me with so many other opportunities. I also need to thank my colleagues and fellow PhD students in the Linguistics department for their support and camaraderie, in particular the Tibeto-Burman Interest Group for hearing out my ideas and giving such great feedback in the lead up to conferences, Lisa, Carly, Kathryn and Gus, for your support throughout our teaching and marking trials and tribulations, and my colleages in the NTEU, for fighting for me and with me.

While my time in Bhutan comprises only a small part of this thesis, it was definitely one of the most impactful parts of this project. With that, I need to thank the Lhohpu community for their willingness to share their language with me, in particular Karma Dorji Namba and Tshering Wangchuk Namba, and look forward to continuing to work with them and their community in the coming years. I also need to thank the Centre for Bhutan and GNH Studies for hosting me, and more broadly for their support on research on endangered languages in Bhutan.

I also wish to acknowledge the assistance of Dr Maurice Quirk, who provided feedback on grammar and style on the final version of this thesis.

I have, of course, relied on the support of my incredible friends throughout my academic journey, who will have to accept a simple list of names in the interest of space. My RPG groups, Ed, Sean, Zoe, Seb, Eddie, Aled, Izzy, Chis, Pace and Alexi, who have given me so many fond memories and told so many wonderful stories, the Tuesday night bouldering crew, Remi, Mo, Pei Yin, and the many other communities who have given me an escape from the insurmountable task (or so it felt at times) of writing this thesis.
There are of course others who will have slipped through the cracks of these acknowledgements, but for them I am equally as thankful (and apologetic).

Lastly, I want to thank my family for their endless support. Ally and Karina, for their advice and attempts at understanding the topic, Drew, for his encouragement and unwavering patience with me, and of course, my parents, for their support and belief in me throughout my academic journey so far, and into the future.