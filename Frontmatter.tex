\begin{center}
This is to certify that to the best of my knowledge, the content of this thesis is my own work. This thesis has not been submitted for any degree or other purposes.

I certify that the intellectual content of this thesis is the product of my own work and that all the assistance received in preparing this thesis and sources have been acknowledged.

\vspace{3cm}
Carl Bodnaruk
\end{center}
\newpage
\begin{center}
    \large
    \textbf{Abstract}
\end{center}
Grammaticalised marking of epistemic meaning, that is meaning conditioned by the relationship between speech act participants and the information at hand, is widespread across the Trans-Himalayan (Sino-Tibetan, Tibeto-Burman) language family, and more generally the Himalayan region. This thesis provides a typological overview of epistemic marking across the Trans-Himalayan language family by sampling previous literature and analysis on 68 languages across the family in terms of an updated version of van Driem's (2014) Fallen Leaves model of the phylogeny of the family. In doing so, it develops a number of typological categories for the description of epistemic marking in terms of form and function in cross-linguistic terms, arguing that epistemic bases within a system can be meaningfully classified along a gradient with respect to the strength of claim over epistemic authority being made by the speaker, as well as using these typological observations to draw more broadly applicable conclusions regarding the theoretical and cognitive foundations of epistemic marking in language. Specifically, it argues for a functional supercategory of epistemic marking which extends epistemicity as described by Boye (2012) to include not only the functional categories of epistemic modality and evidentiality, but also egophoricity (per San Roque et al. 2018), mirativity (per DeLancey 1997, 2012), and engagement (per Evans et al. 2018a,b), and suggests that this marking, and in fact cooperative conversational in general, is inherently attentive to the perspectives of both speech act participants both in terms of relationships to the information at hand, as well as potentially in social terms. Finally, it compares the typological classifications of epistemic marking across the 68 languages sampled and extralinguistic factors such as economic and sociocultural history to investigate the possible routes for the development of such a widespread epistemic marking sprachbund, finding that the diachronic development of epistemic marking across the Trans-Himalayan family appears to have spread areally rather than vertically from Proto-Trans-Himalayan, but that any solid conclusions regarding the exact nature of this diffusion are limited by a lack of contemporary knowledge surrounding historical population movements, levels of contact, and records of language itself.
\newpage
\begin{center}
    \large
    \textbf{Acknowledgements}
\end{center}